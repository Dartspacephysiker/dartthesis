%------------------------------------------------------------------------%
% Example appendix section for the thesis template.                      %
%                                                                        %
%      Author: Gregory Alexander Feiden                                  %
%   Institute: Dartmouth College                                         %
%        Date: 2014 May 17                                               %
%                                                                        %
%     License: Beerware (revision 42)                                    %
%              ----------------------                                    %
%              Gregory Feiden wrote this file. As long as you retain     %
%              this notice you can do whatever you want with this code.  %
%              If we meet some day and you think this code is worth it,  %
%              you can buy me a beer in return.                          %
%                                                                        %
%------------------------------------------------------------------------%

% \chapter{Uncertainty Analysis}
\chapter{Moment Uncertainty Due to Counting Statistics}
\label{app:A}

  After minimization of systematic error (via laboratory and in situ
  calibration, for example), an important source of uncertainty for
  particle detectors is that arising from counting
  statistics. The strategy for estimating the uncertainty
  associated with moments derived from a sampled particle distribution
  $f ( \mathbf{v} ) $ has been generation of statistics of each moment based
  on Monte Carlo simulation and sampling of $f(\mathbf{v})$ \citep[e.g.][]{Paschmann1998,Gershman2013}
  However, \citet{Gershman2015} have produced an analytical framework
  for calculation of both the uncertainty in moments of $f (
  \mathbf{v} ) $ and the the covariance between moments, by which the
  uncertainty in arbitrary, differentiable functions of moments are
  obtainable via standard techniques of linearized uncertainty
  analysis. In our application these functions are field-aligned
  current density $j_\parallel = \langle n v_\parallel \rangle$ and
  average temperature $T = \big ( \langle P_{\parallel} \rangle + 2
  \langle P_{\perp} \rangle \big ) \big / 3 \langle n \rangle $, where
  gyrotropy is assumed in the expression for $T$. (FAST ion and
  electrons ESAs measure only one direction perpendicular to the
  geomagnetic field.)

  We first derive expressions for $\sigma_{j_\parallel}$ and
  $\sigma_T$ as functions of the moments of $f(\mathbf{v})$, as well
  as the related uncertainties and covariances. We then summarize
  results of \citet{Gershman2015}, who develop analytic expressions
  for moment uncertainties and covariances of an arbitrary sampled
  distribution function. Their work, in turn, allows
  $\sigma_{j_\parallel}$ and $\sigma_T$ to be expressed analytically;
  final analytic forms for $\sigma_T$ and $\sigma_{j_\parallel}$ are
  used to calculate the error bars shown in Figures~\ref{ch6:Fig1} and
  ~\ref{ch6:Fig4}.

  Following \citet{Gershman2015}, let $W$ be a differentiable function
  of plasma moments $\langle n A_i \rangle $; the linearized
  uncertainty $\sigma_W$ may be expressed % [cite cite]
  \begin{equation} \label{appa:eqLinUnc} \sigma_W^2 =
    \sum\limits_{i}\sum\limits_{j} \bigg ( \frac{ \partial W}{\partial
      A_i} \bigg ) \bigg ( \frac{ \partial W}{\partial A_j} \bigg )
    \sigma_{A_i,A_j}.
  \end{equation}
  Uncertainty in $j_{\parallel} $ and $T$ are
  % \begin{linenomath*}
    \begin{align}
      \begin{split}
        \sigma_{j_\parallel}^2 &= \langle v_\parallel \rangle
        \sigma_{\langle n \rangle}^{2} + \langle n \rangle
        \sigma_{\langle v_{\parallel} \rangle }^{2} + \sigma_{\langle n
          \rangle,\langle v_{\parallel} \rangle}.
        \\
        \sigma_T^2 &= \frac{1}{9 \langle n \rangle^2} \bigg [
        \sigma_{\langle P_{\parallel} \rangle}^2 + 4 \bigg (
        \sigma_{\langle P_{\parallel} \rangle,\langle P_{\perp} \rangle}
        + \sigma_{\langle P_{\perp} \rangle }^2 \bigg ) \bigg ] -
        \frac{2 T}{ 3 \langle n \rangle^2} \bigg [ \sigma_{\langle
          P_{\parallel} \rangle,\langle n \rangle} + 2 \sigma_{\langle
          P_{\perp} \rangle ,\langle n \rangle} \bigg ] + \bigg (
        \frac{T}{\langle n \rangle} \bigg )^2 \sigma_{\langle n
          \rangle}^2.
      \end{split}
    \end{align}
  % \end{linenomath*}

  Calculation of moments of $f(\mathbf{v})$ in these expressions is
  straightforward. Calculation of moment uncertainties and covariances
  from $f(\mathbf{v})$, however, requires the following assumptions:

  \begin{enumerate}

  \item The sampling of each phase space volume is unique. For FAST
    ESAs, which sample energy and pitch angle, this assumption means,
    for example, that there is no overlap between regions of phase
    space sampled by each energy-angle detector bin, and that there is
    no crosstalk.

  \item The sampled phase space density $f(\mathbf{v})$ corresponds to
    a number of counts $N(\mathbf{v}) = f(\mathbf{v}) \Delta
    V(\mathbf{v})\Delta X(\mathbf{v})$, where $\Delta V(\mathbf{v})$
    and $\Delta X (\mathbf{v})$ are respectively the phase space
    velocity and position volumes sampled by FAST ESAs, and
    $N(\mathbf{v})$ is a Poisson-distributed random variable.

  \end{enumerate}

  The covariance between moments $\langle n A_i \rangle$ and $\langle
  n A_j \rangle$ is $\sigma_{\langle n A_i \rangle,\langle n A_j
    \rangle} = E \big [\langle n A_i \rangle\langle n A_j \rangle \big
  ] - E \big [\langle n A_i \rangle \big ] E \big [\langle n A_j
  \rangle \big ]$, where
  % \begin{linenomath*}
    \begin{align}
      \begin{split}
        E \big [\langle n A_i \rangle\langle n A_j \rangle
        \big ] &= \iiint \mathbf{d}^3 \mathbf{v} A_i (\mathbf{v}) \iiint \mathbf{d}^3 \mathbf{v'} A_j (\mathbf{v'}) E \big [f(\mathbf{v}) f(\mathbf{v'}) \big ];
        \\
        E \big [\langle n A_i \rangle \big ] E \big [\langle n A_j
        \rangle \big ] &= \iiint \mathbf{d}^3 \mathbf{v} A_i (\mathbf{v}) \iiint \mathbf{d}^3 \mathbf{v'} A_j (\mathbf{v'}) E \big [f(\mathbf{v}) \big ] E \big [ f(\mathbf{v'}) \big ].
      \end{split}
    \end{align}
  % \end{linenomath*}
  It follows that $ \sigma_{\langle n A_i \rangle,\langle n A_j
    \rangle } = \iiint \mathbf{d}^3 \mathbf{v} A_i (\mathbf{v}) \iiint
  \mathbf{d}^3 \mathbf{v'} A_j (\mathbf{v'})
  \sigma_{f(\mathbf{v}),f(\mathbf{v'})}$; that is, the covariance
  between any two moments of $f(\mathbf{v})$ depends on the covariance
  between the points in phase space $\mathbf{v}$ and
  $\mathbf{v'}$. \citet{Gershman2015} show that if
  $\sigma_{f(\mathbf{v}),f(\mathbf{v'})}$ is written in terms of the
  correlation between regions of phase space,
  % \begin{linenomath*}
    \begin{equation}
      \sigma_{f(\mathbf{v}),f(\mathbf{v'})} = \sigma_{f(\mathbf{v})} \sigma_{f(\mathbf{v'})}r(\mathbf{v},\mathbf{v'}),
    \end{equation}
  % \end{linenomath*}
  the first assumption implies $r(\mathbf{v},\mathbf{v'}) \approx
  \delta_{\mathbf{v v'}}$, while the second assumption implies that
  the uncertainty of the sampled phase space density is
  $\sigma_{f(\mathbf{v})} = f (\mathbf{v}) \big / \sqrt{N
    (\mathbf{v})}$. Thus
  % \begin{linenomath*}
    \begin{equation}
      \sigma_{f(\mathbf{v}),f(\mathbf{v'})} \approx
      \frac{f^2(\mathbf{v})}{N(\mathbf{v})}, 
    \end{equation}
  % \end{linenomath*}
  which leads to the analytic expression
  % \begin{linenomath*}
    \begin{equation} \label{appa:eqFin}
      \sigma_{\langle n A_i \rangle,\langle n A_j\rangle } \approx \iiint \big ( \mathbf{d}^3 \mathbf{v} \big )^2 A_i (\mathbf{v}) A_j (\mathbf{v}) \frac{f^2(\mathbf{v})}{N(\mathbf{v})} = \Bigg \langle n A_i A_j \big (\mathbf{d}^3 \mathbf{v} \big ) \frac{f(\mathbf{v})}{N(\mathbf{v})} \Bigg \rangle,
    \end{equation}
  % \end{linenomath*}
  where the RHS of \ref{appa:eqFin} represents $\sigma_{\langle n A_i
    \rangle,\langle n A_j\rangle }$ as a moment of $f(\mathbf{v})$.

\bibliographystyle{agufull08}
\bibliography{refsa}
