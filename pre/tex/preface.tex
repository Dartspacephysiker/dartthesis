%------------------------------------------------------------------------%
% Preface for thesis.                                                    %
%                                                                        %
%      Author: Gregory Alexander Feiden                                  %
%   Institute: Dartmouth College                                         %
%        Date: 2014 May 17                                               %
%                                                                        %
%     License: Beerware (revision 42)                                    %
%              ----------------------                                    %
%              Gregory Feiden wrote this file. As long as you retain     %
%              this notice you can do whatever you want with this code.  %
%              If we meet some day and you think this code is worth it,  %
%              you can buy me a beer in return.                          %
%                                                                        %
%------------------------------------------------------------------------%
%\vspace*{1in}

{\Huge \bf Acknowledgments} \\

Would I be too humdrum to offer the usual thanks to my mother and father? If you
think so maybe your wrist ought to be slapped. I certainly \emph{do} thank my
Momma, as I am wont to call her, for being my first and foremost friend among
friends in mortality. I likewise thank my Poppy, for giving a lifelong example
of humility sufficient to keep my mother from taking his life and to provide a
``Father's blessing'' as often as I have found myself in town. I'll take it a
step further and thank each of my six siblings, whose gifts are known to them
and to me.

Thanks to my many academic parents. Among them are my adviser (I always found
that spelling odd, but whatever---Dartmouth's style guide, not mine) Jim for
tolerating my presence over the course of years, for frequently providing me a
doorway to the joy of exploring Gaia over much more of her face than I had
previously known, and for shaping the way I approach my discipline.

Kristina Lynch, for making herself available almost at a moment's notice despite
crushing timelines associated with rockets, teaching, and the rest of life---the
parts that matter most, not the science---that gets ignored when one is locked
in ``academic mode.''

Dave Kieda, whose association even now I so appreciate, for giving me a timely
and timeless word of encouragement: ``Good luck finding your inner compass.''

Colin Inglefield, for introducing me to the magic of dungeon laboratories in
unoccupied basements and to ``the famous'' one-over-sine-to-the-fourth
dependence in the only differential cross section I can vaguely remember,
helping me land a sweet scholarship, and teaching me about ``three-sigma''
personalities sitting at the back of the class.

Adam Johnston, for introducing me to Kuhn, Popper, the spirit of free inquiry,
and for believing that someday my writing could become more than garbage.

John Armstrong, for setting a snare in Introductory Astronomy that is a direct
cause in the chain of events leading to the present work, and for sagely
instructing my intro physics cohort that $p =$~1, where $p$ is the probability
that the universe would turn out as it has.

Stephan Le Bohec, for introducing me to \emph{galette des rois} (yes I did
get to wear the crown that year!) and French mannerisms that, to this day, I
love and think of nearly daily.

Thanks to dear Spencer Kirk, my intellectual father. I never knew that it was in
any Spencer to be so sharp, so business-like, so kind, all while maintaining
laser-sharp focus.

A salute and a nod to Micah, who introduced the necessary seed of doubt, as a
sower of tares, that graduate school ``is academic hazing.'' Though sight was
lacking aforetime, I now see, true yokefellow.

A word to Matt Broughton, thou who goest before: you remain the man I turn to
when statistics chops are the order, and when it is time to be reminded that
\textit{Finnegans Wake} exists.

Few graduate students in Wilder Hall between 2010 and 2015 missed out on
interacting with the Friend of the Department and \textit{scientifique
  d\'{e}bonnaire}, Phil Fernandes. I am among that lucky majority who have been
inspired and encouraged by Phil's empathy and candor. I am also among the
luckier and decidedly fewer who have had the pleasure of knowing the rest of
Phil's family, Jackie, Bia, and Angelica. Thank you for letting me be a small
part of your lives, and for seeing me through a few nasty spells.

Mike, I look forward to seeing you again. I hope we'll yet have occasion to talk
birds, Faraday rotation, the banality of space physics, the Roast Beef Loop, and
the sky edition of \textit{Thrasher}.

I once was witness to the granting of immortality---my own, in fact---with a
single stroke of breath produced by David McGaw as we stood contemplating the
meaning of attention deficit disorder: ``I don't have a problem paying
attention; I'm keeping track of a million things at once!''  Talk about a power
statement! With the suddenness of a thunderclap I ascended from slight
disadvantage relative to my peers to superiority, no less than \emph{one million
  times over!}  It has been a self-muttered refrain and consolation, and thanks
are duly accorded.

To use that insipid turn of phrase, I would be remiss if I failed to likewise
accord thanks to a SWAT team appointed by the fates and bent on seeing that I
survive Dartmouth. These are primarily the angelic (no hyperbole) Lauren Costa,
my dear and faithful (though he may not reciprocate these adjectives) friend,
Yipeng Shi, and another of the [Art Blakey's Jazz] Messengers, Ayobami
Olufadeji. I am happy at the thought of each of you. Why have you put up with
me? Thank you.

But to whom is reserved the dedication? Above all, \emph{to the One who is
  fairer and purer than both meadow and woodland,} in whom I shall never want.

\newpage

%---BEGIN INSPIRATION QUOTE 
\begin{flushright}
\begin{minipage}[]{0.55\linewidth}
% {\small \emph{To the One who is fairer and purer than both meadow and woodland}}
    \begin{flushright}
      Han vil med glede gi dere alt dette hvis dere bare gir ham f\o rsteplassen
      i livene deres. \\{\small \emph{--- Den Gamle av dager, den hovedhj\o
          rnesten} }
    \end{flushright}
\end{minipage}
\end{flushright}
%\vspace{\baselineskip}
%---END INSPIRATIONAL QUOTE

{\Huge \bf Preface} \\

``I bet you've never heard a conversation like this before.'' That's
where you're right. I haven't---as the pauper's mite before the
throne, or as the grass blade's glory at the door of the
% oven. And what bearing, so trite a subject in the face of so grave a
oven. And what bearing, such trifling in the face of so grave a matter?  The
years tell. One before remarked (and we marked well, or else did well to mark)
that a great ship is turned by a very small helm. Doubtless---though somehow not
comprehended, world without end. They will not have forgotten the words of Comte
Anatole de Montesquiou-F\'{e}zensac:

\begin{quote} 
\begin{spacing}{1.1}
Comme ils sont provocants! Comme ils sont fiers toujours! \\
Comme on ose r\'{e}gner sur nos sorts et nos jours! \\
\\
% \^{O} la mortelle injure! La cadence est moins lente! \\
% Et la chute plus s\^{u}re! Nous rabattrons bien leur caquets! \\
% Nous serons bient\^{o}t leurs laquais! \\
% Qu'ils sont laids! Chers minois! \\ 
% Qu'ils sont fols! (Airs coquets!) \\
% \\
\dots \\
\\
Adieu donc et bons jours aux tyrans de nos coeurs! \\
\end{spacing}
\end{quote}
Which shall be had of another's hand; we lie down in sorrow.

So while it is called, Today: it is heightless and depthless, this hope.


