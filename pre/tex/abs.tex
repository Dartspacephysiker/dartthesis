%------------------------------------------------------------------------%
% Abstract formatting for a thesis.                                      %
%                                                                        %
%      Author: Gregory Alexander Feiden                                  %
%   Institute: Dartmouth College                                         %
%        Date: 2014 May 17                                               %
%                                                                        %
%     License: Beerware (revision 42)                                    %
%              ----------------------                                    %
%              Gregory Feiden wrote this file. As long as you retain     %
%              this notice you can do whatever you want with this code.  %
%              If we meet some day and you think this code is worth it,  %
%              you can buy me a beer in return.                          %
%                                                                        %
%------------------------------------------------------------------------%

\vspace*{1in}
{\Huge \bf Abstract} \\

The magnetosphere-ionosphere (M-I) transition region is the several
thousand--kilometer stretch between the cold, dense and variably
resistive region of ionized atmospheric gases that begins some tens of
kilometers above the terrestrial surface, and the hot, tenuous, and
highly conductive plasmas that interface with the solar wind at higher
altitudes. The M-I transition region is therefore the site through
which magnetospheric conditions, which are strongly susceptible to
dynamics in the solar wind, are communicated to ionospheric plasmas,
and vice versa.

We systematically study the influence of solar wind conditions and
geomagnetic storms on energy input and particle precipitation in the
M-I transition region, emphasizing the role of inertial \Alf waves
both as a preferred mechanism for dynamic (as opposed to static)
energy transfer and particle acceleration between the magnetosphere
and the ionosphere, and as a low-altitude manifestation of
high-altitude dynamic interaction between the solar wind and the
magnetosphere. We examine the influence of inertial \Alf waves via the
electromagnetic and electron energy deposition, electron
precipitation, and ion outflow coincident with these waves as observed
by the Fast Auroral SnapshoT Explorer (FAST) satellite. 

We investigate the effect of storms on these measures via a superposed
epoch analysis, and find that dayside and nightside occurrences rates
of \Alf waves, as well as the concomitant rates of energy deposition,
electron precipitation, and ion upflow, rapidly and dramatically
increase at storm sudden commencement. We derive high-latitude
distributions of these measures as a function of storm phase, showing
that storm main and recovery phase correspond to strong variation in
each signature of \Alfic activity, particularly on the dayside at and
in the vicinity of the cusp as well as premidnight. From calculation
of global rates of \Alfic energy deposition, electron precipitation,
and ion outflow in each hemisphere, we explicitly demonstrate that
storm main and recovery phases together account for more than 65\% of
global \Alfic energy deposition and electron precipitation, and more
than 70\% of the coincident ion outflow, even though the combined
occurrence rate of storm main and recovery phase is statistically only
33\%.

We also carry out an investigation of interplanetary magnetic field
(IMF) control of inertial \Alf wave activity in the M-I transition
region. Our investigation takes the form of an observational test of
the Lyon-Fedder-Mobarry global MHD simulations conducted by
\citet{Zhang2014}, which predict that southward IMF
conditions lead to enhanced generation of power in the magnetotail
that is then manifest as bursty bulk flows and \Alfic power on the
nightside. We confirm this prediction, as well as the simulation
prediction that in the Northern Hemisphere, dayside \Alfic power is
preferentially enhanced in the prenoon sector under predominantly
duskward IMF conditions. The observed and predicted prenoon
enhancement in \Alfic power stands in contrast with the established
response \citep[e.g.,][]{Zhou2000} to duskward IMF exhibited
by direct-entry precipitation, which becomes enhanced in the postnoon
sector. This situation reverses under dawnward IMF conditions. Though
the effect is present in both simulation and observation, the
mechanism for dayside generation of \Alfic power remains unclear. 

We then compare \Alfic Poynting flux distributions under several IMF
orientation to corresponding distributions of broadband precipitation,
identified as such using a version of the \citet{Newell2009}
scheme for categorizing electron precipitation adapted for FAST
instrumentation. We incidentally find that $\sim$95\% of
cusp-region inertial \Alf wave events are associated with
precipitation that is categorized by the \citet{Newell2009}
scheme as diffuse, purely on the basis of the criterion imposed on
minimum average energy ($E_{av} \geq$~300~eV) for cusp-region broadband
precipitation. This suggests that the prevailing definition of diffuse
precipitation, as it applies to the cusp region, needs further
qualification. 

Last, we present FAST observations on the nightside of discrete,
field-aligned precipitation with corresponding differential number
flux spectra that are much better described by a kappa distribution
than by a Maxwellian distribution. With only a few exceptions in the
literature discrete precipitation has been theoretically treated and
modeled as with Maxwellian distributions, while the spectral shapes of
discrete precipitation that we present have often been viewed as a
feature of diffuse precipitation. Guided by theoretical work performed
by \citet{Dors1999}, we examine the implications of
these observations for the commonly used Knight Relation.


% Jack Eddy version (January 2017)
% 
% The magnetosphere-ionosphere (M-I) transition region is the several
% thousand--kilometer stretch between the cold, dense and variably
% resistive region of ionized atmospheric gases that begins some tens of
% kilometers above the terrestrial surface, and the hot, tenuous, and
% highly conductive plasmas that interface with the solar wind at higher
% altitudes. The M-I transition region is therefore the site through
% which magnetospheric conditions, which are strongly susceptible to
% dynamics in the solar wind, are communicated to ionospheric plasmas,
% and vice versa.  We systematically study the influence of solar wind
% conditions and geomagnetic storms on energy input and particle
% precipitation in the M-I transition region, emphasizing the role of
% inertial \Alf waves both as a preferred mechanism for dynamic (as
% opposed to static) energy transfer and particle acceleration between
% the magnetosphere and the ionosphere, and as a low-altitude
% manifestation of high-altitude dynamic interaction between the solar
% wind and the magnetosphere. We examine the influence of inertial \Alf
% waves via the electromagnetic and electron energy deposition, electron
% precipitation, and ion outflow coincident with these waves as observed
% by the Fast Auroral SnapshoT Explorer (FAST) satellite.  We
% investigate the effect of storms on these measures via a superposed
% epoch analysis, and find that dayside and nightside occurrences rates
% of \Alf waves, as well as the concomitant rates of energy deposition,
% electron precipitation, and ion upflow, rapidly and dramatically
% increase at storm sudden commencement. We also derive high-latitude
% distributions of these measures as a function of storm phase, showing
% that storm main and recovery phase correspond to strong variation in
% each signature of \Alfic activity, particularly on the dayside at and
% in the vicinity of the cusp as well as premidnight. From calculation
% of global rates of \Alfic energy deposition, electron precipitation,
% and ion outflow in each hemisphere, we explicitly demonstrate that
% storm main and recovery phases together account for more than 65\% of
% global \Alfic energy deposition and electron precipitation, and more
% than 70\% of the coincident ion outflow, even though the combined
% occurrence rate of storm main and recovery phase is statistically only
% 33\%.  We also carry out an investigation of interplanetary magnetic
% field (IMF) control of inertial \Alf wave activity in the M-I
% transition region. Our investigation takes the form of an
% observational test of the Lyon-Fedder-Mobarry global MHD simulations
% conducted by \textsl{Zhang et al.} [2014], which predict that
% southward IMF conditions lead to enhanced generation of power in the
% magnetotail that is then manifest as bursty bulk flows and \Alfic
% power on the nightside. We confirm this prediction, as well as the
% simulation prediction that in the Northern Hemisphere, dayside \Alfic
% power is preferentially enhanced in the prenoon sector under
% predominantly duskward IMF conditions. The observed and predicted
% prenoon enhancement in \Alfic power stands in contrast with the
% established response [e.g., \textsl{Zhou et al.}, 2000] to duskward
% IMF exhibited by direct-entry precipitation, which becomes enhanced in
% the postnoon sector. This situation reverses under dawnward IMF
% conditions. Though the effect is present in both simulation and
% observation, the mechanism for dayside generation of \Alfic power
% remains unclear.  We then compare \Alfic Poynting flux distributions
% under several IMF orientation to corresponding distributions of
% broadband precipitation, identified as such using a version of the
% \textsl{Newell et al.} [2009] scheme for categorizing electron
% precipitation adapted for FAST instrumentation. We incidentally find
% that $\approx$95\% of cusp-region inertial \Alf wave events are
% associated with precipitation that is categorized by the
% \textsl{Newell et al.} [2009] scheme as diffuse, purely on the basis
% of the criterion imposed on minimum average energy ($E_{av}
% \geq$~300~eV) for cusp-region broadband precipitation. This suggests
% that the prevailing definition of diffuse precipitation, as it applies
% to the cusp region, needs further qualification.  Last, we present
% FAST observations on the nightside of discrete, field-aligned
% precipitation with corresponding differential number flux spectra that
% are much better described by a kappa distribution than by a Maxwellian
% distribution. With only a few exceptions in the literature discrete
% precipitation has been theoretically treated and modeled as with
% Maxwellian distributions, while the spectral shapes of discrete
% precipitation that we present have often been viewed as a feature of
% diffuse precipitation. Guided by theoretical work performed by
% \textsl{Dors and Kletzing} [1999], we examine the implications of
% these observations for the commonly used Knight Relation.
