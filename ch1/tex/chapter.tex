%------------------------------------------------------------------------%
% Example chapter for the thesis template.                               %
%                                                                        %
%      Author: Gregory Alexander Feiden                                  %
%   Institute: Dartmouth College                                         %
%        Date: 2014 May 17                                               %
%                                                                        %
%     License: Beerware (revision 42)                                    %
%              ----------------------                                    %
%              Gregory Feiden wrote this file. As long as you retain     %
%              this notice you can do whatever you want with this code.  %
%              If we meet some day and you think this code is worth it,  %
%              you can buy me a beer in return.                          %
%                                                                        %
%------------------------------------------------------------------------%

\chapter{Introduction}
\label{chp:1}

%---BEGIN INSPIRATION QUOTE 
% \begin{flushright}
% \begin{minipage}[]{0.55\linewidth}
%     \begin{flushright}
%       I don't have ADD; I'm keeping track of a million things at once!  \\{\small \emph{--- David McGaw} }
%     \end{flushright}
% \end{minipage}
% \end{flushright}
%\vspace{\baselineskip}
%---END INSPIRATIONAL QUOTE

\section{A word before things pick up}

If the pursuit of knowledge fills the soul with joy, then the
communication of it brings the soul to overflowing\footnote{We pass in
  piety over the many and ongoing instances in which respectable
  science has evolved into bloodsport.}. Science worth its mettle and
of certifiable vintage will be steeped in a laconic grammar befitting
pursuit of that brand of truth first minted by Bacon, and much later,
by Popper's inveterate philosophy, turned to diamond: careful,
measured, and repeatable observation, followed by deduction and
falsifiable conclusion, corrected for faulty intuition and attended to
with utter impartiality---or at least a display of it sufficing to
avoid being given the proverbial lie.

Having drunk reverently, deeply, and frequently from this cup, this
veritable goblet of gladness, I opt to lay the rudiments of the
present work at its outset in the parlance of the commoner---the
salesman, the nurse, the brickmason, the entrepreneur$^*$---that all
who desire may sup the elixir, if only for a chapter. But if I do not
disallow that the imbibing of these spirits may spur, in its course,
to greater heights, I am contrariwise attended by an assurance of the
impenetrability of succeeding chapters to the mind unequipped with the
several keys of knowledge which I forthwith commence to afford.

\section{Talkin' waves}

% Keep the old parindent and parskip in \savedparindent and \savedparskip
\newlength{\savedparindent}
\newlength{\savedparskip}
\setlength{\savedparindent}{\parindent}
\setlength{\savedparskip}{\parskip}

% Now set the new'ns
\setlength\parindent{0pt}
\setlength\parskip{1ex plus 2pt minus 1pt}
\newcommand\X{\par\noindent---~}

\X ``What is an \Alf wave?''  

A natural low-frequency mode of
oscillation of a plasma.

\X ``Are there high-frequency modes?''  

Yes! Plasma mode \& Co.

\X ``What makes them high frequency?''


They deal with oscillations too rapid for an ion to participate,
and such oscillations are totally neglected i ngoing from two-fluid
theory to MHD---specifically when the Hall term (), which includes (),
is discarded.

\X ``What's the importance of them?''  

You mean in relative terms, or in some abstract, general way?

\X ``Both, but start with relative terms.''

All right. Generality. Wait---let's
start with some background and history. You're familiar with soun
waves, right? The idea that the air can vibrate and carry the sound of
music, the sound of a fog horn, the drop of a pin, yadda yadda?

\X ``Yeah.''


Cool. Know why the air does that?


\X ``Uhh \dots''


Right. Just think about the speed of sound, which I bet you've heard
of. We'll call it $v_s$. Here's the thing: in simple
cases\footnote{I'm glossing over a lot of the legwork because we've
  got a long way to go in a different direction. Check out your
  favorite intro text if you want to dive into details.}  $v_s$
dictates a relationship between the wavelength $\lambda$ and frequency
$f$ of a particular sound wave. Namely,
\begin{equation}
  \label{ch1:eqvs}
  \lambda f = v_s,
\end{equation}
or, just as good,
\begin{equation}
  \label{ch1:eqvsdiff}
  \lambda = \dfrac{v_s}{f}.
\end{equation}
What does it mean? That $\lambda$ and $f$ are \emph{proportional} to
each other--inversely proportional, that is-- and the so-called
\emph{constant of proportionality} is $v_s$, the speed of
sound!\footnote{In practice, scientists prefer the form \ref{ch1:eqvs}
  because $\lambda$ is a length and $f$ indicates ``how many per
  time;'' if you multiply $\lambda$ and $f$ together, you get a
  ``distance per time,'' or a speed!} $v_s$ is a magical number that
nature picks based on the air temperature $T$ and number density $n$
(i.e., how many air molecules or atoms fit in a box of a size that we
agree upon, like 1~cm~$\times$~1~cm~$\times$~1~cm). We could say that
$v_s = v_s (T, N)$, which is a compact way of saying that ``the speed
of sound depends on air temperature and density.'' For example, when
you pluck the low E string on a guitar, which is the fattest string,
it vibrates and launches vibrations in the air that vibrate at a
frequency of about 82~Hz; when you hear it, you might say it is
``knocking at your door,'' or your eardrum, 82 times each second: $f =
$~82~Hz. Well, suppose $v_s =$~340~m/s---which is a pretty good answer
by cereal box standards---the wavelength of that low E, $\lambda$ is
4~m, or about the length of a 1993 Buick Century (my stallion of
choice). How about we ascend in pitch to the next string on a guitar,
the A string?\footnote{No time for quibbling with different tunings,
  now.} $f = $~110~Hz, and according to equation~(\ref{ch1:eqvs}),
$\lambda \simeq$~3.1~m--about the distance from the ground to the net
of a basketball hoop.

Why am I telling you all this? Because there's an important concept
illustrated here, what some of us laboratory-bound types call a
\emph{dispersion relation}. In the very particular case at hand,
equation~(\ref{ch1:eqvs}) is telling you that if you start ascending
the E major scale on your guitar so that the pitch steadily increases,
nature has informed us that the wavelength of each note played gets
shorter and shorter so that equation~(\ref{ch1:eqvs}) is always true.

Things don't have to be so simple, you know. In fact, instead of
equation~(\ref{ch1:eqvs}) I could turn up the heat in a
generalized\footnote{In physics speak, when discussing a particular
  theory the word ``generalized'' signals to all physicists in the
  area that the ideas are about to become far-reaching, or impossible
  to comprehend, or take a turn for the worse into quackery. It's a
  word that can only be taken on a case-by-case basis, so beware!}
way, and tell you something like
\begin{equation}
  \label{ch1:eqvsgen}
  \lambda = \lambda (f,T,N),
\end{equation}
which is wonderfully vague. It looks meaningless, doesn't it? But
no--it's telling you that the wavelength $\lambda$ depends on
frequency, temperature, and density, just like
equation~(\ref{ch1:eqvs}); the difference is that I haven't bothered
to specify \emph{how} they are related. Why? For all its vagueness
equation~(\ref{ch1:eqvsgen}) buys us power to dream a little. Suppose
you and I are investigating some exotic fluid---we'll pretend for
kicks that it behaves sort of like a gas, by which I mean that it can
be described by quantities like temperature and density that we know
so well.\footnote{If you think you don't know them well, you simply
  haven't realized that you make everyday decisions, like whether to
  go running, on the basis of temperature (``Is it too hot out
  there?'') and density (``Is it going to be gross and humid?'').}
Let's futher pretend that this exotic fluid is made up of charged
particles, so that it is also subject to electrodynamics and can do
fancy things under special conditions, like freeze a magnetic field
into itself\footnote{Within our solar system, the ``frozen magnetic
  field'' thing happens all over the neighborhood, so to speak. It is
  usually fair to describe the solar wind as an ``MHD fluid''
  (discussed in the text) as some of us are wont to call it, with a
  frozen-in magnetic field whose field lines twist like the skirt of a
  ballerina as she twirls, or the water sprayed by the sprinkler head
  on your lawn as it spins. In the case of the solar wind, these
  spirals happen because of the rotation of the sun, the flow of the
  solar wind \emph{away} from the sun, and the whole ``fluid that
  freezes magnetic fields into itself'' idea.} and conduct electrical
currents.

So what might the dispersion relation, the ``updated'' version of
equation~(\ref{ch1:eqvs}), look like for this exotic fluid? Back in
1942 Hannes \Alf proposed that within this putative
``magnetohydrodynamic,''\footnote{It might surprise you that at the
  healthy length of eight syllables, this is a word that gets used
  regularly in plasma physics.} or MHD fluid,
\begin{equation}
  \label{ch1:eqmhd}
  \lambda = \lambda (f, T, n, B).
\end{equation}
All he did was suggest dependence on the magnetic field $B$! He went a
step further and nailed down an explicit dispersion
relation\footnote{I've swept a few details under the rug. But not
  many, considering Alfv\'{e}n's original letter to \textsl{Nature}
  was only six paragraphs! (In case you didn't know, \textsl{Nature}
  is considered by many to be the darling of the physical sciences as
  far as publication goes, the obvious corollary being that most
  scientists have developed a strong opinion about it or otherwise
  have emotional baggage related to it.)} (I've only changed the
units):
\begin{equation}
  \label{ch1:eqAlf}
  \lambda f = \frac{B}{\sqrt{\mu_o m n}} \equiv v_A,
\end{equation}
where $v_A$ is the eponymous (and, I should add, rather famous among
plasma physicists) \textsl{\Alf speed}, $B$ is the magnetic field
strength of the , and $m$ and $n$ are the mass and number density of
the ions in this fluid. And  

Don't let the details cause you to to miss the point: The dispersion
relation (\ref{ch1:eqAlf}) is \emph{identical} in form to
equation~(\ref{ch1:eqvs}). To wit\footnote{An expression frequently
  employed by D.J. Griffiths, the celebrated author of undergraduate
  physics textbooks. (If you don't believe me that textbook authors
  are on the level of celebrities or---dare I betray the pettiness of
  scientists?---public enemies, ask physics majors what classes they
  are taking. Then ask about the textbook for each class, and how
  these majors feel about the textbook author.)}, $ \lambda f = $(some
speed). And to bring you up to speed, at this point there is a
mountain of textbooks and peer-reviewed literature that refer to the
\Alf speed and (surprise!) \Alf waves. Why? In his review of
experiments dealing with \Alf waves, \cite{Gekelman1999} put it this
way: ``These waves are ubiquitous in space plasmas and are the means
by which information about changing currents and magnetic fields are
communicated.''\footnote{There are also tip-top reviews and reports of
  \Alf waves in the magnetosphere, ionosphere, and the region in
  between \citep{Stasiewicz2000,Berthomier2011,Mottez2015}, in the
  magnetotail \citep{Keiling2009}, at the Sun
  \citep{Mathioudakis2013}, in various combinations of all of these
  places \citep{Wu2016a}, and even around otherworldy objects like
  neutrino stars \citep{Duncan1996}.} So if there is some disturbance
like a coronal mass ejection banging on the nose of the bow shock
(Figure~\ref{ch1:FigBowShock}),

  % ---------------- FIGURE 2

  \begin{figure}
    \centering
    \noindent\includegraphics[width=0.8\textwidth]{./ch1/figs/figbowshock}
    \caption[The Earth's Bow Shock]{The Earth's Bow Shock [Retrieved
      from the WIND Magnetic Field Investigation (MFI) Team
      \href{https://wind.nasa.gov/mfi/team_sciencea.html}{website}].}
    \label{ch1:FigBowShock}
  \end{figure}

  % ----------------

  In fact, with the solitary exception of chapter~6, gratuitous
  references to \Alf this-and-that are found throughout the chapters
  comprising this thesis; I doubt if there are more than five total
  pages in chapters 2--6 that fail to mention the name.


\section{From waves to waves}



\section{From waves to \dots  electrons?}

Let's talk about the aurora then. Even during geomagnetically quiet
times, roughly 20~GW\footnote{[Real-life way to comprehend number]}
\citep{Newell2009} are raining down on the poles of the earth through
various types of aurora, and so-called ``wave aurora'' (because, you
know, it's powered by waviness, like \Alf waves!) contributes about
5\% of the total energy budget. When things pick up---for example,
during a geomagnetic storm---the electron precipitation associated with
\Alf waves increases quite a bit, and broadband electron energy and
number flux can become sizeable fractions of the electron
precipitation budget.

But just what are broadband electrons, and what do they have to do
with \Alf waves? The name refers to the tell-tale signature they leave
when we plot their differential energy flux---a nasty term, I
know. (If you are already familiar with the concept of flux, go ahead
and skip the next two paragraphs.) 

Here's a sketch: Pick up a piece of paper. Imagine that during the
course of one second, an electron comes zipping through it. We'll use
this to define a unit \emph{number flux} of 1, or \emph {1 electron
  per sheet of paper per second}. Fair enough? Now, if instead of
\emph{how many electrons pass through} the sheet of paper in a second
(number flux), we wanted to know \emph{how much energy the electrons
  carry through} the sheet of paper in a second (energy flux), we'd
have to count differently by also asking every individual electron how
fast it's going, and then summing up the energy contribution of all
the electrons passing through the page within a second.

Last of all, suppose we wanted to know not just the \emph{total}
electron energy flux through the page, but the \emph{contribution}
made by electrons \emph{having some limited range of energies} to the
total energy flux?\footnote{Why would this be useful?  For the same
  reasons that it would be useful to know [insert TBD analogy].}
Here's an idea: You could slap an energy filter on the page---never
mind where we got this little device, just imagine---so that we only
count electrons with energies between, say, 100~eV and
200~eV.\footnote{``An 'eV'?''  you say? 1 eV is one
  electron-volt---which is to say, it's the amount of energy that an
  electron has after being accelerated through a 1-volt potential
  difference.} If we added up all the energy carried by \emph{those}
electrons, repeated the process for electrons having energies of
200--500~eV, then 500--1200~eV, so on and so forth, we'd have a list
of electron energy ranges, and an energy flux measurement for each
energy range, \emph{capisce}? Like this:
\begin{equation}
  \label{ch1:eqeFlux}
  \J_{E,\textrm{total}} = J_{E,\textrm{100--200 eV}} + J_{E,\textrm{200--500 eV}} + J_{E,\textrm{500--1200 eV}} + \dots
\end{equation}

Unfortunately even this description of energy flux isn't enough to
convey the meaning of the \emph{differential energy flux} I originally
mentioned, but keep it in mind, because it's close enough for you to
be able to appreciate what I'm referring to when I bring up broadband
electrons.

Speaking of which, back to broaband electrons. When we fly satellites
above the aurora and measure the differential energy flux of auroral
electrons, we often get a response that is typified by what you see in
Figure~\ref{ch1:FigAlfElec}. The plot in the upper panel shows what
snobby space plasma physicists call a ``pitch-angle spectrogram.''
See how the second line of the left-hand axis label says ``Angle
(Deg.)''?  The angle in question---the \emph{pitch angle}---is
measured relative to the direction that the Earth's magnetic field
points so that, for example, 0$^\circ$ is ``field-aligned'' and
90$^\circ$ is perpendicular to the geomagnetic field. See the obvious
horizontal black strip in the pitch-angle spectrogram? That's telling
you these puppies (if I may be permitted the expression) are coming in
hot.

Take a look at the $y$ axis to get a sense of the range of
angles. Right---they're all within about 30$^\circ$ of 0$^\circ$. They
are mostly field-aligned. And the way to make a pitch-angle
spectrogram is to pick a pitch angle (maybe you could start by looking
at 0$^\circ$, straight down the magnetic field line) and measure the
differential energy flux---or $\frac{d J_E}{dE}$---pick a new pitch
angle and again measure $\frac{d J_E}{dE}$, so on and so forth until
you've measured $\frac{d J_E}{dE}$ at each pitch angle.

In the auroral zone---an altitude of a few hundred km and above at
latitudes\footnote{Magnetic latitude, not geographic latitude.} above
$\sim$60$^\circ$---field-aligned electrons barrel toward the
atmosphere and are near the top of the list of causes of aurora. The
complication is, as you might guess, that there are several ways one
could get field-aligned acceleration.\footnote{\citet{Wygant2002},
  \citet{Bostrom2003a}, \citet{Morooka2004}, \citet{Newell2009},
  \citet{Hull2010}, \citet{Mottez2016}, are only a handful among many,
  many current or recently retired researchers who would love to tell
  you more, and they aren't even the historical giants, like
  \citet{Knight1973}, \citet{Evans1974}, \citet{Hasegawa1976}, and
  \citet{Lyons1980a} (to name a few) who laid the groundwork!}
\citet{Chaston2002,Chaston2003a,Chaston2007,Chaston2008} in particular
will tell you that \Alf waves are certainly on the short list of
suspects, but how would they know?

The plot in the lower panel---a typical example of the unimaginatively
named \emph{energy spectrogram}---shows that ``tell-tale signature''
of broadband electrons that I mentioned at the beginning of this
section. Remember our definition of energy flux? Total amount of
energy carried by electrons, through the page, per second?  and
then to plot the energy fluxes

Next, we slap 

  % ---------------- FIGURE

  \begin{figure}
    \centering
    \noindent\includegraphics[width=0.8\textwidth]{./ch1/figs/Chaston2003_alfelec}
    \caption[\Alfically accelerated (broadband) electrons]{\Alfically
      accelerated (often called ``broadband'') electrons observed by
      the FAST satellite [Adapted from Figures~1c and 1d in
      \citealp{Chaston2003a}]. The upper panel shows a so-called
      ``pitch-angle spectrogram,'' and the lower panel an ``energy
      spectrogram'' (see text).}
    \label{ch1:FigAlfElec}
  \end{figure}

  % ----------------

  

% Finished with the parindent stuff? Reset 'em
\setlength\parindent{\savedparindent}
\setlength\parskip{\savedparskip}

% And chapter 1 bib
\bibliographystyle{agufull08}
\bibliography{refs1}
