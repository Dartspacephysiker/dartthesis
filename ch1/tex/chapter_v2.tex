%------------------------------------------------------------------------%
% Example chapter for the thesis template.                               %
%                                                                        %
%      Author: Gregory Alexander Feiden                                  %
%   Institute: Dartmouth College                                         %
%        Date: 2014 May 17                                               %
%                                                                        %
%     License: Beerware (revision 42)                                    %
%              ----------------------                                    %
%              Gregory Feiden wrote this file. As long as you retain     %
%              this notice you can do whatever you want with this code.  %
%              If we meet some day and you think this code is worth it,  %
%              you can buy me a beer in return.                          %
%                                                                        %
%------------------------------------------------------------------------%

\chapter{Introduction}
\label{chp:1}

% ---BEGIN INSPIRATION QUOTE 
\begin{flushright}
  \begin{minipage}[]{0.5\linewidth}
    \begin{flushright}
      You have no responsibility to live up to what other people think you ought
      to accomplish. I have no responsibility to be like they expect me to
      be. It's their mistake, not my failing.  \\{\small \emph{--- Se\~{n}or
          Feynman} }
    \end{flushright}
  \end{minipage}
\end{flushright}
\vspace{\baselineskip}
% ---END INSPIRATIONAL QUOTE

\section{A word before things pick up}

If the pursuit of knowledge fills the soul with joy, then the
communication of it brings the soul to overflowing.\footnote{We pass
  in piety over the many and ongoing instances in which otherwise
  respectable science has evolved into bloodsport.} Science worth its
mettle and of certifiable vintage will be steeped in a laconic grammar
befitting pursuit of that brand of truth first minted by Bacon, and
much later, by Popper's inveterate philosophy, turned to diamond:
careful, measured, and repeatable observation, followed by deduction
and falsifiable prediction, corrected for faulty intuition and
attended to with utter impartiality---or at least a display of it
sufficing to avoid being given the lie.

Having drunk reverently, deeply, and frequently from this cup, this
veritable goblet of gladness, I opt to lay the rudiments of the
present work at its outset in the parlance of the laity---the
salesman, the nurse, the blockmason, the entrepreneur, the
custodian\footnote{All honorable occupations held by the author's
  family members, some of which he has also held.}---that all who
desire may sup the elixir, if only for a chapter. But if I do not
disallow that the imbibing of these spirits may spur, in its course,
to greater heights, I am contrariwise attended by an assurance of the
impenetrability of succeeding chapters to the mind unequipped with the
several keys of knowledge which I forthwith commence to afford.

\section{Talkin' waves}

% % Keep the old parindent and parskip in \savedparindent and \savedparskip
% \newlength{\savedparindent}
% \newlength{\savedparskip}
% \setlength{\savedparindent}{\parindent}
% \setlength{\savedparskip}{\parskip}

% % Now set the new'ns
% \setlength\parindent{0pt}
% \setlength\parskip{1ex plus 2pt minus 1pt}
% \newcommand\X{\par\noindent---~}

% \X ``What is an \Alf wave?''  

% A natural low-frequency mode of
% oscillation of a plasma.

% \X ``Are there high-frequency modes?''  

% Yes! Plasma mode \& Co.

% \X ``What makes them high frequency?''


% They deal with oscillations too rapid for an ion to participate,
% and such oscillations are totally neglected i ngoing from two-fluid
% theory to MHD---specifically when the Hall term (), which includes (),
% is discarded.

% \X ``What's the importance of them?''  

% You mean in relative terms, or in some abstract, general way?

% \X ``Both, but start with relative terms.''

% Let's start with some background and history. You're familiar with soun
% waves, right? The idea that the air can vibrate and carry the sound of
% music, the sound of a fog horn, the drop of a pin, yadda yadda?

% \X ``Yeah.''


% Cool. Know why the air does that?


% \X ``Uhh \dots''


% Finished with the parindent stuff? Reset 'em
% \setlength\parindent{\savedparindent}
% \setlength\parskip{\savedparskip}

Let's start with some background and history. I guess you're familiar with sound
waves? The idea that the air can vibrate and carry the sound of music, the sound
of a fog horn, the drop of a pin, yadda yadda?

Right. Just think about the speed of sound, which I bet you've come across
before. We'll call it $v_s$. Here's the thing: in simple cases\footnote{I'm
  glossing over a lot of the legwork because we've got a long way to go in a
  different direction. Check out your favorite intro text if you want to dig
  into details.}  $v_s$ dictates a relationship between the wavelength $\lambda$
and frequency $f$ of a particular sound wave. Namely,
\begin{equation}
  \label{ch1:eqvs}
  \lambda f = v_s,
\end{equation}
or, just as good,
\begin{equation}
  \label{ch1:eqvsdiff}
  \lambda = \dfrac{v_s}{f}.
\end{equation}
What do (\ref{ch1:eqvs}) and (\ref{ch1:eqvsdiff}) mean? That $\lambda$ and $f$
are \emph{proportional} to each other---inversely proportional, that is---and
the so-called \emph{constant of proportionality} is $v_s$, the speed of
sound!

For example, when you pluck the low E string on a guitar (which is the fattest
string)\footnote{We're assuming standard EADGBE tuning. The time fails to
  quibble with different tunings.} it vibrates; in turn, the vibrating E string
causes the air pressure to pulsate\footnote{In the lingo, a sound wave is what
  is known as a \emph{compressional mode}. As we'll see later on, some fluids
  also support \emph{transverse modes}.} at a frequency of about 82~Hz, and
those pulsations travel to your ear and everywhere else. When you hear it, it is
knocking at your eardrum 82 times each second: $f = $~82~Hz.  Well, suppose
$v_s =$~340~m/s---a pretty good answer by cereal box standards---the wavelength
$\lambda$ of the sound wave carrying that low E is 4~m, which is about the
length of my 1993 Buick Century Special. What if we ascend in pitch to the
next string on guitar, the A string?  $f = $~110~Hz, and according to
equation~(\ref{ch1:eqvs}), $\lambda \simeq$~3.1~m---about the distance from the
ground to the net of a basketball hoop.

You get the picture: if you start ascending the E major scale on your guitar so
that the pitch of each note steadily increases, nature has informed us that the
wavelength of the sound wave carrying each note you play gets shorter and
shorter so that equation~(\ref{ch1:eqvs}) is always true---which is to say that
sound waves always propagate at the speed of sound.\footnote{Except when they
  don't, but we'll not rove in this direction because we have other destinations
  in mind.}

Why are we talking about this? Because there's an important concept illustrated
here, what some of us laboratory-bound types call a \emph{dispersion relation}
that I promise you will make cameo appearances, albeit in different costume,
throughout this thesis. We'll come back to dispersion relations shortly. A more
pressing issue that deserves your attention is that in their present forms the
relations (\ref{ch1:eqvs}) and (\ref{ch1:eqvsdiff}), while not wrong, would
nevertheless make some physicists (shy and delicate creatures as they are)
uncomfortable. The short explanation is that $f$ measures inverse time while
$\lambda$ has units of length, which, you may be surprised to learn, would
represent a gross imbalance to the mind of many a physicist.\footnote{The
  imbalance is not simply aesthetic, either. There are very good reasons to put
  time and space on equal footing, far from the least of which is the paving of
  a conceptual pathway between physical domains of time and space and their
  spectral counterparts. This pathway turns out to be important for any area of
  physics that deals heavily in waves, including quantum mechanics.}

\subsection{The wave number $\mathbf{k}$, and a philosophical excursion}
\label{subsec:k}

To put time and space on equal footing (admittedly only in terms of
units\footnote{Even if our present aim to achieve equal footing for
  Equation~(\ref{ch1:eqvs}) is base (we're just changing units, after all), the
  spirit of our aim is lofty. To truly put time and space on equal footing, on
  simultaneously quantum and relativistic scales, is among the highest and most
  sweeping challenges facing the last century or so of physics.}) let me
introduce you the concept of \emph{spatial frequency}, the typical,\footnote{We
  are not talking about the \emph{spectroscopic wave number}
  $\nu \equiv 1 / \lambda$ that I understand is common among chemists.}
mathematical definition\footnote{Definitions, the troublesome things. I have
  observed that beginning students in physics are prone to confuse definitions,
  such as the definitions of wave number (\ref{ch1:eqk}) and work (i.e., the
  energy resulting from a force acting over a distance), with physical laws. A
  definition is no such thing. Instead, a definition is some quantity that keeps
  presenting or suggesting itself in the process of research that makes sense to
  set aside and give formally bestowed meaning---a \emph{definition}. The root
  of confusion seems to be that each discipline is heavily steeped in naturally
  arising, field-specific lingo; intertwinement of lingo with the laws that get
  described in terms of that lingo produces a trap for unwary laity and budding
  physicists alike.} of which is
\begin{equation}
  \label{ch1:eqk}
  k = \dfrac{2 \pi}{\lambda}.
\end{equation}
``Great,'' I can hear you saying to yourself. ``You managed to obfuscate the
meaning of wavelength by burying it in the denominator and decorating it with
superfluous numbers.'' Well, give me a minute to explain.

One way to understand the reason for placing $2 \pi$ in the numerator is that we
want to compare any frequencies we observe against a comprehensive
standard.\footnote{I'm altogether avoiding the subject of phase of a cycle; you
  can find it in any introductory textbook that is worth even an iota of its
  salt.} ``What standard?'' Great question! Think: a pattern that repeats itself
at a regular interval (whether spatial or temporal) is a cycle, and the simplest
and most universal example of a cycle is a \emph{circle}---particularly a circle
defined to have a radius of 1 (i.e., the unit circle), which has a circumference
of $2 \pi$.  Adopting the unit circle as our ``standard cycle'' against which to
measure every other cycle, we compare both temporal and spatial frequencies
against $2 \pi$, which is the numerical representation of our adopted
standard.\footnote{At least to me, by comparing various types of cycles against
  the most basic form of a cycle (which I have unilaterally asserted is the unit
  circle) we are toeing the line at which we implicitly accept the existence of
  an archetype---a concept that I understand was first presented by Plato.}

Now that we see the light, may we likewise outfit temporal frequency with a
factor of $2 \pi$? Let's define
\begin{equation}
  \label{ch1:eqomega}
  \omega = 2 \pi f.
\end{equation}
It feels nice, doesn't it? With the definitions (\ref{ch1:eqk}) and
(\ref{ch1:eqomega}) for $k$ and $\omega$, both spatial and temporal cycles (or
frequencies) are being compared against the archetypal cycle---the ``gold
standard''---which is the unit circle, in the form of the number 2$\pi$.

I hope you find the foregoing explanation as aesthetically pleasing as I do. But
I am aware that aesthetic satisfaction doesn't correspond to comprehension, so
let me illustrate the concept of spatial frequency at work. For my first
example, you don't even have to take your eyes off this or the succeeding lines
of text: the distance between the top of one letter on this line to the top of a
letter on the next line is 40 pts,\footnote{A pt (or ``point''), which is a
  typographical unit, is such that 72~pts~$=$~1~inches.}  or roughly 0.5~inches
(get a ruler if you want to check yourself). We could say
$\lambda \simeq$~0.5~inches, or
$k = \frac{2 \pi}{\lambda} \simeq 4 \pi$~rad/inch. We could probably do some
simple tests to find that books that humans can read with the naked eye must
have $k_{\textrm{max}} \lesssim$~16$\pi$~rad/in. It's nothing more than a mathy
way of saying, ``If the text is too small, no one will be able to read
it''---something you already knew, cast in terms of spatial frequency.

I'll give you one more example, which is shown in
Figure~\ref{ch1:FigDesert}. There were clearly waves in the sand when the
picture was taken, and they clearly weren't going anywhere anytime soon; by all
appearances the temporal frequency of these waves was $f =$~0~Hz.  What about
spatial frequency? My estimate is that the average distance between wave crests,
or the average wavelength $\lambda =$~10~cm (about 4~inches), which is
equivalent to $k =$~$\frac{\pi}{5}$~rad/cm.  If you pay attention to clouds in
the sky, it won't be long before you're rewarded with a sighting of spatially
periodic structures in the clouds.

% ---------------- FIGURE

\begin{figure}
  \centering
  \noindent\includegraphics[width=0.9\textwidth]{./ch1/figs/Shaybah-Wavy-Sand-Dunes.jpg}
  \caption[Desert waves]{Waves in desert sand that, unlike ocean waves (for
    example), are \emph{fixed} in time and therefore have a temporal frequency of
    zero! Their spatial frequency is not zero, however.}
  \label{ch1:FigDesert}
\end{figure}

% ----------------

% OK, we now see that spatial frequency is ``out in the real world.'' Still,
% couldn't we stick with $\lambda$ instead? Figure~\ref{ch1:FigDesert} beautifully
% illustrates the reason why there's no special reason to stick with $\lambda$. A
% question for you: Can you tell me where the peak of the wave is in
% Figure~\ref{ch1:FigDesert}? I hope you are bothered by my question, and that you
% are saying to yourself, ``Spencer is off his rocker. Those are ridgelines, not
% peaks, and there are probably fifty separate ridgelines in that shot!''
% Exactly. Compared to talking about wavelength, it makes at least as much sense
% to talk about the repeating pattern of waves in the sand in terms of spatial
% frequency.

I keep threatening to put time and space are on equal footing. If I rearrange
(\ref{ch1:eqk}) and (\ref{ch1:eqomega}) so that $\lambda = \frac{2 \pi}{k}$ and
$f = \frac{\omega}{2 \pi}$, and insert these expressions for $\lambda$ and $f$
into (\ref{ch1:eqvs}), I get
\begin{equation}
  \label{ch1:eqvomegak}
  \lambda f = \Big ( \dfrac{\cancel{2 \pi}}{k} \Big ) \Big ( \dfrac{\omega}{\cancel{2 \pi}} \Big ) = \dfrac{\omega}{k} = v_s.
\end{equation}
The two rightmost expressions, $\omega / k = v_s$, are indeed
satisfying. Equation \ref{ch1:eqvomegak} would make any physicist who deals with
continuous matter\footnote{This includes physicists working in hydrodynamics,
  tectonics, solid state and condensed matter, plasma, and statistical
  mechanics, to name a few.} feel more or less at home.

\subsection{What determines the speed at which waves propagate in a fluid?}
\label{subsec:wavspeed}

Besides talking about the existence of the speed of sound and fiddling with
definitions, there's another question we might ask: How did the air decide to
pick $v_s$ as the ``right speed'' for sound waves to travel? Why not the speed
of light, $c$? Why not my Buick's top speed, 80~mph ($\sim$129~kph)? If I hadn't
skipped the details that lead to equation (\ref{ch1:eqvs}), we would have seen
that nature picks $v_s$ based on the air temperature $T$ and the mass of
constituent molecules\footnote{We'll keep things simple and say the
  \emph{average} mass of a molecule in the air you're breathing. Poke around in
  the right books and you'll find that the average mass of an air molecule is
  5.6$\times$10$^{-26}$~kg.} $m$.
% WRONG VERSION. It doesn't depend on number density!!
% that nature picks $v_s$ based on the air temperature $T$ and number
% density\footnote{``Number density'' meaning ``the number of air molecules and
%   atoms that fit in a box of a size that we agree upon.'' For example we could
%   use a box with dimensions 1~cm~$\times$~1~cm~$\times$~1~cm, which is 1 cm$^3$,
%   or a cubic centimeter. If you stuck a box of this size into the open air and
%   did some counting, you would find roughly \numprint{25000000000000000000}
%   molecules and atoms inside, or $n$ = 2.5$\times$~10$^{19}$~cm$^{-3}$.} $n$.

We're all grown-ups here, so let me show you exactly what I mean. Try
the following definition\footnote{This is the definition appropriate for an
  ideal gas. I've replaced the usual adiabatic index $\gamma$ with 1.4, which is
  roughly true for diatomic molecules when $T = 0 ^{\circ}$C, and
  $k_B \simeq$~1.38$\times$10$^{-23}$~J/K.} of $v_s$ on for size:
\begin{equation}
  \label{ch1:eqvsreal}
  v_s (T,m) = \sqrt{\dfrac{1.4 k_B T}{m}}.
\end{equation}
What you see in (\ref{ch1:eqvsreal}) is the explicit demonstration of what I
already mentioned---that the speed of sound depends on air temperature and the
(average) mass of an air molecule. And I think you'll agree that
(\ref{ch1:eqvsreal}) makes sense: If $T = 0$~K, $v_s = 0$; i.e., there are no sound
waves. On the other hand as $m$ increases, $v_s$ decreases.

What to conclude? That the properties of a system determine how (or perhaps
whether) it resonates. Stated in alternate fashion, \emph{the system properties
  determine the dispersion relation}.  As another example, if you pluck a guitar
string it will pick a single frequency (and probably harmonics) at which to
resonate based on the string tension as well as how thick and how long the
string is.\footnote{Look up standing waves on a string. It's a classic in
  introductory physics courses.} That's the system---a string fixed at two
ends. Like we discussed in the last section, there is a whole spectrum of
frequencies and wavelengths that the air allows, including the whole range of
human hearing,\footnote{My own unfounded speculation is that humans must have
  evolved to hear over the range of frequencies that carry information most
  meaningful to our survival. However, I am not ignorant of those without
  hearing, and the words of Nietzsche in \textit{Menschliches,
    Allzumenschliches} come to mind: ``There is rarely a degeneration, a
  truncation, or even a vice or any physical or moral loss without an advantage
  somewhere else. In a warlike and restless clan, for example, the sicklier may
  have occasion to be alone, and may therefore become quieter and wiser; the
  one-eyed will have one eye the stronger; the blind will see deeper inwardly,
  and certainly hear better. To this extent, the famous theory of the survival
  of the fittest does not seem to me to be the only viewpoint from which to
  explain the progress of strengthening of [\dots] a race.''} and much more. But
equations (\ref{ch1:eqvomegak}) and (\ref{ch1:eqvsreal}) say that no matter how
you slice it, the waves move at the constant speed $v_s$.

It doesn't have to be so simple, you know. Instead of
equation~(\ref{ch1:eqvsreal}) I could turn up the heat in a
generalized\footnote{In physics speak, the word ``generalized'' signals to all
  physicists in the area that the ideas are about to become far-reaching, or
  impossible to comprehend, or worse, descend into quackery. It's a word that
  can only be taken on a case-by-case basis, so beware!}  way, and tell you
something like
\begin{equation}
  \label{ch1:eqvsgen}
  v_s = v_s (T, m),
\end{equation}
which is wonderfully vague. It looks meaningless, doesn't it? But no---it's
telling you that the ``the speed of sound is related to air temperature and the
mass of an air molecule,'' just like equation~(\ref{ch1:eqvsreal}). The
difference is that (\ref{ch1:eqvsgen}) hasn't bothered to specify the
relationship between system variables and the speed of sound, and instead only
states that a relationship exists.

Why? For all its vagueness equation~(\ref{ch1:eqvsgen}) buys us power to dream a
little. Suppose you and I are investigating some exotic fluid---we'll pretend
for kicks that it behaves sort of like a gas, so that it can be described by
quantities like temperature and particle mass. Let's further pretend that this
exotic fluid is made up of charged particles, so that it is also subject to
electrodynamics and can do fancy things under special conditions, like freeze a
magnetic field into itself\footnote{Within our solar system, the ``frozen
  magnetic field'' thing happens all over the neighborhood, so to speak. It is
  usually fair to describe the solar wind as an ``MHD fluid'' (discussed in the
  text) as some of us are wont to call it, with a frozen-in magnetic field whose
  field lines twist like the skirt of a ballerina as she twirls, or the water
  sprayed by the sprinkler head on your lawn as it spins. In the case of the
  solar wind, these spirals happen because of the rotation of the sun, the flow
  of the solar wind \emph{away} from the sun, and the whole ``fluid that freezes
  magnetic fields into itself'' idea.}  and conduct electrical currents.

So what might the dispersion relation, the ``updated'' version of
equation~(\ref{ch1:eqvsgen}), look like for this ungodly
``magnetohydrodynamic,''\footnote{It might surprise you that at the healthy
  length of eight syllables, this is a word that gets used regularly in plasma
  physics.} or MHD, fluid? Back in 1942 Hannes \Alf proposed
\begin{equation}
  \label{ch1:eqmhd}
  v = v ( T, m, n, B).
\end{equation}
Comparing equation \ref{ch1:eqmhd} with equation (\ref{ch1:eqvsgen}) informs you
of this: all he did was suggest additional dependence on the magnetic field $B$
and the number density\footnote{``Number density'' meaning ``the number of gas
  particles that fit in a box of a size that we agree upon.'' For example we
  could use a box with dimensions 1~cm~$\times$~1~cm~$\times$~1~cm, which is 1
  cm$^3$, or a cubic centimeter. If you stuck a box of this size into the open
  air and did some counting, you would find roughly
  \numprint{25000000000000000000} molecules inside, or $n$ =
  2.5$\times$~10$^{19}$~cm$^{-3}$.} of particles $n$!  The reasons why he
suggested this dependence, though, are more sophisticated than his simply
``taking a crack at it'' and shoving some extra alphabet letters into the
right-hand side of (\ref{ch1:eqvsgen}).\footnote{A good, old-fashioned wild
  guess is nonetheless a respectable strategy when attacking, for instance,
  nonlinear differential equations; solving them seems to be the mathematician's
  equivalent of shooting beer cans off a fencepost.} He went a step further and
nailed down an explicit dispersion relation\footnote{Some dust-bunny details are
  going under the rug, but not many, considering Alfv\'{e}n's original letter to
  \textsl{Nature} was only six paragraphs! (In case you didn't know,
  \textsl{Nature} is considered by many to be the darling of the physical
  sciences as far as publication goes, the obvious corollary being that most
  scientists have developed a strong opinion about or have emotional baggage
  related to it.)} comparable to (\ref{ch1:eqvsreal}):
\begin{equation}
  \label{ch1:eqAlf}
  \dfrac{\omega}{k_\parallel} = v_A \equiv \frac{B}{\sqrt{\mu_o m_i n}},
\end{equation}
where $v_A$ is the eponymous and rather famous \textsl{\Alf speed} (I doubt \Alf
ever called it that), $B$ is the strength of the background magnetic
field,\footnote{When physicists talk about a ``background'' element of their
  storyboard, they usually mean that that element is changing slowly enough
  compared to whatever they want to investigate that it makes more sense to
  treat the element as static in time. For equation (\ref{ch1:eqAlf}), \Alf was
  assuming the presence of a background (or static) magnetic field.} and $m_i$
and $n$ are the mass and number density of the ions that make up the fluid.

If you are paying close attention to (\ref{ch1:eqAlf}), you might wonder why I
choose to mark ion mass with the subscript `$i$', but not the ion number
density. The short answers are these: We only care about ions when it comes to
mass, and we care about ions \emph{and} electrons when it comes to number
density.

I recognize the short answers probably fall short of helpful, so let's expand:
First, we only care about the ion mass $m_i$ (and ignore the electron mass,
$m_e$) in equation (\ref{ch1:eqAlf}) because as you may know, even the smallest
ion\footnote{Naked hydrogen, which is just a proton.} outweighs an electron by a
factor of almost two thousand.\footnote{To be a bit more exact the
  proton-to-electron mass ratio is \numprint{1836.15267389}(17), according to
  the \textsl{2014 CODATA recommended values.}}

``Fair enough,'' you say, ``but why no subscript on the number density $n$?  Why
not write `$n_i$' instead of writing `$n$'? We only care about ions, right?''
Aha! If there were \emph{only} ions present, and no electrons to counterbalance
the mutual repulsion of the positively charged ions, you can easily appreciate
that we couldn't have this conversation: ions simply would not hang around each
other without the presence of a roughly equal number electrons. The presence of
roughly equal numbers of both positive and negative particles---ions and
electrons---is a condition known as \emph{quasineutrality} among plasma
physicists. More formally stated,
\begin{equation}
  \label{ch1:eqQuasi}
  n_i \approx n_e.
\end{equation}
Since the ion number density $n_i$ and the electron number density $n_e$ are
more or less the same throughout the plasma, we don't bother making a
distinction between the number densities $n_i$ and $n_e$; in other words, we are
satisfied with the approximation $n_i = n_e = n$.\footnote{Be aware that this
  is a white lie; it is a good approximation in a lot of situations, but it
  certainly not true in every situation.}


% ---------------- FIGURE

\begin{figure}
  \centering
  \noindent\includegraphics[width=0.95\textwidth]{./ch1/figs/HaywardDungey1982}
  \caption[\Alf wave propagation]{This figure has too much information. Just
    notice that the path of propagation of an \Alf wave, marked with a dashed
    line, is along geomagnetic field lines [Adapted from
    \citeauthor{Hayward1983}, 1983].}
  \label{ch1:FigAlfProp}
\end{figure}

% ----------------

One more point of trickery, this time on an important difference between the
parallel wave number $k_\parallel$ that you see in the denominator of the
left-hand side of (\ref{ch1:eqAlf}), and the wave number $k$ that you see in
(\ref{ch1:eqvomegak}): The subscript, $\parallel$, informs you that the wave in
question is propagating (i.e., physics speak for ``moving'') in a particular
direction---\emph{along} the magnetic field.\footnote{In plasma physics,
  whenever the subscripts $\parallel$ and $\perp$ (``parallel'' and
  ``perpendicular'', as you probably guessed) start appearing, they indicate the
  orientation or measurement of the quantity in question relative to the
  geometry of the local magnetic field.}

The idea of a preferred direction of propagation is not totally foreign to you,
even if you don't recognize it immediately. For example, if you want to get your
friend's attention with a whistle or a shout, why do you turn your head to face
them? Presumably because you know that there is a better chance to be heard if
you whistle or shout in your friend's direction. \textit{Voil\`{a}}---you
understand that there \emph{can be} a preferred direction of wave
propagation. Notice, however, that information about the direction of
propagation is utterly absent in
(\ref{ch1:eqvs})--(\ref{ch1:eqvsreal}).\footnote{If we wanted to theoretically
  investigate, for example, how far a wave could propagate before dissipating
  into thermal noise (which is something like air friction), we would need to
  supplement the foregoing dispersion relations with equations describing how a
  sound wave interacts with a gas as it propagates---stuff like wave energy and
  wave spreading. For examples that can only be described as awesome, you could
  check out Sir Geoffrey Taylor's [\citeyear{Taylor1950},\citeyear{Taylor1950a}]
  famous papers dealing with the propagation of a nuclear blast wave. The first
  is thorough and necessary, while the second, in my opinion, is elegant; to
  read it borders on pleasure.}

% A preferred direction of propagation is probably a little difficult to
% appreciate. To illustrate the difference, consider what happens when you drop a
% pebble in a still pond. The ripples move outward from the point on the
% surface where you dropped the pebble, and if the pond is truly
% still, the ripples are concentric---that is, no particular direction  that \Alf waves, however,
% \emph{do} have a preference: along magnetic field lines. The idea is shown in
% Figure~\ref{ch1:FigAlfProp}.

I've been slinging details, but don't let them cause you to to miss the
point. The dispersion relation (\ref{ch1:eqAlf}) is \emph{identical} in form to
equation~(\ref{ch1:eqvomegak}). To wit,\footnote{An expression frequently
  employed by D.J. Griffiths, the celebrated author of undergraduate physics
  textbooks. (If you don't believe me that textbook authors are on the level of
  celebrities or---dare I betray the pettiness of scientists?---public enemies,
  ask physics majors what classes they are taking. Then ask about the textbook
  for each class, and how these majors feel about each textbook author.)}
$ \omega / k = $(some speed). You should be aware of something else: \Alf
obtained the result (\ref{ch1:eqAlf}) by assuming that there \emph{is} a fluid
(translation: $n >$~0) whose constituent particles are not massless
(translation: $m >$~0) and that a magnetic field $B$ exists in the fluid. As
soon as we start two-timing on equation (\ref{ch1:eqAlf}) and telling it that
$B =$~0, or that $n =$~0 or $m =$~0, violating the assumptions that brought it
into being, equation (\ref{ch1:eqAlf}) seeks to repay the favor by helpfully
informing you either that $v_A =$~0 or that $v_A =$~(nonsense). Give it a shot:
plug $B =$~0 into (\ref{ch1:eqAlf}). See? $v_A =$~ 0.\footnote{This result makes
  sense: No magnetic field (translation: $B =$~0) means the \Alf speed
  $v_A =$~0, which means no \Alf waves.}  Now try the other two options by
plugging $m =$~0 or $n =$~0 into (\ref{ch1:eqAlf}). See?
$v_A =$~(nonsense).\footnote{Instead of just writing the situation off as
  nonsense, a more interesting route would be to imagine a plasma such that the
  ion mass, number density, and magnetic field were such that the \Alf speed
  approached the speed of light, in which case things get relativistic---you
  know, \textit{\`{a} la} Einstein---in which case \citet{Gedalin1993} has
  reported that the dispersion relation (\ref{ch1:eqAlf}) is rather expressed
  $\omega / k = c \sqrt{B^2/\mu_o / (B^2/\mu_o + \epsilon + P)}$. (You'll need
  to visit \citeauthor{Gedalin1993}'s [1993] treatise if you want to know more;
  I mention it here to give you the flavor of other
  possibilities.)}% \footnote{Siri has explained a form of this phenomenon thus:
  % ``Imagine that you have zero cookies and you split them evenly among zero
  % friends. How many cookies does each person get? See? It doesn't make
  % sense. And Cookie Monster is sad that there are no cookies, and you are sad
  % that you have no friends.'' A more interesting route would be to imagine a
  % plasma such that the ion mass, number density, and magnetic field were such
  % that the \Alf speed approached the speed of light, in which case
  % \citet{Gedalin1993} has reported that the dispersion relation
  % (\ref{ch1:eqAlf}) is rather
  % $\omega / k = c \sqrt{B^2/\mu_o / (B^2/\mu_o + \epsilon + P)}$. (Gonna have to
  % visit \citeauthor{Gedalin1993}'s [1993] treatise if you want to know more.)}

Finally, to bring you up to speed, there is a mountain of
textbooks and peer-reviewed literature that refer to the \Alf speed and
(surprise!) \Alf waves. Why? In his review of experiments dealing with \Alf
waves, \cite{Gekelman1999} put it this way: ``These waves are ubiquitous in
space plasmas and are the means by which information about changing currents and
magnetic fields are communicated.''\footnote{There are also tip-top reviews and
  reports of \Alf waves in the magnetosphere, ionosphere, and the region in
  between \citep{Stasiewicz2000,Berthomier2011,Mottez2015}, in the magnetotail
  \citep{Keiling2009}, at the Sun \citep{Mathioudakis2013}, in various
  combinations of all of these places \citep{Wu2016a}, and even around
  otherworldly objects like neutrino stars \citep{Thompson1996}. If you wanted a
  summary of the entire scientific enterprise surrounding \Alf waves, you might
  have a look at \citeauthor{Cramer2001}'s [2001] expansive review of the
  subject.}

In fact, with the solitary exception of chapter~6, gratuitous references to \Alf
this-and-that are found throughout the chapters comprising this thesis; I doubt
if there are more than five total pages in chapters 2--6 that fail to mention
the name.

\section[When dispersion relations become dispersive]{When dispersion relations
  become dispersive\protect{\footnote{*gasp!*}}}

Let's summarize in a few points what we just picked up.
\begin{itemize}
% \item The relationship between spatial and temporal frequencies of waves 
\item The properties of a system (e.g., temperature, density, magnetic field
  strength) determine the speed of wave propagation.
\item The most basic wave mode that exists in a neutral gas is a wave of
  pressure pulsations, otherwise known as a sound wave.
\item The \Alf wave is a basic wave mode that exists in a \textsl{quasineutral}
  plasma when a background magnetic field is present.\footnote{There are one or
    two other basic wave modes we could talk about; you'll find the discussion
    of them in any introductory text on magnetohydrodynamics.}
\item In their simplest form, \Alf waves propagate along magnetic field lines.
\item Physicists use \textsl{dispersion relations}, which are dependent on the
  properties of a system (among other things), to describe how wave modes
  propagate within that system.\footnote{Dispersion relations can also describe
    how different wave modes couple, how different systems couple (e.g., how a
    vibrating guitar string causes the air to vibrate), and the energy transfer
    between particles and waves in the same system or different systems.}
\end{itemize}

To introduce the present topic, which is \emph{wave dispersion}, notice that
(\ref{ch1:eqvomegak}) and (\ref{ch1:eqAlf}) both describe waves that propagate
at a constant speed ($v_s$ in regular old air and $v_A$ in a magnetized plasma)
no matter what the wave number or frequency of the waves in question. We need to
emphasize this: the wave speeds described by the dispersion relations
(\ref{ch1:eqvomegak}) and (\ref{ch1:eqAlf}) are \emph{independent of wave number
  and frequency}, which is to say that they are \emph{nondispersive}. 

Does it feel like a contradiction in terms to describe nondispersive waves using
a dispersion relation? If so, you may be comforted to know that
(\ref{ch1:eqvomegak}) and (\ref{ch1:eqAlf}) still achieve the purpose of a
dispersion relation that I originally stated, which is specification of the
relationship between wave number and frequency as a function of system
properties. Remember what I stated about $\lambda$ and $f$ being inversely
proportional---or equivalently, that $\omega$ and $k$ are directly
proportional---when we discussed (\ref{ch1:eqvsdiff})? Even that basic statement
constitutes a dispersion relation.

You might ask, ``So wait---I understand that (\ref{ch1:eqvomegak}) and
(\ref{ch1:eqAlf}) are examples of some sort of fringe case in which waves aren't
dispersive, but all that tells me is what dispersive waves \emph{aren't}. Can we
venture out of the negative space and into the positive?'' Your wish is my
command.

Imagine waves traveled at a speed proportional to their frequency so that
higher-frequency waves consistently arrived before waves at lower
frequencies. Your experience at a live concert could be rather different,
depending on how close you were to the performers: audience members on the
front-row might have a shot at a coherent experience, but people in the
nosebleeds would run the risk of hearing the violas about a second or two after
they heard the violins!\footnote{The element of surprise in Haydn's famed
  ``Surprise'' symphony would be a rather difficult, and conceivably impossible,
  feat to achieve. Evenly arpeggiated chords, coherent runs, and
  fast improvisation would be right out.} Why? In my contrived example with wave
speed proportional to frequency, the C produced by bowing the lowest string of a
cello (otherwise known as $\textrm{C}_3$) would take twice as long\footnote{As
  you know, the frequency of the first octave above a particular note is double
  the frequency (e.g., the frequency of middle C, or $\textrm{C}_4$ is about
  262~Hz, and the frequency of the open note produced by an openly bowed first
  string of a viola, $\textrm{C}_3$, is $\sim$131~Hz.} to reach your ears as the
lowest C playable by a violin ($\textrm{C}_4$, which is middle C). Take this
fictitious little example as an instantiation of strong\footnote{Before you
  accuse me of tacking a content-free adjective like ``strong'' onto dispersion,
  be aware that in the literature of space physics, this adverb and others like
  it are often intended to signal an effect or phenomenon that we cannot gloss
  over or ignore in understanding the physical picture being presented. I have
  used the phrase ``strongly dispersive'' to present a (hypothetical) situation
  in which sound waves are dispersive to a degree that is totally unfamiliar to
  humans.}  dispersion.

It now behooves us to get on a nittier, grittier level with \Alf waves. I have
now exhausted my stores of knowledge on sound waves, and I'm afraid I'm under
necessity of submerging you in plasma---theory, that is---and subjecting you to
the harsh reality that by the numbers, the universe at present likes plasma
much, much more than it likes the ocean of neutral gas that we live
under.\footnote{People like Dennis Gallagher propound the idea that 99\% or more
  is made up of plasma, which is not hard to believe even if one only takes a
  stroll around our solar system.} We have to wander outside the realm of
direct, or at least regular, human experience, where as far as any human that
has investigated (and of whom I am aware) can tell, in many situations the
universe would not care if our nosebleed symphony experience were confused and
upset by wave dispersion; that is, examples of dispersive waves are easy to find
in nature. 

Call our situation as humans whatever you want---chance, good luck, providence,
order within chaos,\footnote{Regarding humans themselves, I realize a case could
  be made for oppositely ordering these words.}  hyper-order within order, or
the only plausible scenario\footnote{By definition!}---the fact is that when we
carefully do the science, it turns out our almost exclusively
\emph{nondispersive} experience with sound waves is not representative of the
way various types of waves behave in plasmas,\footnote{As a solitary example,
  see \citeauthor{Stix1962}'s [1962] famous treatise on waves in plasmas.}
solids,\footnote{A classic example of the is the dispersion relation for
  vibrations in Peter Debye's model of a solid; his model and related extensions
  (say, Sommerfeld's model of fermions) predict, for example, the heat capacity
  of a solid.}, or neutral fluids, including air itself.\footnote{In fact, no
  earlier than 2008 \citeauthor{Alvarez2008} gave a general dispersion relation
  for sound waves in air, which they show to be accurate for a variety of
  humidity levels and frequencies.}  I mean, come on: the reason rainbows appear
in the sky?  Dispersion of the electromagnetic waves we call
light!\footnote{Refraction is the answer from your high-school or freshman
  physics class, and that's right, but refraction itself is a result of
  dispersion.}

In order to forge ahead (and especially to understand what we discuss in
upcoming chapters), we've got to develop some understanding about what causes
wave dispersion, and such as my craft is, we've got to do it in the context of
plasma physics. Never fear! Many faithful guides have made the trek before us;
we're going to follow a few in right now, in fact: \citet{Goertz1979}, and
\citet{Kletzing1994}. My presentation is essentially a rip-off of theirs,
supplemented with commentary at the level of junior trail boss. Boots on?

\subsection[A ``big wheel'' derivation of the IAW dispersion
relation]{A ``big wheel''\protect{\footnote{Remember that trike with
      the fat wheel up front that your neighbor had when you were
      three?}} derivation of the inertial \Alf wave dispersion
  relation} \label{ch1:ssDerivation}

The math is about to get girthy---pack some real \emph{oomph}---and since a
proper, from-the-ground-up explanation would drag both of us into a mud bog
filled with heaving, steamy piles of calculus I will simply present the math in
full gore and comment upon the meaning as we go.\footnote{I used the 6th edition
  of James Stewart's \textsl{Calculus} to pick up the fundamentals as an
  undergraduate. Later on, I used (as has some non-negligible majority of
  physics majors in the past two decades) D.J. Griffiths's \textsl{Introduction
    to Electrodynamics} to develop vector calculus chops. Would I recommend
  them? Heartily.} That said, we serve ourselves well to mark our destination at
the outset. Here is the goal:
\begin{equation}
  \label{ch1:eqIAW}
  % \dfrac{\omega}{k_\parallel} = v_A \sqrt{\dfrac{1}{1 + k_\perp^2 \lambda_e^2}}
  % \dfrac{\omega}{k_\parallel} = v_A \big ( 1 + k_\perp^2 \lambda_e^2 \big )^{-1/2}
  \dfrac{\omega}{k_\parallel} = \dfrac{v_A}{ \sqrt{1 + k_\perp^2 \lambda_e^2 }}
\end{equation}

Not bad as I made it sound, right? Equation (\ref{ch1:eqIAW}) is the dispersion
relation for---guess what?---\emph{dispersive} \Alf waves, and more particularly
\emph{inertial \Alf waves} (I'll refer to them as IAWs),\footnote{This is just
  one of the flavors that dispersive \Alf waves come in. If you aren't satisfied
  with merely walking the beaten path and tend to prefer craggier climbs and
  more majestic views, you might follow \citeauthor{Lysak1996}'s [1996]
  \emph{generalized}(!) approach to dispersive \Alf waves. The bonus vista is
  discussion of imaginary terms for wave growth and damping.} in which you
immediately recognize the definition of the \Alf speed $v_A$ given in
(\ref{ch1:eqAlf}).  But oh---what's that $\perp$ doing, dangling off $k$ on the
right-hand side of (\ref{ch1:eqIAW})?  That, \textit{mes cheres amis}, indicates
the wave number that is \emph{perpendicular} to the magnetic field line. 

I hear you saying, ``But, but''---I know. I said that \Alf waves like to
propagate \emph{along} field lines, not across them. In the spirit of
Epimenides\footnote{The compatriot-slandering (and possibly mythical)
  philosopher credited with the statement that all Cretans are liars.} I tell
you, all physicists are liars and purveyors of half truths.\footnote{And in the
  spirit of Zeno, I ask: If a full truth exists and a physicist only reveals
  half, and a short while after the first revelation reveals only half of that
  which remains to be disclosed, and this pattern is continued \textit{ad
    infinitum}, may one ever hope to learn the full truth from a physicist?} A
portion of the undisclosed truth during our chat about the \Alf speed near
(\ref{ch1:eqAlf}) is that \Alf waves certainly can propagate at angles to the
magnetic field; they are more generally known as shear \Alf waves, and there is
a good body of literature showing the connection between shear \Alf waves and
IAWs.\footnote{Just a sampling:
  \citet{Heyvaerts1983,Lysak1996,Rankin1999,Drozdenko2000,Vincena2004,Tsiklauri2005,Chaston2007d,Watt2009}. Believe
  it.}

The layers of complication are that there are now \emph{two} terms involving
wave number, $k_\perp$ and $k_\parallel$, and the new term in parentheses
involving $k_\perp \lambda_e$, which serves to \emph{reduce} the \Alf speed
(observe: when $k_\perp \lambda_e \gtrsim$~1, the whole factor multiplying $v_A$
is less than 1). They aren't much to be worried about, but I should be careful
to point out that I have not said (and unfortunately will not say) \emph{a
  thing} about how to generate these waves. Many of the references given in the
footnotes are great resources on this topic, as are the aforementioned work by
\citet{Lysak1996} and \citet{Genot2004a}.

Before beginning to derive (\ref{ch1:eqIAW}), I've got to drag you through a
bunch of preliminaries. It's the theoretician's equivalent of bringing out that
table filled with knives, tools, saws, syringes, instruments for cauterizing
disinfecting, and suturing---you get the idea---at the side of a surgeon before
beginning an operation. Would you want the surgeon to go in with bare hands,
sans tools? That's what I thought.

The preliminaries are (1) the process of linearization, (2) introduction of the
equations and laws that we need to do the job right, and (3) a collection of
assumptions that will allow us to slice and dice the equations into submission.

% So let's open the bag and let the first cat out: $k_\parallel$ represents the
% wave number (which you recall from Section~\ref{subsec:k} is just the inverse of
% wavelength multiplied by 2$\pi$) that is parallel to the magnetic field.

% So let's open the bag and let the first cat out: $k_\parallel$ represents the
% wave number that is parallel to the magnetic field.

% We need to look at that old plasma theorist's standby, the continuity
% equation.
% \begin{equation}
%   \label{ch1:eqCont}
%   J_{E} = J_{E,\textrm{100--200 eV}} + J_{E,\textrm{200--500 eV}} + J_{E,\textrm{500--1200 eV}} + \dots .
% \end{equation}

\subsubsection{Preliminary \#1: ``Linearized what?''}

I'm going to introduce you to a little procedure for making life
simple---sometimes too simple---when doing theoretical work; in the biz, it's
what we call \emph{linearization}. Here's a question to illustrate the idea:
what happens if you pick a number that is small compared to 1---let's say
0.05---and then you square it? Right, you get a number that we might say is
\emph{really} small compared to 1---0.0025, if we start with 0.05.

For the sake of introducing terminology I'll say that 1 is a ``background''
value, and 0.05 is a ``perturbed'' value. With this language I can say that the
perturbation value 0.05 is small relative to the background value 1. Get it?
The next point to consider is that the product of two perturbed values is
\emph{really} small compared to the product of two background values. To be
concrete, $0.05^2 = 0.0025 \ll 1^2 = 1$.

Leaping headfirst into the physics, I introduce the \emph{background quantities} and
the \emph{perturbed quantities} \par
\begin{itemize}
\item \makebox[6.5cm]{$ n = n_o + \tilde n $} (number density),
\item \makebox[6.5cm]{$ \mathbf{v} = \cancel{\mathbf{v}_o} + \mathbf{\tilde v} $ } (velocity),
\item \makebox[6.5cm]{$ \mathbf{B} = \mathbf{B}_o + \mathbf{b} = B_o \hat{z} + \langle b_x, b_y, b_z \rangle $} (magnetic field).
\item \makebox[6.5cm]{$ \mathbf{E} = \cancel{\mathbf{E}_o} + \mathbf{\tilde E} =
    \langle \tilde E_x, 0, \tilde E_z \rangle $} (electric field).
\end{itemize}
\emph{Background quantities} have the ``o'' dangling off the front of them and
do not vary in time or space (i.e., they are constants). \emph{Perturbed
  quantities} are all marked with a tilde over the top with the exception of
$\mathbf{b}$, and they \emph{can} vary in time and space. 

I hope you noticed that I killed $\mathbf{E}_o$ and $\mathbf{v}_o$ before they
even got to show their true colors. I also murdalized $\tilde E_y$ without so
much as showing it the light of day. In this instantiation of the universe you
and I will have no need of them, and they are accordingly cast off as dross.

% \subsubsection{Preliminary \#1a:  How to kill nonlinear perturbed quantities: an oh-so-brief illustration}
\textbf{Preliminary \#1a:  How to kill nonlinear perturbed quantities: an oh-so-brief illustration}

With the foregoing ideas about linearized quantities in mind, let's look at the
number flux
\begin{equation}
  % n \mathbf{v} = \Big ( n_o + \tilde n \Big ) \Big ( \mathbf{v}_o + \mathbf{\tilde v} \Big ) = n_o \mathbf{v}_o + n_o \mathbf{\tilde v} + \tilde n \mathbf{v}_o + \tilde n \mathbf{\tilde v}.
  n \mathbf{v} = \Big ( n_o + \tilde n \Big ) \Big ( \mathbf{\tilde v} \Big ) = n_o \mathbf{v}_o + n_o \mathbf{\tilde v} + \tilde n \mathbf{\tilde v}.
\end{equation}
Can you spot the term on the right-hand side that involves the product of two
perturbed quantities? Yes, it's $\tilde n \mathbf{\tilde v}$, and we have to
kill this term because it is \emph{nonlinear} with respect to perturbed
quantities, meaning it is made up of the product of the perturbed variables
$\tilde n$ and $\mathbf{\tilde v}$.

Why deal in such a ruthless and merciless manner with nonlinear terms? Nonlinear
terms are the fastest, most basic, and most obvious route to making a
theoretical problem intractable that I am aware of, and we are going to rid
ourselves of them at every step possible. Especially if our aim is to keep
distracting details , they spell nothing but \emph{trouble}!

The \emph{linearized} number flux is therefore
\begin{equation}
  % n \mathbf{v} = n_o \mathbf{v}_o + n_o \mathbf{\tilde v} + \tilde n \mathbf{v}_o.
  n \mathbf{v} = n_o \mathbf{\tilde v} + \tilde n \mathbf{v}_o.
\end{equation}
See how nice that is? Only linearized terms.

% \subsubsection{Preliminary \#2: Meet the equations}
\subsubsection{Preliminary \#2: Meet the equations}

Now I'll introduce you to two of the old plasma theorist's standbys,
\begin{subequations}
  \begin{align} 
    \dfrac{\partial \tilde n_{\alpha}}{\partial t} + \grad \cdot \Big (n_{o,\alpha} \mathbf{\tilde v}_{\alpha} \Big ) &= 0, \label{ch1:eqCont} \\
    \dfrac{\partial \mathbf{\tilde v}_{\alpha}}{\partial t} &=
    \dfrac{q_{\alpha}}{m_{\alpha}} \Big ( \mathbf{E} + \mathbf{\tilde
      v}_{\alpha} \times \mathbf{B_o} \Big ), \label{ch1:eqMom}
  \end{align}
\end{subequations}
which are respectively the continuity equation and the momentum
equation.\footnote{The latter is sometimes referred to with affection as the
  ``mom equation.''} In order to focus on the physics of interest I have already
junked all terms that are nonlinear in perturbed quantities (i.e., the equations
are linearized) and thrown away a bunch of terms that describe the effects of
pressures asymmetric or otherwise,\footnote{Junking the pressure terms, by the
  way, amounts to assuming that the plasma is cold, or that the temperature
  $T_\alpha$ is effectively zero. It might seem like a bad assumption to make,
  but it turns out to be perfectly safe for our purposes, and for many others.}
and other, external forces.\footnote{Just to frighten you, let me flash these
  two equations before gutting the pressure terms and linearizing the others:
\begin{equation*}
  \begin{aligned} 
    \dfrac{\partial n_{\alpha}}{\partial t} + \grad \cdot \big (n_{\alpha} \mathbf{v}_{\alpha} \big ) &= 0; \\
    m_{\alpha} n_{\alpha} \Big ( \dfrac{\partial \mathbf{v}_{\alpha}}{\partial
      t} + \mathbf{v}_{\alpha} \cdot \grad \mathbf{v}_{\alpha} \Big ) + \grad
    p_{\alpha} + \grad \cdot \overleftrightarrow{\Pi}_{\alpha} &= q_{\alpha}
    n_{\alpha} \Big ( \mathbf{\tilde E} + \mathbf{v}_{\alpha} \times \mathbf{B} \Big )
    + \mathbf{F}_{\alpha}.
  \end{aligned}
\end{equation*}
In the momentum equation, $p_\alpha$ is the scalar pressure,
$\overleftrightarrow{\Pi}_{\alpha}$ the plasma tensor with off-diagonal terms,
and $\mathbf{F}_\alpha$ represents any other external force---for example, the
earth's gravitational field---that one might want to toss in. Compare these
full-blown equations with their linearized counterparts (\ref{ch1:eqCont}) and
(\ref{ch1:eqMom}), keeping in mind that $n_\alpha = n_{o,\alpha} + \tilde
n_\alpha$ and $\mathbf{v}_\alpha = \mathbf{v}_{o,\alpha} + \mathbf{\tilde
  v}_\alpha$. I hope you'll agree that the linearized equations are much less
spooky.}

These equations will serve as our description of the way that the collection of
particles constituting the plasma collectively respond to each other (``internal
forces,'' including the electric fields, magnetic fields, and currents that a
plasma can house), and to external forces, like a background magnetic field.

\textbf{Preliminary \#2a: the continuity equation}
% \subsubsection{Preliminary \#2a: the continuity equation}

In the continuity equation (\ref{ch1:eqCont}) you can see that the key players
are $n_{\alpha}$ and $\mathbf{\tilde v}_{\alpha}$, which are (perturbed and
background) number density and (perturbed) velocity. In the plainest English I
can muster, (\ref{ch1:eqCont}) simply says that no particles are created or
destroyed within the system under investigation. ``Um, what do you mean by
'system'?'' Don't be baffled by my language. When I say
``system''\footnote{Others might use ``setup'' or ``configuration'' instead of
  ``system.'' They're all the same in this context.} all I mean is the cast of
characters, the amount of time and space, and the level of detail necessary to
carry out a theoretical investigation\footnote{As opposed to experimental
  investigation.} of the phenomena I have in mind. Here, our cast members are
electrons and ions that are subject to a background magnetic field and to mutual
effects of attraction and repulsion, but are otherwise free to move around. It
turns out that this cast---electrons, ions, background magnetic field---can
provide a pretty good description of much of the plasma goings-on within and
around the terrestrial magnetosphere, as well as in the solar wind.

``But how many electrons and ions, and over what period of time and what region
of space?'' Prepare for a level of freedom that is often unsettling for
newcomers: we have as much time and space as we need, and $N$ electrons and
ions---by which I mean an unspecified number of electrons and ions that is
sufficiently huge for statistics about electrons and ions to be
meaningful. Think millions, billions, $10^{23}$---those sorts of numbers. We're
just fooling around on notepads and paper, so we can afford to dream big! Don't
bother asking yet again, ``But how many electrons and ions, exactly?'' I already
told you: $N$! We have $N$ of each.

I should acknowledge that it takes time to appreciate such abstract thinking,
but make no mistake about the value and power of the abstract. If you're still
worried and want to know if we'll ever come out of the clouds, the answer is
yes, when we want to apply our theoretical results to a particular situation. So
in a word, save your worries for \emph{application}!\footnote{Or if you're a
  theorist, leave all the worrying to the experimentalists! They'll let you know
  where you screwed up, weren't detailed enough, or otherwise made fatal
  omissions or assumptions.}

As long as we're talking about the abstract, what about the subscript $\alpha$?
It can indicate electrons or ions (i.e, $\alpha = e,i$). To be more explicit,
whenever you see the subscript $\alpha$ in an equation you should be cognizant
that there are really \emph{two} equations on display---one for electrons and
one for ions. I use $\alpha$ just to keep things compact. My dual-purpose of
$\alpha$ is an example of what physicists might refer to as ``notational
convenience.''

But where were we? Oh yes, as far as (\ref{ch1:eqCont}) is
concerned,\footnote{And of course neglecting whatever the momentum equation
  (\ref{ch1:eqMom}) dictates about particle motion.}  particles are free to move
all over the system, bunch into heaping piles, spread out uniformly---you name
it, pal. But if we start with $N$ total particles in our system, we must
maintain at all times and end with $N$ particles, let the number density at a
particular time and location, $\tilde n_{\alpha}(\mathbf{x},t)$, vary how it
will.\footnote{In math speak I am saying that from eternity to eternity,
  $\int_{\mathrm{sys}} d^3\mathbf{x} \, n(\mathbf{x},t) = N$ regardless of the
  variation of number density in time and space within the confines of our
  system.}

% \subsubsection{Preliminary \#2b: the momentum equation, and two questions on validity}
\textbf{Preliminary \#2b: the momentum equation, and two questions on validity}

How about the momentum equation (\ref{ch1:eqMom})? It only has two things going
on: the left-hand side is the temporal evolution (represented by $\partial
/ \partial t$ of the velocity perturbations $\mathbf{\tilde v}_\alpha$, and the
right-hand side describes the effect of the perturbation electric field
$\mathbf{\tilde E}$ and the background magnetic field $\mathbf{B}_o$ (which
together comprise what is called the \emph{Lorentz force}) on electrons and
ions.

The next customer is (\ref{ch1:eqMom}), which, like (\ref{ch1:eqCont}), I have
already linearized.\footnote{}


Two questions now arise. They are questions so worthy that if you haven't
thought to ask, I'm asking on behalf of us all: Where did (\ref{ch1:eqCont}) and
(\ref{ch1:eqMom}) come from? Are they laws of some sort? 

To the first question, (\ref{ch1:eqCont}) and (\ref{ch1:eqMom}) came from the
\emph{Vlasov equation}, also known as the ``collisionless Boltzmann
equation.''\footnote{Wanna see it?  Here's the nonrelativistic version:
  $\frac{\partial f}{\partial t} + \mathbf{v} \cdot \frac{\partial f}{\partial
    \mathbf{x}} + \mathbf{a} \cdot \frac{\partial f}{\partial \mathbf{v}}$.
  $f = f (\mathbf{x}, \mathbf{v}, t)$ is known as a distribution function, and
  it gives the density of particles at $\mathbf{x}$---which is the position
  vector---having velocity $\mathbf{v}$.} The Vlasov equation completely ignores
particle-particle (or pair) collisions. (Hence the ``collisionless'' in
``collisionless Boltzmann equation.'') Believe it or not, it is often valid to
assume space plasmas are collisionless.\footnote{But not always!  When the
  assumption is not valid one turns of necessity to the ``collisionful'' (my own
  neologism) Boltzmann equation, which is known for leaving bruised, black-eyed,
  and bloodied physicists in its wake.} On the other hand, the Vlasov equation
allows for long-range\footnote{As opposed to ``short-range'' interaction, or
  direct collisions between particles.}  interaction between all particles via
the Coulomb force.\footnote{The eponymous Coulomb force, after Charles-Augustin
  de Coulomb, just describes the repulsive or attractive force between charged
  particles: $F_{\textrm{Coul}} = k \frac{q_1 q_2}{r^2}$. Can you see that the
  force is positive (i.e., repulsive) if $q_1$ and $q_2$ have the same sign, and
  negative (i.e., attractive) otherwise? Coulomb's law is scientific bedrock.}

To the second question, I suppose in a manner of speaking we could call
(\ref{ch1:eqCont}) and (\ref{ch1:eqMom}) laws---or rather, you can think of them
as true statements with the following proviso: At no point in the course of our
using (\ref{ch1:eqCont}) and (\ref{ch1:eqMom}) are we permitted to treat a
situation violating the assumptions that brought them about, which I summarized
in the previous paragraph. I am not going to launch into an exposition of the
Vlasov equation, or else we'd never get out of here,\footnote{You can turn to
  someone like \citet{Chen1974} or \citet{Bellan2008} for that.} but I will
broadly warn that we wander into dangerous territory when we take imprudent
liberty and practice Blind application Of Formulas---let's call it ``boffing,''
and those who practice it ``boffers.'' I will try to point out where the
unsuspecting may be led into boffing, though all the ways in which one might
possibly boff are far too many to hope to enumerate, or (as I have learned by
sad experience) even know.\footnote{Undergraduates are among the most expert
  boffers, having seemingly limitless capacity to sleuth out and demonstrate new
  ways to boff. It is tempting to believe (though I know better) that they take
  up boffing as an innocent-seeming, though ultimately sinister, pastime
  calculated to harm their physics professors and teaching assistants.}

I would not leave you hanging without at least one concrete example to
demonstrate when one ought to become suspicious of (\ref{ch1:eqCont}) and
(\ref{ch1:eqMom}): the ionosphere. In the ionosphere, where the electron number
density is waaay higher than in the magnetosphere, the total number of particles
$N$ can change all the time\footnote{I.e., the right-hand side of
  (\ref{ch1:eqCont}) is not zero.} as a result of---oh, I don't know, let's say
sunlight-induced photoionization of ionospheric particles. Strictly speaking,
when the total number of particles changes the continuity equation
(\ref{ch1:eqCont}) is no longer true.

In short, once we have decided to make the assumptions that make
(\ref{ch1:eqCont}) and (\ref{ch1:eqMom}) valid, there's no going back; we can't
introduce elements to our model that end up violating those assumptions. And now
you know: (\ref{ch1:eqCont}) and (\ref{ch1:eqMom}) are \emph{not} always
true. They are not laws of nature, but they tell us how particles move around in
a plasma in situations where it is safe to neglect particle-particle collisions,
and where the number of particles $N$ is sufficiently large to render valid a
statistical treatment of the plasma.\footnote{You'd have to venture into a
  textbook on statistical mechanics to get a feel for what it means for $N$ to
  be ``sufficiently large.''}

Fewf! This is a good place to stop and do some ruminating, both on what I've
just told you and on a sandwich, if so inclined. Otherwise, let's continue.

% \subsubsection{Preliminary \#2c: a subset of Maxwell's equations}
\textbf{Preliminary \#2c: a subset of Maxwell's equations}

Besides (\ref{ch1:eqCont}) and (\ref{ch1:eqMom}), we need two of Maxwell's
equations,\footnote{The four laws constituting Maxwell's equations---Gauss's
  law, Divergenceless B, Faraday's law, and Amp\`{e}re's law---are the
  foundation of classical electrodynamics. As gates of understanding through
  which all would-be physicists must pass, and given both their ubiquity and
  their overall degree of importance in more fields than I am aware exist, in my
  opinion emotional and reverential responses to the mention of their names are
  not improper.}
\begin{subequations}
  \begin{align} \grad \times \mathbf{E} &= - \dfrac{\partial \mathbf{b}}{\partial t}; \label{ch1:eqFar} \\
    \grad \times \mathbf{b} &= \mu_o \mathbf{\tilde j}. \label{ch1:eqAmp}
  \end{align}
\end{subequations}
In contrast to (\ref{ch1:eqCont}) and (\ref{ch1:eqMom}) which I continue to
emphasize are \emph{not} laws of nature, (\ref{ch1:eqFar}) and (\ref{ch1:eqAmp})
\emph{are} laws. As far as I know, they are genuinely immutable.\footnote{Some
  of you will note I've discarded Maxwell's famous addition to
  (\ref{ch1:eqAmp}), the displace current $\mu_o \epsilon_o \frac{\partial
    \mathbf{E}}{\partial t}$. We're assuming our system evolves slowly enough
  that we can neglect it.}

The first of these, (\ref{ch1:eqFar}) or Faraday's law, is telling you that
wherever you see an electric field having nonzero curl,\footnote{The concept of
  ``curl''---represented in (\ref{ch1:eqFar}) and (\ref{ch1:eqAmp}) as $\grad
  \times$---comes from vector calculus, but you and I know that vector calculus
  is totally contrived, a concession to our human inabilities to think of a
  better description! Here's my attempt to circumvent the feeling that vector
  calculus could be a fiction: Can you see a tornado in your mind? If so, take a
  closer look at the flow of winds around the eye of a tornado. The winds swirl
  around the eye, right? More precisely stated, if you drew a circle just
  outside the boundary of and centered on the eye, and measured the direction
  that the wind blows at each point around the circle, would you not find at
  every point that the direction that the wind blows is tangent to the circle?
  Well, that's a cheap example of a curling, or ``vortical,'' field of
  vectors. (The vector field in this case is fluid flow.)  } somewhere in the
vicinity there is a corresponding magnetic field changing in time. The second,
(\ref{ch1:eqAmp}) or Amp\`{e}re's law, is telling you that a magnetic field
$\mathbf{b}$ with nonzero curl is the same as, and cannot be separated from, a
current density $\mathbf{\tilde j}$---up to a constant of proportionality
$\mu_o$, at any rate. Together, (\ref{ch1:eqFar}) and (\ref{ch1:eqAmp}) tell us
how electric fields and magnetic fields evolve in time, and they are
\emph{laws}.

\subsubsection{Preliminary \#3: Taming the equations with simplifying assumptions} \label{ch1:sssAssume}

Now for one of the fun parts. We get to make a bunch of assumptions about (for
example) the way that magnetic and electric fields vary in time, and how the
constituent particle species (which are ions and electrons) move and interact,
all of which whittle away the unnecessary details of
(\ref{ch1:eqCont}--\ref{ch1:eqFar}). At least right here and right now, my
motive in making any assumption is to save us from distraction by unnecessary
details in the course of investigation. The commonness of this practice is the
reason you will sometimes hear reference made to ``simplifying assumptions''; I
have yet to hear reference made to ``complicating assumptions.''

First of all, I propose a Cartesian coordinate system\footnote{We could use a
  dipolar coordinate system, as do \citet{Lysak2013}, but I'd rather keep it
  simple for the sake of both of us.} such that the $z$ coordinate is along the
local geomagnetic field $\mathbf{B}_o$ (that is, $\mathbf{B}_o = B_o \hat{z}$),
the $x$ coordinate is perpendicular to the local geomagnetic field, and the $y$
coordinate is negligible (i.e., nothing varies in the $y$ direction, which means
that inside the symbol $\grad$, the differential operator $\partial / \partial y
= 0$). However, I won't outrightly say that the $y$ coordinate doesn't exist
since, among other issues, the curl operator $\grad \times$ is a
three-dimensional creature and doesn't make sense if we are strict in saying
that there are only two dimensions in our system.

So much for spatial variation. What to say about variation in time? I shouldn't
even talk about temporal variation without first introducing you to a
bread-and-butter concept in plasma physics, cyclotron motion. It's like this: a
charged particle that is moving perpendicular to a steady magnetic field (and
that is free to move without bumping into a neighboring particle) will begin to
gyrate---or execute periodic, circular motion---around the magnetic field line
as a consequence of the (linearized) magnetic force
\begin{equation} \label{ch1:eqMagForce} 
\mathbf{F}_m = q_\alpha \mathbf{\tilde v}_\alpha \times \mathbf{B}_o.
\end{equation} 
If the particle has some velocity component along the magnetic field, its path
around the magnetic field line is helical. (Never mind what happens when an
electric field is also present, since it complicates the description I just
gave.) The frequency at which this gyration about the field line occurs is the
cyclotron frequency $\omega_{c \alpha} = \vert q_\alpha B /m_\alpha \vert$.
Also, see how the direction of $\mathbf{F}_m$ depends on the sign of $q_\alpha$
in (\ref{ch1:eqMagForce})? It means that the force due to the magnetic field
$\mathbf{B}_o$ on electrons, which are negatively charged, is in a direction
opposite the force exerted on ions, which are positively charged; electrons and
ions thus gyrate about the magnetic field line in opposite senses.

Back to our question about variation in time. You can see that
$\omega_{ci} / \omega_{ce} = m_e/m_i$, or in English, that the ion-cyclotron
frequency $\omega_{ci}$ is less than the electron-cyclotron frequency
$\omega_{ce}$ by a factor of the ion-electron mass ratio. There are therefore
three frequency ranges implied by $\omega_{ce}$ and $\omega_{ci}$, \par
\begin{enumerate}
\item \makebox[4.5cm]{\qquad $ \partial / \partial t \gg \omega_{ce} $} (high frequencies $\leftrightarrow$ short timescales),
\item \makebox[4.5cm]{$ \omega_{ci} < \partial / \partial t < \omega_{ce} $} (intermediate frequencies $\leftrightarrow$ intermediate timescales),
\item \makebox[4.5cm]{\qquad $ \partial / \partial t \ll \omega_{ci} $} (low frequencies $\leftrightarrow$ long timescales).
\end{enumerate}
We'll assume that the electric and magnetic field as well as the motion of
particles in our system vary at low frequencies (option \#3) so that
$\partial^2 / \partial t^2 \ll \omega_{ci}^2$.


It is fair for me to assume that throughout the scenario I'm about to concoct,
the variation in density is always small compared to the background density

I'm also going to assume the existence of a (perturbation) electric field
$\mathbf{\tilde E}$ that only varies in the $x-z$ plane:
$\mathbf{\tilde E} (x, z, t) = \langle \tilde E_x (x, z, t), 0, \tilde E_z (x,
z, t) \rangle$.
(And yes, I have assumed $\tilde E_y = 0$; we simply don't need $\tilde E_y$, so why bother
with it?)

Last of all, I will assume a singly charged ion species so that $q_i = q$ (and
of course $q_e = -q$).

\subsubsection{Without further ado, the derivation}

With the linearized equations in hand and the baby mountain of assumptions I
have just made, let's bob and weave, baby. Start by multiplying each continuity
equation by the corresponding charge of the species $q_\alpha$---positive for
ions and negative for electrons---and adding the two continuity equations
(\ref{ch1:eqCont}) together (remember that there are really two equations, since
$\alpha$ denotes electrons or ions),
\begin{equation} \label{ch1:eqCont1} \dfrac{\partial}{\partial t} \Big (
  \tilde n_i - \tilde n_e \Big ) + \grad \cdot \Big (n_{o,i} \mathbf{\tilde v}_i - n_{o,e}
  \mathbf{\tilde v}_{e} \Big ) = 0.
\end{equation}
If we now recall the quasineutrality condition (\ref{ch1:eqQuasi}) you can see
that the terms involving the time derivative of density go to zero (i.e., $n_i -
n_e \approx 0$) and if I define the (perturbation) current density $\mathbf{\tilde j} \equiv q n_o (
\mathbf{v}_i - \mathbf{v}_e )$, the divergence term (that's the symbol $\grad
\cdot$ and all the stuff in parentheses that follows after) can be rewritten
$\grad \cdot \mathbf{\tilde j}$ :
\begin{equation*} 
  q\dfrac{\partial}{\partial t} \Big ( \cancel{n_i - n_e} \Big )
  + q n_o \grad \cdot \Big ( \mathbf{\tilde v}_i - \mathbf{\tilde v}_{e} \Big ) = \grad
  \cdot \mathbf{\tilde j} = 0.
\end{equation*}

For to-be-disclosed reasons, I take the time derivative of $\grad \cdot
\mathbf{\tilde j} = 0$ and write out the components of the divergence term
\emph{as\'{i}}:
\begin{equation} 
  \label{ch1:eqCont2} \dfrac{\partial^2 \tilde j_x}{\partial x \partial t} + \dfrac{\partial^2 \tilde j_z}{\partial z \partial t} = 0.
\end{equation}
You'll recall that our assumption, $\partial / \partial y = 0$, accounts for the
conspicuous absence of any terms or derivatives involving $y$. Let's keep this
equation in our back pocket and keep moving.

How about taking a look at the $y$ component of the momentum equations
(\ref{ch1:eqMom})? This suggestion might seem strange since I've explicitly
assumed that $\partial / \partial y = 0$, but hang on to your hat: writing out
the $y$ components explicitly for each species, we find that
\begin{equation} 
  \begin{alignedat}{2}
    \label{ch1:eqMomy} \dfrac{\partial \tilde v_{y,e}}{\partial t} &=
    &\omega_{ce} \tilde v_{x,e} &= 0; \\
    \dfrac{\partial \tilde v_{y,i}}{\partial t} &=
    - &\omega_{ci} \tilde v_{x,i} &= 0. \\
  \end{alignedat}
\end{equation}
Why are the right-hand sides of (\ref{ch1:eqMomy}) zero?  Well, if they weren't
zero the possibility of a time-varying current density $\tilde j_y \neq 0$
arises, which we simply won't hear of! This isn't to say that nonzero
$\tilde j_y$ is impossible---of course it's \emph{possible}---but once we're
through deriving a dispersion relation for IAWs, you'll see that
a nonzero $\tilde j_y$ would amount to a distraction. In the meantime I ask you
to trust that I have the benefit of foresight. Plus, if I assume
$\tilde j_y = 0$ I can equate the two expressions (\ref{ch1:eqMomy}) to find
that
\begin{equation} 
  \tilde v_{x,e} = - \dfrac{m_e}{m_i} \tilde v_{x,i}.
\end{equation}
Ah, so! In this case the $x$ direction electron motion is opposite that of the
ions, and at a speed that is less than that of ions by a factor of $m_e /
m_i$. (In plain English: electrons move much, much more slowly than ions in the
$x$ direction.) This is somewhat counterintuitive since electrons are much
lighter than ions and presumably a lot easier to shove around, but that's why
the math and physics do the driving and intuition takes the backseat.

Next, why don't we take a look at the $x$ component of the momentum equation
(\ref{ch1:eqMom})? It is
\begin{equation} \label{ch1:eqMomx} \dfrac{\partial \tilde v_{x,\alpha}}{\partial t} =
  \dfrac{q_\alpha}{m_\alpha} \Big ( \tilde E_x + \tilde v_{y,\alpha} B_o - \cancel{\tilde v_{z,\alpha} b_y}
  \Big ) = \dfrac{q_\alpha}{m_\alpha} \tilde E_x \pm \omega_{c \alpha} \tilde v_{y,\alpha},
\end{equation}
where I've junked the $\tilde v_{z,\alpha} b_y$ term because it is \emph{nonlinear}
(i.e., the product of two small quantities, as mentioned above), and the upper
sign on the right-hand side is for ions and the lower sign is for electrons,
here and in the future. The extra signage is necessary to account for the
charge-dependent direction of the magnetic force (\ref{ch1:eqMagForce}).

Can you see that if I take the time derivative of (\ref{ch1:eqMomx}),
$\frac{\partial \tilde v_{y,\alpha}}{\partial t}$ will appear on the right-hand side?
I can then ladle in (\ref{ch1:eqMomy}) for $\frac{\partial
  \tilde v_{y,\alpha}}{\partial t}$,
\begin{equation*} \label{ch1:eqMomxdt} \dfrac{\partial^2 \tilde v_{x,\alpha}}{\partial
    t^2} = \dfrac{q_\alpha}{m_\alpha}\dfrac{\partial \tilde E_x}{\partial t} \pm
  \omega_{c \alpha} \dfrac{\partial \tilde v_{y,\alpha}}{\partial t} =
  \dfrac{q_\alpha}{m_\alpha}\dfrac{\partial \tilde E_x}{\partial t} - \omega_{c
    \alpha}^2 \tilde v_{x,\alpha},
\end{equation*}
and moving the term at far right to the left-hand side,
\begin{equation*} \Big ( \dfrac{\partial^2}{\partial t^2} + \omega_{c \alpha}^2
  \Big ) \tilde v_{x,\alpha} = \dfrac{q_\alpha}{m_\alpha}\dfrac{\partial \tilde E_x}{\partial
    t},
\end{equation*}
I draw on my assumption that only low frequencies are of interest so that
$\partial^2 / \partial t^2 + \omega_{c i}^2~\approx~\omega_{c i}^2$. Thus
\begin{equation} \label{ch1:eqvxEx}
  \tilde v_{x,\alpha} = \pm \dfrac{1}{\omega_{c \alpha} B_o}\dfrac{\partial
    \tilde E_x}{\partial t}.
\end{equation}

Only the $z$ component of (\ref{ch1:eqMom}) remains to be examined. It's a cakewalk:
\begin{equation} \label{ch1:eqvzEz} \dfrac{\partial v_{z,\alpha}}{\partial t} =
  \dfrac{q_\alpha}{m_\alpha} \Big ( \tilde E_z + \cancel{\tilde v_{x,\alpha} b_y} -
  \cancel{\tilde v_{y,\alpha} b_x} \Big ) = \pm \dfrac{\omega_{c \alpha}}{B_o} \tilde E_z.
\end{equation}
Baller, right? For my next trick, I'll plug (\ref{ch1:eqvxEx}) and
(\ref{ch1:eqvzEz}) into the appropriate components of $\mathbf{\tilde j}$,
\begin{equation*} 
  \begin{alignedat}{4}
    \tilde j_x                              &= q n_o &\Big ( \tilde v_{x,i} - & \tilde v_{x,e} \Big ) = & \dfrac{n_o}{B_o^2} &\Big( m_i + m_e \Big) \dfrac{\partial \tilde E_x}{\partial t}; \\
    \dfrac{\partial \tilde j_z}{\partial t} &= q n_o \dfrac{\partial}{\partial t} &\Big ( \tilde v_{z,i} - & \tilde v_{z,e} \Big ) =& q^2 n_o &\Big( \dfrac{1}{m_i} + \dfrac{1}{m_e} \Big) \tilde E_z. \\
  \end{alignedat}
\end{equation*}
Now we do the famous ``drop all terms of order [uninteresting science].'' In
this case I'll drop all terms of order $m_e/m_i$, which means that I tug and
pull at the expressions for $\tilde j_x$ and $\tilde j_z$ that I just produced
to try to make the combination $m_e/m_i$ appear. When it does, I blast the whole
term to which it is joined into next week (i.e., set it to zero)! Here, take my
pistol\footnote{A pencil \emph{is} a theorist's pistol, don't you know.} and
give it a shot; you should find
\begin{equation} \label{ch1:eqDaddy}
  \begin{alignedat}{4}
    \tilde j_x                              &= \dfrac{m_i n_o }{B_o^2} \dfrac{\partial \tilde E_x}{\partial t}; \\
    \dfrac{\partial \tilde j_z}{\partial t} &= \dfrac{q^2 n_o }{m_e} \tilde E_z. \\
  \end{alignedat}
\end{equation}

I'm sure you're feeling more and more disoriented: ``Where on earth is he taking
us? I've always hated camping and I wish I were at home playing video games.''
That's OK---the type of exploration we embark on now is best suited for the
hardier and more sanguinely disposed wayfarer. You're better off leaving now if
you aren't rounding your shoulders and stiffening your back, since we are about
to enter a brief but veritable thicket of maths\footnote{An intentional nod to
  the English-speaking world outside the United States.} in which you are almost
definitely going to lose your sense of direction if you've never before
traversed such terrain.

If I plug the expressions (\ref{ch1:eqDaddy}) for $\tilde j_x$ and $\tilde j_z$ into
(\ref{ch1:eqCont2}), I obtain the seemingly meaningless result
\begin{equation*}
  % \begin{alignedat}{4}
    \dfrac{\partial^2}{\partial t \partial x} \Big( \tilde j_x \Big) + \dfrac{\partial}{\partial z}
    \Big( \dfrac{\partial \tilde j_z}{\partial t} \Big) = \dfrac{\partial^2}{\partial
      t \partial x} \Big(\dfrac{m_i n_o }{B_o^2}\dfrac{\partial \tilde E_x}{\partial t} \Big) +
    \dfrac{\partial}{\partial z} \Big( \dfrac{q^2 n_o}{m_e} \tilde E_z \Big) = 0.
  % \end{alignedat}
\end{equation*}
If I multiply the whole thing by $\mu_o c^2$ I get 
\begin{equation} \label{ch1:eqMeaningless}
  \dfrac{c^2}{v_A^2}\dfrac{\partial^3 }{\partial t^2 \partial x} \tilde E_x + \dfrac{1}{\omega_{pe}}\dfrac{\partial}{\partial z} \tilde E_z = 0,
\end{equation}
which you have to admit is a little better; old friends like $v_A$ and the speed
of light $c$ are also out in the woods today! Say, have I introduced you to that
leathery-looking bloke standing underneath $\tilde E_z$? That's the electron
plasma frequency
\begin{equation*}
\omega_{pe} = \sqrt{ \dfrac{n_o q^2}{m_e \epsilon_o}};
\end{equation*}
in terrestrial plasmas it is well above $\omega_{ci}$. $\omega_{pe}$ has not
finished rearing its head, so don't completely forget about it.

To really start a grease fire we need to get together with the $y$ component of
Faraday's law,
\begin{equation*}
  \Big ( \grad \times \mathbf{\tilde E} \Big )_y = \dfrac{\partial \tilde E_x}{\partial z} - \dfrac{\partial \tilde E_z}{\partial x} = - \dfrac{\partial b_y}{\partial t},
\end{equation*}
the expressions for $\tilde j_x$ and $\frac{\partial \tilde j_z}{\partial t}$ in
(\ref{ch1:eqDaddy}), and the $x$ component of Amp\`{e}re's law. The left-hand
side of the $x$ component of Amp\`{e}re's law is
\begin{equation*}
  \dfrac{\partial}{\partial t} \Big ( \grad \times \mathbf{B} \Big )_x = \dfrac{\partial}{\partial t} \Big ( \cancel{\dfrac{\partial}{\partial y} B_o} - \dfrac{\partial}{\partial z} b_y \Big ) = \dfrac{\partial}{\partial z} \Big ( - \dfrac{\partial b_y}{\partial t} \Big ) = \dfrac{\partial^2 \tilde E_x}{\partial z^2} - \dfrac{\partial^2 \tilde E_z}{\partial x \partial z},
\end{equation*}
while the right-hand side is 
\begin{equation*}
  \mu_o \dfrac{\partial}{\partial t} \tilde j_x = \mu_o \dfrac{\partial}{\partial t} \Big ( \dfrac{n_o m_i}{B_o^2} \dfrac{\partial \tilde E_x}{\partial t} \Big ) = \mu_o \Big ( \dfrac{n_o m_i}{B_o^2} \Big ) \dfrac{\partial^2 \tilde E_x}{\partial t^2} = \dfrac{1}{v_A^2} \dfrac{\partial^2 \tilde E_x}{\partial t^2}.
\end{equation*}
Equating each side, 
\begin{equation} \label{ch1:eqMeaningless2} \dfrac{\partial^2 \tilde
    E_x}{\partial z^2} - \dfrac{\partial^2 \tilde E_z}{\partial x \partial z} =
  \dfrac{1}{v_A^2} \dfrac{\partial^2 \tilde E_x}{\partial t^2},
\end{equation}
we obtain our second seemingly meaningless result.

And now the fire is burning, and you're utterly bewildered! Don't stop there;
we've got to take the derivative of (\ref{ch1:eqMeaningless}) with respect to
$x$ and divide by $\omega_{pe}^2$, which will give us a term
$-\frac{\partial^2 \tilde E_z}{\partial x \partial z}$ that we can slip into
(\ref{ch1:eqMeaningless2}). Take heart, tenderfoot; we're near the top! Taking
the derivative with respect to $x$ and dividing by $\omega_{pe}^2$, we find
\begin{equation*}
  -\dfrac{\partial^2 \tilde E_z}{\partial x \partial z} = \dfrac{c^2}{\omega_{pe}^2 v_A^2} \dfrac{\partial^4 \tilde E_x}{\partial x^2 \partial z^2}.
\end{equation*}
See how nicely that'll slip right into (\ref{ch1:eqMeaningless2}) and rid us of any terms having to do with smelly old $\tilde E_z$? Go
ahead and slap it in, multiplying both sides by $\frac{\omega_{pe}^2}{v_A^2}$ as you go, to find
\begin{equation} \label{ch1:eqAlmost}
  \Big [ \omega_{pe}^2 \Big ( \dfrac{\partial^2}{\partial z^2} - \dfrac{1}{v_A^2} \dfrac{\partial^2}{\partial t^2} \Big ) + \dfrac{c^2}{v_A^2} \dfrac{\partial^4}{\partial x^2 \partial t^2} \Big ] \tilde E_x = 0.
\end{equation}

Whoa: a millionth-order differential equation in $\tilde E_x$---all right,
fourth-order equation, but who has actually heard of those? It doesn't matter;
let's slay the beast in a single stroke by making the cheapest
assumption\footnote{The experienced will recognize this assumption as the good
  old \emph{plane wave ansatz}. To quote Jimi Hendrix, ``Are you experienced?''}
about $E_x$ that we possibly could,
\begin{equation*}
  \tilde E_x = \tilde E_{x,o} \mathrm{exp} \Big ( k_\perp x + k_\parallel z - \omega t \Big ),
\end{equation*}
so that 
\begin{equation} \label{ch1:bro}
  \begin{alignedat}{2}
\frac{\partial \tilde E_x}{\partial t} &=  -i \omega \tilde E_x; \\
\frac{\partial \tilde E_x}{\partial x} &=   i k_\perp \tilde E_x; \\
\frac{\partial \tilde E_x}{\partial z} &=   i k_\parallel \tilde E_x. \\
  \end{alignedat}
\end{equation}

OK, let's take it to the top! Plugging the expressions (\ref{ch1:bro}) into
(\ref{ch1:eqAlmost}) finally gives
\begin{equation} \label{ch1:bro}
  \Big [ \omega_{pe}^2 \Big ( \dfrac{\omega^2}{v_A^2} - k_\parallel^2 \Big ) + \dfrac{c^2}{v_A^2} k_\perp^2 \omega^2 \Big ] \tilde E_x = 0,
\end{equation}
and casting $\tilde E_x$ aside like the piece of garbage it has shown itself to
be, we can easily rearrange the spoils of victory to show
\begin{equation}
  \dfrac{\omega}{k_\parallel} = \dfrac{v_A}{ \sqrt{1 + k_\perp^2 \lambda_e^2 }},
\end{equation}
which was to be demonstrated, and is triple baller. (Oh, and $\lambda_e = c /
\omega_{pe}$.)

\subsubsection{What exactly does the plasma skin depth $\lambda_e$ mean?} 

Now that we've arrived let's survey the scene a little. Hey, tell me what you
see if you set $k_\perp \lambda_e = 0$. \dots What's that? You say it bears
uncanny resemblance to (\ref{ch1:eqAlf})? Excellent observation! I guess it
means that when the mood is right, the IAW dispersion relation can be
equivalent to the dispersion relation (\ref{ch1:eqAlf}) for nondispersive \Alf
waves. But what role does $\lambda_e = \frac{c}{\omega_{pe}}$ play?  Why is it
relevant? More specifically, why are the waves nondispersive if $k_\perp
\lambda_e \ll 1$?

Let's come at the question from a different angle. Why does an \Alf wave need to
have a perpendicular wavelength $\lambda_\perp \lesssim 2 \pi \lambda_e$ to be
dispersive? Well I hope you're sitting down, because I'm going to tell
you. $\lambda_e$ is the electron inertial length, or \emph{plasma skin depth},
OK? Now, forget plasmas for a moment. In general terms, if you throw
electromagnetic waves of increasing frequency at a conductor, the conductor acts
increasingly resistive, and vice versa. There will be some high-frequency limit
above which the current density associated with an electromagnetic wave is
essentially confined to the surface, or ``skin,'' of a conductor. On the other
hand, for some incredibly low frequency the conductor is lulled into a false
sense of security by the siren-like movement of the wave, and the conductor lets
the wave \emph{right inside}---that is, inside itself. These are the two
extremes.

So, in general terms, what is the skin depth of a conductor? It is a measure of
how the conductor in question is going to deal with a wave of a particular
frequency that you try to throw at it, or into it. Normally when people have a
discussion about skin depths (e.g., in a course on E\&M or an electrical
engineering course or something like that) they're talking about the more common
skin depth $\delta = \sqrt{2 / \sigma \mu \omega}$, which says, ``If you tell me
the frequency you're interested in, I'll tell you how deeply into the conductor
the current associated with that frequency can penetrate.'' In math (ignoring
wave phase) the answer is, in a vanilla-ized scenario, that the current density
at a depth $d$ in the conductor is $J (d) = J_s \mathrm{exp} ( - d / \delta )$,
where $J_s$ is the current density at the surface of the conductor.

Leaving generalities and coming back to our plasma scenario, $\lambda_e$ is
indeed a curiosity. Even though $\lambda_e$ \emph{is} a skin depth, it is
\emph{fixed} by the plasma frequency $\omega_{pe}$. Contrast this with the
expression for the frequency-dependent skin depth $\delta$ in the previous
paragraph. With $\delta$, you get to dictate which frequency you ask about. With
$\lambda_e$ you don't. The plasma wants to conduct\footnote{Pun!} its business
at $\omega_{pe}$---that's the natural frequency of oscillations for the plasma,
you see---and instead of asking what the plasma is going to do with a particular
frequency, instead you can ask what the plasma is going to do with a particular
\emph{wavelength}, or rather a particular perpendicular wave number $k_\perp$:
$J ( k_\perp ) = J_s \mathrm{exp} ( - / k_\perp \lambda_e )$.  See? The current
density associated with an IAW is as good as dead in a plasma if $k_\perp
\lambda_e \ll 1$! But never mind my musings, talk to \citet{Morales1994} if you
want the skinny (depth). Here's a snippet from their abstract to get you
excited:

\begin{displayquote}
  Shear \Alf waves having transverse scale on the order of the electron skin
  depth exhibit a collisionless divergence determined by propagation cones that
  emanate from the edges of the exciting structures. Axial current channels are
  found to spread radially due to the skin effect up to the cone trajectories
  and at distances of a few wavelengths from the exciter develop radial
  diffraction patterns.
\end{displayquote}

Oh yeah!

\section{From waves to electrons}

Let's talk about the aurora then. Even during geomagnetically quiet times,
roughly 20~GW\footnote{``How much is that?'' The average rate of energy
  consumption per household in the United States is $\sim$1.4~kW (as of 2014,
  according to the World Energy Council's ``Average electricity consumption per
  electrified household''). So 20~GW is enough to power roughly
  $\sim$\numprint{14000} homes in the United States. (For a more conservative
  rate of consumption such as that in New Zealand, 0.84~kW, 20~GW could cover
  almost \numprint{24000} homes.)} \citep{Newell2009} are raining down on the
poles\footnote{Magnetic, not geographic.} of the earth through various types of
aurora, and so-called ``wave aurora''\footnote{Because, you know, it's powered
  by waves. (And not just any kind of wave: \Alf waves!)} contributes about 5\%
of the total energy budget. When things really start moving---for example,
during a geomagnetic storm---the electron precipitation associated with \Alf
waves increases quite a bit, and broadband electron energy and number flux can
become sizable fractions of the electron precipitation budget.

\subsection{You mean ``broadband'' like DSL?}

But just what are broadband electrons, and what do they have to do with \Alf
waves? The name refers to the tell-tale signature they leave when we plot their
differential energy flux---a nasty-sounding term, I know.

Here's a sketch: Pick up a piece of paper. Imagine that during the course of one
second, an electron comes zipping through it. We'll use this to define a unit
\emph{number flux}, or \emph {1 electron per sheet of paper per second}--we'll
call them \emph{epps}. So if over the course of two seconds we counted ten
electrons come through the page, we'd say the average number flux during those
two seconds was 5 epps. Fair enough?  Now, if instead of \emph{how many
  electrons pass through} the sheet of paper in a second (number flux), we
wanted to know \emph{how much energy the electrons carry through} the sheet of
paper in a second (energy flux), we'd have to count differently by also asking
every individual electron how much energy it has (that is, how fast it's
going)\footnote{The relationship between kinetic energy---the energy due to
  motion---and speed is bread and butter for classical (but not quantum)
  physicists: $E = \frac{1}{2}m v^2.$} and then summing up the energy
contribution of all the electrons passing through the page within a second.

Last of all, suppose we wanted to know not just the \emph{total} electron energy
flux through the page, but the \emph{contribution} made by electrons
\emph{within some energy range} to the total energy flux.\footnote{What's the
  use, or utility, in sorting the electrons by energy? This information can tell
  us a lot about what is driving, or accelerating the electrons. In a word, it's
  knowing the difference between one bowling ball coming toward you at 100~mph
  versus 100 bowling balls coming toward you at 1~mph. Rather distinct
  scenarios, don't you think?} Here's an idea: You could slap an energy filter
on the page---never mind where we got this little device, just imagine---so that
we only count electrons with energies between, say, 100~eV and
200~eV.\footnote{``An `eV'?''  you say? One eV is one electron-volt---which is
  to say, it's the amount of energy that an electron has after being accelerated
  through a one-volt potential difference.} If we added up all the energy
carried by \emph{those} electrons, repeated the process for electrons having
energies of 200--500~eV, then 500--1200~eV, so on and so forth, we'd have a list
of electron energy ranges, and an energy flux measurement for each energy range,
\emph{capisce}? Like this:
\begin{equation}
  \label{ch1:eqeFlux}
  J_{E} = J_{E,\textrm{100--200 eV}} + J_{E,\textrm{200--500 eV}} + J_{E,\textrm{500--1200 eV}} + \dots .
\end{equation}
$J_E$ is the total energy flux.

Unfortunately even this description of energy flux isn't enough to
convey the meaning of \emph{differential energy flux}, which is what I
originally mentioned,\footnote{The ``deets,'' as I understand young
  people call them now, can be found in a killer introduction to the
  subject located in Appendix~E of the review by
  \citet{Bruno2013}. You could also check out some staples (for me,
  anyway) like Chapter~5 in \citet{Paschmann1998}.} but keep the
description and the following discussion in mind, because it's close
enough for you to be able to appreciate what I'm referring to in later
chapters when I bring up broadband electrons.

% ---------------- FIGURE

\begin{figure}
  \centering
  \noindent\includegraphics[width=0.95\textwidth]{./ch1/figs/Chaston2003_alfelec}
  \caption[\Alfically accelerated (broadband) electrons]{\Alfically
    accelerated (often called ``broadband'') electrons observed by
    the FAST satellite [adapted from Figures~1c and 1d in
    \citealp{Chaston2003a}]. The upper panel shows a so-called
    ``pitch-angle spectrogram,'' and the lower panel an ``energy
    spectrogram'' (see text).}
  \label{ch1:FigAlfElec}
\end{figure}

% ----------------

Speaking of which, back to broadband electrons. When we fly satellites
above the aurora and measure the differential energy flux of auroral
electrons, we often get a response that is typified by what you see in
Figure~\ref{ch1:FigAlfElec}. The plot in the upper panel shows what
snobby space plasma physicists call a ``pitch-angle spectrogram.''
See how the second line of the left-hand axis label says ``Angle
(Deg.)''?  The angle in question---the \emph{pitch angle}---is
measured relative to the direction that the Earth's magnetic field
points so that, for example, 0$^\circ$ is ``field-aligned'' and
90$^\circ$ is perpendicular to the geomagnetic field. See the obvious
horizontal black strip in the pitch-angle spectrogram? That's telling
you these puppies (if I may be permitted the expression) are coming in
hot.

Take a look at the $y$ axis to get a sense of the range of
angles. Right: they're all within about 30$^\circ$ of 0$^\circ$. They
are mostly field-aligned. And the way to make a pitch-angle
spectrogram is to pick a pitch angle (maybe you could start by looking
at 0$^\circ$, straight down the magnetic field line), sum the
``filtered'' electron energy fluxes to get $J_E$\footnote{This should
  really be $\frac{dJ_E}{dE}$, but I'm afraid the return on investment
  (of \emph{your} time, not mine!) is too low to bother explaining it
  here. I'm telling you, see what \citet{Bruno2013} have to
  say. They'll even tell you about those atrocious units you see on
  the right-hand side of the lower panel, ``eV/cm$^2$-s-sr-eV.''} (use
equation \ref{ch1:eqeFlux}), pick a new pitch angle and again
calculate $J_E$, so on and so forth until you've calculated $J_E$ at
each pitch angle. Now, repeat the entire process over and over again
at a regular time interval and \emph{plot those data.} Bam!
Pitch-angle spectrogram.

Let me back up for a second. By combining this business of pitch
angles with our discussion of ``filtered'' measurements of electron
energy flux, the way we did in the last paragraph, you are not far
from what we actually do with particle detectors on rockets and
satellites. The goal of these detectors is to measure \emph{how many
  particles} (electrons, in our case) at \emph{some pitch angle} and
over \emph{some range of energies}. If you had made these types of
calculations for the electrons hanging out at $\sim$2700~km on Feb~2,
1997 somewhere on the nightside of the earth, and had decided to turn
those calculations into spectrograms, you might get something like
Figure~\ref{ch1:FigAlfElec} depending on where exactly you were.

% ---------------- FIGURE

\begin{figure}
  \centering
  \noindent\includegraphics[width=0.95\textwidth]{./ch1/figs/dombeck2013}
  \caption[Inverted V (monoenergetic) electrons]{Monoenergetically
    accelerated (often called ``inverted V) electrons observed by
    the FAST satellite [adapted from Figure~4b in \citealp{Dombeck2013}].}
  \label{ch1:FigDombeck}
\end{figure}

% ----------------

The plot in the lower panel---a typical example of unimaginatively
named \emph{energy spectrograms}---shows that ``tell-tale signature''
of broadband electrons that I mentioned at the beginning of this
section. What is it saying? Look at the second line of the $y$ axis in
this case: ``Energy (eV).'' See how virtually all of the action in
this panel is below 1000~eV, and how the action is more or less
uniformly hot over energies from 4~eV up to wherever the differential
energy flux cuts off? Compare that with what you see in
Figure~\ref{ch1:FigDombeck}, and maybe now you can appreciate why the
former are called ``broadband.'' 

The way to make this plot is, instead of adding up the contribution to
differential energy flux from each \emph{energy range} to a given
\emph{pitch angle} , to add up the contribution to differential energy
flux from each \emph{pitch angle} to a given \emph{energy range}, and
\emph{average}. Therefore, one might say that each type of spectrogram
shown in Figure~\ref{ch1:FigAlfElec} is a different way of looking at
the same coin, and clearly each type of spectrogram conveys different
information about the local electrons.

You might be saying to yourself (and rightly so), ``What does that
have to do with \Alf waves? Nothing about Figure~\ref{ch1:FigAlfElec}
says 'jiggly magnetic field lines' or 'guitar string' to me. Wasn't
that the idea?'' Yes, young padowan. 

In the auroral zone---an altitude of a few hundred km and above at
latitudes\footnote{As before, I mean magnetic latitude and not geographic.}
above $\sim$60$^\circ$---field-aligned electrons barrel toward the atmosphere
and are near the top of the list of causes of aurora. The complication is, as
you can imagine, that there are several ways one could get field-aligned
acceleration.\footnote{\citet{Wygant2002}, \citet{Bostrom2003a},
  \citet{Morooka2004}, \citet{Newell2009}, \citet{Hull2010}, \citet{Mottez2016},
  are only a handful among many, many current or recently retired researchers
  who would love to tell you more, and they aren't even the historical giants
  (like \citet{Knight1973}, \citet{Evans1974}, \citet{Hasegawa1976}, and
  \citet{Lyons1980a}, to name a few) who laid the groundwork!}
\citet{Chaston2002,Chaston2003a,Chaston2007,Chaston2008} in particular will tell
you that \Alf waves are certainly on the short list of culprits, and are in fact
the cause of what we see in Figure~\ref{ch1:FigAlfElec}. You are left to your
own devices to understand why; I am merely asserting that that is what has been
learned.

% \subsection{Why are electrons accelerated along the field line?}



% \section{Geomagnetic storms, interplanetary magnetic fields}

% So if there is some disturbance like a coronal mass ejection banging
% on the nose of the bow shock (Figure~\ref{ch1:FigBowShock}),

% % ---------------- FIGURE

% \begin{figure}
%   \centering
%   \noindent\includegraphics[width=0.8\textwidth]{./ch1/figs/figbowshock}
%   \caption[The Earth's Bow Shock]{The Earth's Bow Shock [Retrieved
%     from the WIND Magnetic Field Investigation (MFI) Team
%     \href{https://wind.nasa.gov/mfi/team_sciencea.html}{website}].}
%   \label{ch1:FigBowShock}
% \end{figure}

% ----------------



% Let's get into it

% \subsection{The solar wind (that old bully!)}

% Solar this 

% \subsection{Things that go ``bump'' on the nightside}

% And nightside that 

\section{``\dots They've gone to plaid!''}

Your ship has been prepared for ludicrous speed.\footnote{Think Dark Helmet from
  Spaceballs (the movie, of course, not the cereal).} You now know what \Alf
waves are, what inertial \Alf waves (or IAWs) are, the icky math bogs and demon-
and derivative-filled canyons that conjure IAWs, and what the signatures of
\Alfic electron acceleration are in pitch-angle and energy spectrograms.

``I mean, yeah! That was an extremely engaging introduction to the topic,
Spencer. You really know your stuff.'' I sincerely appreciate the compliment,
and you'll see in succeeding chapters that it is unsubstantiated: Many, many
questions remain around the role of ionospheric IAWs in dynamic
coupling\footnote{\textit{Dynamic coupling}: Space-physics speak for those
  periods of time when the magnetosphere-ionosphere coupling is too rapid to
  ignore the scary math that we normally try to avoid because it's complicated,
  but on the other hand is capable of describing ``dynamic'' effects. To cite a
  concrete example or two, I mention studies like those by \citet{Mishin2015} or
  \citet{Verkhoglyadova2016}, who investigate definite instances when dynamic
  (or sometimes ``inductive'') effects can't be ignored.} in thermospheric
heating, in the exact processes by which large-scale waves become small-scale,
and the like. In fact, let me give you a rundown on each of the succeeding
chapters so you can appreciate what questions my advisers, mentors, and
colleagues and I have been researching and why. Reading the following summary
will be your first exercise in synthesizing all the new concepts you just picked
up! There will be a handshake waiting for you on the other side.

\subsubsection{Study \#1: IAWs during geomagnetic storms}
% \textbf{Study \#1: IAWs during geomagnetic storms}

People like \citet{Chaston2007} and \citet{Newell2009} have given some
preliminary indicators of the importance of \Alf waves during ``geomagnetically
active'' periods,\footnote{In kid speak, an example of a geomagnetically active
  period is when the sun, in a fit of anger or sadness or joy, barfs a
  larger-than-normal, or faster-than-normal, or larger-and-faster-than-normal
  blob of itself into space, and the earth crashes into the sun barf. (The
  adults in the audience might be more familiar with terminology like
  \emph{corotating interaction region} or \emph{coronal mass ejection} or
  \emph{magnetic cloud}.) Another possibility is the sun barfing on the earth
  while hanging upside-down on the monkey bars (i.e., the interplanetary
  magnetic field is predominantly southward).}, but things like the frequency of
\Alf wave observation, the amount of energy directly carried by \Alf waves into
the ionosphere, the amount of energy that \Alf waves dump into electrons in
order to accelerate them \textit{en masse} toward earth, etc., had not been
systematically studied as a function of, or in response to, geomagnetic
storms. (By the way, I'll use the code word \emph{measures of \Alfic activity}
in place of ``the frequency of \Alf wave observation, the amount of energy
directly carried by \Alf waves into the ionosphere, the amount of \dots''
etc. etc., in succeeding paragraphs.)

% % ---------------- FIGURE

\begin{figure}
  \centering
  \noindent\includegraphics[width=0.8\textwidth]{./ch1/figs/Sec2--stormExample_labeled--w_SSC}
  \caption[Anatomy of a geomagnetic storm \`{a} la \textit{Dst}]{Anatomy of a
    geomagnetic storm as manifest in the \textit{Dst} index [Credit: Lund
    Observatory].}
  \label{ch1:fDstStorm}
\end{figure}

% ----------------

Enter: Chapter 1. My coauthors and I have performed just such a study, chunking
things up by geomagnetic storm phase and relative to so-called \emph{storm
  sudden commencement.} I've provided a more or less canonical picture of a
geomagnetic storm in Figure~\ref{ch1:fDstStorm}, with colored lines to give you
a sense for each of the three storm phases: quiescence (green line), storm main
phase (red line), and storm recovery phase (blue line).\footnote{If you're in
  need of help in order to understand what causes each storm phase, or if you're
  plumb curious, pay a visit to \citet{Gonzalez1994}, \citet{Denton2006}, or
  \citet{Lakhina2008}. I also recommend \citet{Lundstedt2002}; they'll tell you
  how they go about \emph{predicting} geomagnetic storms. (More honestly,
  they'll describe how they predict the \textit{Dst} index---but the
  \textit{Dst} index has been, I dare say, the go-to measure for geomagnetic
  storms for years and years.} As suggested by \citet{Chaston2007}, my coauthors
and I have found that the \emph{measures of \Alfic activity} that I mentioned in
the last paragraph all pick up dramatically near and just following storm sudden
commencement (golden line) and during storm main phase.

\subsubsection{Study \#2: IAW production: ``peak usage'' and ``off hours''}
% \textbf{Study \#2: IAW production: ``peak usage'' and ``off hours''}

As you might glean from the suggestive meaning of this sub-subsection, the
second study my coauthors and I have undertaken uses Study \#1 as context and
asks in effect, ``How much of the grand total of each measure of \Alfic
activity---for example, the total number of \Alf wave--associated electrons or
the total amount of IAW energy dumped into the ionosphere---observed over a
four-year period occurs during storm main phase, storm recovery phase, and
during quiescent periods?'' This question is related to the broader question,
``What is the relative contribution of each storm phase to the gross totals of
each measure of \Alfic activity?''

The (perhaps unsurprising) answer is that while storm main phase and recovery
phase collectively comprise only $\sim$30\%of all time during the four-year
period we studied, they account for $\sim$66\% of the grand total of each
measure of \Alfic activity. (If you do the math, you can convince yourself that,
on average, the global rates of IAW-associated electron precipitation, energy
deposition, and ion outflow must therefore be 4--5 times the corresponding
global rates during quiescent periods.)

\subsubsection{Study \#3: IAWs, with interplanetary magnetic field starring as taskmaster}
% \textbf{Study \#3: IAWs, with interplanetary magnetic field starring as taskmaster}

For our third study, a group of coauthors and I formulated a to-be-revealed
question related to some basic facts about the solar wind and the awesomely
named \emph{interplanetary MF}. ``An \emph{interplanetary} MF? I can barely
fathom what an MF is, let alone an interplanetary one!'' Ha! You kill me, would
you stop that? As every Joe on the street knows, MF stands for---that's right,
``magnetic field.'' So yes, my coauthors and I were in pursuit of understanding
the role of this interplanetary-flavored MF in controlling IAWs. ``A flavored
MF?'' You jokester, I'm not going to egg you on by answering any more silly
questions. This is graduate school! You have to grow up!

So like I was saying, MFs and waves. Scientists of yesteryear (notably
\citet{Parker1958}) figured out that there is a magnetic field frozen into the
solar wind; these days it goes under the name of interplanetary magnetic field,
or IMF---what I was calling interplanetary MF just to razz you. The direction
that the IMF is pointing at any given time dictates, or (perhaps better stated)
leads to variation in a surprising number of magnetospheric and ionospheric (and
even thermospheric) phenomena. You can find lots of details at the beginning of
Chapter 4, where coauthors and I review the literature related to a selection of
these phenomena.

The thing is, among the ocean of studies detailing the role of the solar wind
and IMF in everything from wave generation \citep[][for
example]{Claudepierre2008} and low-latitude boundary layer formation
\citep{Song2003a}\footnote{You no doubt realize that these examples are
  completely arbitrary.}  to which direction water in your toilet
swirls\footnote{I'm lying; to my knowledge no one has actually studied this.}
and your horoscope,\footnote{Another bold-faced lie.} no one has systematically
studied the role of the solar wind and IMF in controlling IAWs in the
magnetosphere-ionosphere transition region!  It thus behooved my coauthors and
me to say something on the subject, and one of the results seems to confirm an
effect observed in simulation that is counterintuitive based on what other
studies have established. In specific, a simulation-based study of IMF control
of \emph{nondispersive, large-scale} \Alf waves (as opposed to
\emph{small-scale, dispersive} \Alf waves).

\subsubsection{Study \#4: Old dogs doing new tricks}
% \textbf{Study \#4: Old dogs doing new tricks}

The backdrop for this study is classic. A former member of the LaBelle group,
Matt Broughton (super high-quality chap), occasionally contacts me to ask if I
can relay some recently published study or other to him---yes, skipping the cost
of subscription\footnote{Does this practice cause me to scruple?  Frankly, no,
  although maybe it should. I am pained by the reality that knowledge has a
  price in today's world; I feel it a matter about which all humans ought to
  reflect and ask internally if they feel any shame (and I think we should). The
  extent to which the reality exists is, to me, a more-or-less direct measure of
  the degree to which our hearts, collectively speaking, are fixated upon the
  banal, the perishable, the mammon from which springs forth so much inequity,
  pride, and unhappiness. I remove from the pulpit.} and all that---and I say,
``Sure, have it right over to ya, buddy.'' Well, I get to reading this
thing---\citet{Bellan2016}, that is---and I says to myself, ``Hmmm \dots'' And
then I think a little more and I says ``Hmmm \dots . \dots I'm pretty sure I can
give this a shot.'' Fast-forward eight months: BAM! Study \#4 (or Chapter 5).

The idea is that in days of yore (i.e., the fourth quarter of last century or so
up until---well, the present, I suppose) if one wanted to make an in situ
estimate of the wave vector of a an electromagnetic wave using a rocket or a
satellite, one's option was basically spaced probe measurements of the local
electric field. It's essentially an exercise in interferometry. So what's new
about \citet{Bellan2016}, then? Get this: no electric field measurements
necessary to estimate the wave vector! All one needs is
high-resolution\footnote{By \emph{high resolution} I mean that the Nyquist
  frequency lies pretty well above twice the frequency of interest.}, reasonably
accurate measurements of the local magnetic field and particle-based
measurements of the local current density. \citet{Bellan2016} walks through the
relevant theory to show how it all works.

So here's one last kicker: Please tell me that if you were on a quiz show and
you were asked how the frequency of IAWs compares to electron- and ion-cyclotron
frequencies, you could confidently respond \dots come on, say it! YES! IAWs are
\emph{low-frequency} beasts.

See what's going on here?  \citeauthor{Bellan2016}'s [2016] methodology is prime
for application to low-frequency waves such as IAWs! From this standpoint Study
\#4, which is application of the \citet{Bellan2016} methodology to IAW
observations, fits in nicely with Studies 1--3 (Chapters 2--4), which are all
IAW-centric. If you're the sort to get jazzed about gorgeous math, you should
definitely pay a visit to Study \#4 (Chapter 5) and \citet{Bellan2016}.

\subsubsection{Study \#5: Imbuing the expression ``from left field'' with new meaning}
% \textbf{Study \#5: Imbuing the expression ``from left field'' with new meaning}

Maybe I'm going overboard in suggesting that Study \#5 (Chapter 6) has its
origins in left field, but that it is the greatest departure from the IAW theme
in the rest of this thesis is objectively undeniable. How have events thus transpired?

It all happened one fateful day in early 2016 (April or May, if memory serves)
when I got an email from Chris Chaston asking if I'd be interested in a little
project involving reexamination of the famous relationship (among space plasma
physikers) between auroral current density and potential drop along a
geomagnetic field line---also know ans the J-V relation---predicted by
\citet{Knight1973}. One could fairly ask, ``If it's famous and accepted, why
bother to undertake reexamination? And what do you mean by 'reexamination'
anyway?''

Hark back momentarily to the derivation of the IAW dispersion relation in
Section~\ref{ch1:ssDerivation} that we walked through. Do you remember how much
time I devoted to making assumptions and explaining (or at least trying to
explain) why those assumptions were valid or perhaps even necessary? We threw
away a \emph{ton} of information in order to keep the derivation simple!
\citeauthor{Knight1973}'s [1973] work is no different: it's Assumption City, and
Chris Chaston's idea for reexamination of the Knight relation was to see
whether we, as a research community, have inadvertently contented ourselves with
only a fraction of the truth.

In specific, the Knight relation assumes that the population of electrons (the
\emph{magnetospheric source population}) that gets accelerated toward earth by
these large-scale\footnote{``large-scale'' meaning roughly the same size as the
  system itself---in this case the coupled magnetosphere--ionosphere system.}
potential drops is fairly represented by a Maxwellian velocity
distribution.\footnote{Look in virtually any introductory textbook on
  thermodynamics or statistical mechanics (or plasma kinetic theory, if you're
  less humble) to learn about the famous \emph{Maxwellian velocity distribution}
  (or, when cast in scalar form, the \emph{Maxwellian speed distribution}).}
This assumption is equivalent to assuming hat the source population is in
thermal equilibrium, but multiple works using a variety of spacecraft over the
course of years\footnote{Check out the introduction in Chapter~6 for a little
  review of these studies.} have found that plasmas in the magnetospheric source
region are often \emph{not} in thermal equilibrium, which could mean that
assuming the magnetospheric source population is well represented by a
Maxwellian distribution throws away altogether too much information to always be
valid.

The stage is set: a longstanding and generally accepted theoretical result, the
Knight relation, is subjected to all the hellfire (in form of observational
evidence) that a graduate student can brandish against it. The stakes are no
less than total glory, either for \citet{Knight1973}, or Hatch et al. [please
let it pass peer review], or for neither. The last possibility is far and away
most probable, since for the main, science only proceeds haltingly and in fits
and spurts.\footnote{I am hosing around with the whole ``total glory''
  thing. Did you see the very first footnote in this chapter? Though I here mock
  that sentiment, to truth, and to truth only, be total glory.}

\subsubsection{The finalest of words}
% \textbf{The finalest of words}

The first four studies (Chapters 2--5) form a coherent thesis, while the last
study (Chapter~6) seems to have been thrown in by the nape of its neck---such is
my personal feeling. It doesn't matter; each study represents work that was
accomplished during my time as a graduate student, and apparently that's good
enough for my adviser Jim. 

So what if I were to distill my thesis into two sentences? People have ``taught
and thought a lot'' (to quote Ronald A. Rasband) about \Alf waves, and even the
more specialized brand known as inertial \Alf waves, but until the body of work
I present in this thesis appeared, there have not been any systematic studies of
the influence of the interplanetary MF,\footnote{There I go again!} the solar wind,
and geomagnetic storms on high-latitude inertial \Alf waves. I believe my
coauthors and I will ultimately have produced from the depths of the sea of
knowledge several basic facts on these points, all while standing on the
unknowably vast shore of deeply (and I believe incorrigibly) couched human
ignorance---once and if the studies in following chapters make it through peer
review, that is. 

In acknowledgement of the many areas where our work may be expanded, I have
buried several treasure maps in Chapter~7. If you take a shovel and are willing
to make a sacrificial offering of five of the best years of your life, I think
those treasure maps could lead you to one or two of the gazillion theses that
are buried along the shore.

As we stand here at the trailhead of the places I have been walking for the past
five years, I want to thank you for having reached this line. You are among the
0.000000025\% of humanity who I expect will achieve this feat.\footnote{I have
  grossly assumed a steady population of 8 billion. If you were to do the math,
  you'd find I don't expect even two members of my committee to have reached
  the end of this chapter.}

% And chapter 1 bib
\bibliographystyle{agufull08}
\bibliography{refs1}
