%------------------------------------------------------------------------%
% Example chapter for the thesis template.                               %
%                                                                        %
%      Author: Gregory Alexander Feiden                                  %
%   Institute: Dartmouth College                                         %
%        Date: 2014 May 17                                               %
%                                                                        %
%     License: Beerware (revision 42)                                    %
%              ----------------------                                    %
%              Gregory Feiden wrote this file. As long as you retain     %
%              this notice you can do whatever you want with this code.  %
%              If we meet some day and you think this code is worth it,  %
%              you can buy me a beer in return.                          %
%                                                                        %
%------------------------------------------------------------------------%

\chapter{Outro}
\label{chp:outro}

%---BEGIN INSPIRATION QUOTE 
\begin{flushright}
\begin{minipage}[]{0.55\linewidth}
    \begin{flushright}
      The life of every man is a diary in which he means to write one story, and writes another; and his humblest hour is when he
      compares the volume as it is with what he vowed to make it. \\{\small \emph{--- J.M. Barrie} }
    \end{flushright}
\end{minipage}
\end{flushright}
\vspace{\baselineskip}
%---END INSPIRATIONAL QUOTE

\section{A word before things pick up}

If the pursuit of knowledge fills the soul with joy, then
communication of it brings the soul to overflowing\footnote{With due
  piety, we pass over the many and regularly occurring instances of
  science turned bloodsport.}. Science that is worth its mettle and of
respectable vintage---the ``hard stuff,'' so to speak---will be
steeped in a laconic grammar befitting pursuit of that brand of truth
first minted by Bacon, and much later, by Popper's inveterate
philosophical craft, turned to diamond: careful, measured, and
repeatable observation, followed by deduction and falsifiable
conclusion, corrected for faulty intuition and attended to with utter
impartiality---or at least a show of it that is on par for winning
proposals, scoring grant money and keeping up the guise (and make sure
to say something about the Roadmap!).

Having drunk reverently, deeply, and frequently from this cup, this
veritable goblet of gladness, I opt to lay the rudiments of the
present work at its outset in the parlance of the commoner---the
salesman, the nurse, the brickmason, the entrepreneur$^*$---that all
who desire may sup the elixir, if only for a chapter.

The drink of the day \emph{Allons-y!}

\section{}

% Keep the old parindent and parskip in \savedparindent and \savedparskip
\newlength{\savedparindent}
\newlength{\savedparskip}
\setlength{\savedparindent}{\parindent}
\setlength{\savedparskip}{\parskip}

% Now set the new'ns
\setlength\parindent{0pt}
\setlength\parskip{1ex plus 2pt minus 1pt}
\newcommand\X{\par\noindent---~}


\X ``What is an \Alf wave?''  

\X ``A natural low-frequency mode of
oscillation of a plasma.''  

\X ``Are there high-frequency modes?''  

\X
``Yes! Plasma mode \& Co.''  

\X ``What makes them high frequency?''


\X ``They deal with oscillations too rapid for an ion to participate,
and such oscillations are totally neglected i ngoing from two-fluid
theory to MHD---specifically when the Hall term (), which includes (),
is discarded.''  

\X ``What's the importance of them?''  

\X ``You mean
in relative terms, or in some abstract, general way?''  

\X ``Both, but
start with relative terms.''  

\X ``All right. Generality. Wait---let's
start with some background and history. You're familiar with soun
waves, right? The idea that the air can vibrate and carry the sound of
music, the sound of a fog horn, the drop of a pin, yadda yadda?''

\X ``Yeah.''


\X ``Cool. Know why the air does that?''


\X ``Uhh \dots''


\X ``Right. Just think about the speed of sound, which I bet you've
heard of. We'll call it $v_s$. Here's the thing: in simple
cases\footnote{We've got to make all kinds of assumptions to make the
  argument work, but they're not unreasonable for the purpose of our
  conversation. Check out your favorite intro text if you want to dive
  into assumptions.} $v_s$ dictates a relationship between the
wavelength $\lambda$ and frequency $f$ of a particular sound wave. Namely,
\begin{equation}
  \label{ch1:eqvs}
  \lambda f = v_s,
\end{equation}
or, just as good,
\begin{equation}
  \label{ch1:eqvsdiff}
  \lambda = \frac{v_s}{f}
\end{equation}
What does it mean? That $\lambda$ and $f$ are \emph{proportional}, and
the so-called \emph{constant of proportionality} is
$v_s$\footnote{$v_s$ is a magical number that nature picks based on
  the air temperature $T$ and number density $n$ (i.e., how many air
  molecules or atoms fit in a box of a size that we agree upon, like
  1~cm~$\times$~1~cm~$\times$~1~cm). We could say that $v_s = v_s (T,
  N)$, which is a compact way of saying that 'the speed of sound
  depends on air temperature and density.'}. For example, when you
pluck the low E string on a guitar\footnote{No time to quibble with
  different tunings, now.}, which is the fattest string, it vibrates
and launches vibrations in the air that vibrate at a frequency of
about 82~Hz; when you hear it, you might say it is 'knocking at the
door' of your brain, at your eardrum, 82 times each second: $f_E =
$~82~Hz. Well, suppose $v_s =$~340~m/s---which is a pretty good answer
by cereal box standards---the wavelength of that low E, $\lambda_E$ is
4~m, or about the length of a 1993 Buick Century (my stallion of
choice). How about the next string up on a guitar (in pitch, that is),
the A string? $f_A = $~110~Hz, and according to
equation~(\ref{ch1:eqvs}), $\lambda_A \simeq$~3.1~m, or two road bikes
placed end-to-end.

\X ``Why am I telling you all this? Because there's an important
concept illustrated here, what some of us laboratory-bound types call
a \emph{dispersion relation}. In the very particular case at hand,
equation~(\ref{ch1:eqvs}) is telling you that if you start ascending
the E major scale on your guitar so that the pitch steadily increases,
nature has informed us that the wavelength of each note played gets
shorter and shorter so that equation~(\ref{ch1:eqvs}) is always true.

\X ``Things don't have to be so simple, of course. In fact, let's turn
up the heat in a generalized\footnote{In physics speak, when
  discussing a particular theory the word ``generalized'' signals to
  all physicists in the area that the ideas are about to become
  far-reaching, or impossible to comprehend, or take a turn for the
  worse into quackery. It's a word that can only be taken on a
  case-by-case basis, so beware!} way. Instead of
equation~(\ref{ch1:eqvs}), I could tell you something like
\begin{equation}
  \label{ch1:eqvsgen}
  \lambda = \lambda (f,T,N),
\end{equation}
which is wonderfully vague. It looks meaningless, doesn't it? But
no--it's telling you that the wavelength $\lambda$ depends on
frequency, temperature, and density, just like
equation~(\ref{ch1:eqvs}); the difference is that I haven't bothered
to specify \emph{how} they are related. Why? For all its vagueness
equation~(\ref{ch1:eqvsgen}) buys us power to dream a little. Suppose
you and I are investigating some exotic fluid---we'll pretend for
kicks that it behaves sort of like a gas, by which I mean that it can
be described by quantities like temperature and density that we know
so well\footnote{If you think you don't know them well, you simply
  haven't realized that you make everyday decisions, like whether to
  go running, on the basis of temperature (``Is it too hot out
  there?'') and density (``Is it really humid?'').}. Let's futher
pretend that this exotic fluid is made up of charged particles, so
that it is also subject to electrodynamics and can do fancy things
under special conditions, like freeze a magnetic field into
itself\footnote{Within our solar system, the ``frozen magnetic field''
  thing happens all over the neighborhood, so to speak. It is usually
  fair to describe the solar wind as an electrodynamic fluid, or
  ``plasma,'' as some of us are wont to call it, with a ``frozen-in''
  magnetic field whose field lines twist like the skirt of a ballerina
  as she twirls, or the water sprayed by the sprinkler head on your
  lawn as it spins. In the case of the solar wind, these spirals
  happen because of the rotation of the sun, the flow of the solar
  wind \emph{away} from the sun, and the whole ``fluid that freezes
  magnetic fields into itself'' idea.} and conduct electrical
currents.

\X ``So what might the dispersion relation, the ``updated'' version of
equation~(\ref{ch1:eqvs}), look like for this bizarre electromagnetic
fluid? I propose that within our newly discovered exotic fluid,
\begin{equation}
  \label{ch1:eqmhd}
  \lambda = \lambda (f, T, n, B).
\end{equation}
On second thought, it's not as exotic as one might ahve hoped. I mean,
all we did was add the magnetic field $B$! Back in 1942 Hannes
Alfv\'{e}n pulled this very number\footnote{All right, I've swept a
  few details under the rug. But not many, considering Alfv\'{e}n's
  original letter to the publication \textsl{Nature} was only six
  paragraphs! (In case you didn't know, as far as publishing research
  goes \textsl{Nature} is considered by many to be the darling of the
  physical sciences, which comes with all the attendant baggage, or at
  least a lot of it, that you can imagine.))}, assuming the existence
of a conducting fluid like we just did, and guess what he found? A
dispersion relation (I've only changed the units):
\begin{equation}
  \label{ch1:eqAlf}
  \lambda f = \frac{B}{\sqrt{\mu_o m n}} \equiv v_A,
\end{equation}
where $v_a$ is the eponymous (and rather famous among plasma
physicists) Alfv\'{e}n speed, $B$ is the magnetic field strength of
the bizarro-fluid, and $m$ and $n$ are the mass and numberdensity of
the ions in this fluid.


\X ``Don't let the details cause to to miss the point. The dispersion
relation (\ref{ch1:eqAlf}) is \emph{identical} in form to
equation~(\ref{ch1:eqvs}): 'To wit,' as D.J. Griffiths might put it, $
\lambda f = $(some speed). And to bring you up to speed, at this point
there is a mountain of textbooks and peer-reviewed literature that
refer to the Alfv'{e}n speed and (surprise!) Alfv\'{e}n waves. In
fact, with the solitary exception of chapter~6, gratuitous references
to Alfv\'{e}n this-and-that are found throughout the chapters
comprising this thesis; I doubt if there are more than five total
pages in chapters 2--6 that fail to mention the name.


\X ``


\X ``

% 
% 
% Finished with the parindent stuff? Reset 'em
\setlength\parindent{\savedparindent}
\setlength\parskip{\savedparskip}

% And chapter 1 bib
\bibliographystyle{agufull08}
\bibliography{refs1}
