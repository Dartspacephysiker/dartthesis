%------------------------------------------------------------------------%
% Example chapter for the thesis template.                               %
%                                                                        %
%      Author: Gregory Alexander Feiden                                  %
%   Institute: Dartmouth College                                         %
%        Date: 2014 May 17                                               %
%                                                                        %
%     License: Beerware (revision 42)                                    %
%              ----------------------                                    %
%              Gregory Feiden wrote this file. As long as you retain     %
%              this notice you can do whatever you want with this code.  %
%              If we meet some day and you think this code is worth it,  %
%              you can buy me a beer in return.                          %
%                                                                        %
%------------------------------------------------------------------------%

  \chapter{Denouement}
  \label{chp:outro}

  % ---BEGIN INSPIRATION QUOTE 
  \begin{flushright}
    \begin{minipage}[]{0.55\linewidth}
      \begin{flushright}
        The life of every man is a diary in which he means to write one story,
        and writes another; and his humblest hour is when he compares the volume
        as it is with what he vowed to make it. \\{\small \emph{--- J.M. Barrie}
        }
      \end{flushright}
    \end{minipage}
  \end{flushright}
  \vspace{\baselineskip}
  % ---END INSPIRATIONAL QUOTE

%%%%%%%%%%%%%%%%%%%%%%%%%%%%
  % \section{}
%%%%%%%%%%%%%%%%%%%%%%%%%%%%

  In the author's opinion, the primary contribution of this thesis has been the
  demonstration of the importance of inertial \Alf waves. More than 60\% of
  these studies were large-scale and of a statistical nature, and all of them
  involved only a single spacecraft. As one consequence, the overwhelming and
  unforeseen message of the studies in Chapters \ref{chp:2}--\ref{chp:4},
  Chapter~\ref{chp:4} in particular, is clear illustration of the need for
  real-time, detailed, and local observations of mesoscale
  magnetosphere-ionosphere coupling. Such observations were the goal of the
  unfunded Alfv'{e}n:magnetosphere--ionosphere connection explorers
  \citep{Berthomier2011}.

  In addition to this largest and unintended message,
  Chapters~\ref{chp:2}--\ref{chp:6} have suggested several extension studies
  that could be undertaken with comparatively little additional effort.

  Of the unanswered questions left over from Chapter~\ref{chp:2}, a natural
  follow-up study that would broaden the results shown in Figure~5 of
  Chapter~\ref{chp:2} is demonstration of the dependence of all types of auroral
  precipitation on storm phase, including relaxation of the restrictions on
  monoenergetic precipitation prescribed by \citet{McIntosh2014}. This would
  involve sorting monoenergetic precipitation based on its description as, e.g.,
  ``highly nonthermal'' ($\kappa <$~1.7), ``nonthermal'' (1.7 $\leq \kappa
  <$~2.5), ``near-equilibrium'' (2.5 $\leq \kappa <$~5), ``quasi-Maxwellian''
  (5~$\leq \kappa <$~10), and ``Maxwellian'' ($\kappa \geq$~10). This study
  would also address the basic question left unanswered by the study in
  Chapter~\ref{chp:6} on whether there exists a connection between geomagnetic
  conditions and production of extremely nonthermal precipitation.

  In Chapter~\ref{chp:4} the manner in which $\phi_{\textrm{IMF}}$ orientation
  statistically controls Alfv\'{e}nic energy deposition leaves open two
  questions that appear to be within the realm of immediate resolvability.

  The first has to do with the mechanism for dayside generation of \Alfic
  power. Observations and simulation exist at virtually every altitude between
  ground level and the magnetopause, but it remains to be conclusively
  demonstrated, or else refuted, that Kelvin-Helmholtz waves are the source for
  the observed motion of what has here been termed the ``\Alfic cusp,'' or the
  site of enhanced \Alfic electromagnetic and electron energy deposition as well
  as enhanced broadband precipitation. Dedicated, high-resolution simulation
  runs involving realistic solar wind inputs during periods of predominantly
  duskward IMF, which corresponds to Northern Hemisphere prenoon \Alfic
  enhancements, and vice versa under dawnward IMF, could potentially determine
  the relationship between net Alfv\'{e}nic power generated on the dayside and
  IMF $B_y$ magnitude. To my knowledge no study has attempted to quantify this
  relationship, though \citet{Zhang2014} have indicated that LFM simulations
  suggest that it is nonlinear.

  The second would constitute resolution of the issue that arose in
  Chapter~\ref{chp:4} dealing with the discrepancy between electrons observed in
  association with IAWs and the MLT-dependent definition of broadband
  precipitation given by \citet{Newell2009}, which requires that at least one
  ESA channel above 300~eV exhibit dJE/dE >
  2.0~$\times$~10$^8$~eV/(cm$^2$~s~sr~eV) within 9.5--14.5~MLT; outside this MLT
  range the threshold energy is 140~eV. This would be accomplished by
  incorporating pitch-angle information into the definition of broadband
  electron precipitation, and would result in a more accurate definition of
  broadband precipitation, or perhaps even definition of additional types of
  precipitation.

  If this study were carried out prior to execution of the storm
  phase--dependent study of the precipitation types defined by
  \citet{Newell2009}that was mentioned above, it would then be possible to
  quantify the degree of discrepancy between electron energy deposition and
  precipitation categorized under the original \citet{Newell2009} definitions
  and under updated definitions of broadband and diffuse precipitation are
  discrepant.

  Finally, the results of application of the \citet{Bellan2016} method for
  calculation of low-frequency wave vectors, presented in Chapter~\ref{chp:5},
  needs to be compared with results of the other, more established methods for
  wave vector calculation such as interferometry \citep{LaBelle1989,Pfaffa} and
  wave-normal analysis \citep{Santolik2003}. Provided that an appropriate
  interval can be identified in the FAST data set the suite of instruments
  aboard FAST will allow for this comparison, and by combining results from each
  method, may allow for calculation of the full wave vector. To my knowledge
  this has never been done with FAST observations, and would not be possible
  without the ability afforded by the \citet{Bellan2016} method to resolve wave
  vector components using magnetic field and particle measurements.
  
  Evidence is given in Chapter~\ref{chp:5} that it is possible to derive the
  perpendicular components of the wave vector associated with inertial \Alf
  waves using FAST measurements. With the addition of the parallel component of
  the wave vector, estimated via E-field interferometry, the manner in which
  inertial \Alf waves propagate relative to large-scale ionospheric convection
  could conceivably be addressed.

  % And chapter 7 bib
  \bibliographystyle{agufull08}
  \bibliography{refs7}
