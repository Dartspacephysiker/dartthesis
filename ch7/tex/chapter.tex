%------------------------------------------------------------------------%
% Example chapter for the thesis template.                               %
%                                                                        %
%      Author: Gregory Alexander Feiden                                  %
%   Institute: Dartmouth College                                         %
%        Date: 2014 May 17                                               %
%                                                                        %
%     License: Beerware (revision 42)                                    %
%              ----------------------                                    %
%              Gregory Feiden wrote this file. As long as you retain     %
%              this notice you can do whatever you want with this code.  %
%              If we meet some day and you think this code is worth it,  %
%              you can buy me a beer in return.                          %
%                                                                        %
%------------------------------------------------------------------------%

  \chapter{Outro}
  \label{chp:outro}

  % ---BEGIN INSPIRATION QUOTE 
  \begin{flushright}
    \begin{minipage}[]{0.55\linewidth}
      \begin{flushright}
        The life of every man is a diary in which he means to write
        one story, and writes another; and his humblest hour is when
        he compares the volume as it is with what he vowed to make
        it. \\{\small \emph{--- J.M. Barrie} }
      \end{flushright}
    \end{minipage}
  \end{flushright}
  \vspace{\baselineskip}
  % ---END INSPIRATIONAL QUOTE

%%%%%%%%%%%%%%%%%%%%%%%%%%%%
  \section{Le denouement}
%%%%%%%%%%%%%%%%%%%%%%%%%%%%

  The contributions of this thesis include, but are not limited to

  Several studies could profitably be undertaken. 

  The questions left open by the study of the \Alfic response in the
  M-I transition region to geomagnetic storms detailed in Chapter 1
  are many. One has to do with a shortcoming of the available data,
  which is that one component of the electric field was as good as
  unmeasured due to malfunction of the boom deployment mechanism. 

  In Chapter~\ref{chp:4} the manner in which $\phi_{\textrm{IMF}}$
  orientation statistically controls Alfv\'{e}nic energy deposition
  left open two questions that appear to be presently within the realm of
  resolvability. 

  The first has to do with the mechanism for dayside generation of
  \Alfic power. Observations and simulation exist at virtually every
  altitude between ground level and the magnetopause, but it remains
  to be conclusively demonstrated, or else refuted, that
  Kelvin-Helmholtz waves are the source for the observed motion of
  what has here been termed the ``\Alfic cusp,'' or the site of
  enhanced \Alfic electromagnetic and electron energy deposition as
  well as enhanced broadband precipitation. Dedicated, high-resolution
  simulation runs involving realistic solar wind inputs during periods
  of predominantly duskward IMF, which corresponds to Northern
  Hemisphere prenoon \Alfic enhancements, and vice versa under
  dawnward IMF, could potentially determine the relationship between
  net Alfv\'{e}nic power generated on the dayside and IMF $B_y$
  magnitude. To my knowledge no study has attempted to quantify this
  relationship, though citet{Zhang2014} have commented that LFM
  simulations indicate that it is nonlinear.

  The second has to do with resolving the degree to which the
  \citet{Newell2009} definition of broadband precipitation,
  particularly that the average electron energy exceed 300 within 9.5--14.5~MLT,
  discards 

  % And chapter 7 bib
  \bibliographystyle{agufull08}
  \bibliography{refs1}
