%------------------------------------------------------------------------%
% Example chapter for the thesis template.                               %
%                                                                        %
%      Author: Gregory Alexander Feiden                                  %
%   Institute: Dartmouth College                                         %
%        Date: 2014 May 17                                               %
%                                                                        %
%     License: Beerware (revision 42)                                    %
%              ----------------------                                    %
%              Gregory Feiden wrote this file. As long as you retain     %
%              this notice you can do whatever you want with this code.  %
%              If we meet some day and you think this code is worth it,  %
%              you can buy me a beer in return.                          %
%                                                                        %
%------------------------------------------------------------------------%

\chapter{Extremely nonthermal monoenergetic precipitation in the auroral
  acceleration region: in situ observations}
\label{chp:6}

% \usepackage{graphicx}
% \graphicspath{ {/home/spencerh/Desktop/Spence_paper_drafts/2017/Kappa_aurora/Figs/} }

% \usepackage[figuresleft]{rotating}

% \usepackage{amsmath}
% \usepackage{amssymb}

% \usepackage{array}
% \newcolumntype{C}[1]{>{\centering\let\newline\\\arraybackslash\hspace{0pt}}m{#1}}

% \begin{document}

%% ------------------------------------------------------------------------ %%
%  Title
% 
% (A title should be specific, informative, and brief. Use
% abbreviations only if they are defined in the abstract. Titles that
% start with general keywords then specific terms are optimized in
% searches)
%
%% ------------------------------------------------------------------------ %%

% Example: \title{This is a test title}

% \title{}

%% ------------------------------------------------------------------------ %%
%
%  AUTHORS AND AFFILIATIONS
%
%% ------------------------------------------------------------------------ %%

% Authors are individuals who have significantly contributed to the
% research and preparation of the article. Group authors are allowed, if
% each author in the group is separately identified in an appendix.)

% List authors by first name or initial followed by last name and
% separated by commas. Use \affil{} to number affiliations, and
% \thanks{} for author notes.  
% Additional author notes should be indicated with \thanks{} (for
% example, for current addresses). 

% Example: \authors{A. B. Author\affil{1}\thanks{Current address, Antartica}, B. C. Author\affil{2,3}, and D. E.
% Author\affil{3,4}\thanks{Also funded by Monsanto.}}

% \authors{Spencer M. Hatch,\affil{1}
% Christopher C. Chaston\affil{2},
% James W. LaBelle\affil{1}}


% \affiliation{1}{Department of Physics and Astronomy, Dartmouth College,
%   Hanover, New Hampshire, USA.}

% \affiliation{2}{Space Sciences Laboratory, University of California,
%   Berkeley, California, USA}

% \affiliation{3}{School of Physics, University of Sydney, Camperdown,
%   New South Wales, Australia}

%% Corresponding Author:
% Corresponding author mailing address and e-mail address:

% (include name and email addresses of the corresponding author.  More
% than one corresponding author is allowed in this LaTeX file and for
% publication; but only one corresponding author is allowed in our
% editorial system.)  

% Example: \correspondingauthor{First and Last Name}{email@address.edu}

% \correspondingauthor{S. M. Hatch}{Spencer.M.Hatch.GR@dartmouth.edu}

%% Keypoints, final entry on title page.

%  List up to three key points (at least one is required)
%  Key Points summarize the main points and conclusions of the article
%  Each must be 100 characters or less with no special characters or punctuation 

% \begin{keypoints}
% \item We present the first reported observations in the auroral
%   acceleration region of extremely nonthermal ($\kappa \simeq$~1.5)
%   electron precipitation.
% \item Observations suggest that a nonthermal source population may
%   significantly modify the current-voltage relationship.
% % \item Intra-arc kappa, temperature, and density variations suggest
% %   distinct precipitating magnetospheric subpopulations
% \end{keypoints}

%% ------------------------------------------------------------------------ %%
%
%  ABSTRACT
%
%
%% ------------------------------------------------------------------------ %%

% \begin{abstract}
%   We present FAST satellite observations of discrete, field-aligned
%   electron precipitation with differential number flux spectra
%   exhibiting remarkably high fluxes at energies well above the energy
%   of the observed flux peak, which indicate a magnetospheric electron
%   source population in a stationary state far from thermal
%   equilibrium. We show that observed distributions are much better
%   described by a kappa distribution with $\kappa \lesssim$~2 than by a
%   Maxwellian distribution. As a second, independent estimate of the
%   observed kappa values we show that the field-aligned current
%   densities and voltages within these auroral arcs appear to be
%   described by the current-voltage relationship predicted under the
%   assumption of a nonthermal magnetospheric source population
%   \citep{Dors1999} with $\kappa~\simeq$~1.5, but markedly deviate from
%   the current-density relation proposed by \citet{Knight1973}, which
%   assumes a source population in thermal equilibrium. These results
%   represent the first observations of $\kappa~\simeq$~1.5 within the
%   auroral acceleration region, and the first direct demonstration of
%   which we are aware that the current-voltage relationship predicted
%   by assuming a magnetospheric source population in thermal
%   equilibrium does not always adequately describe the current
%   densities, potentials, and properties of the precipitating electron
%   distribution that are observed the acceleration region.
% \end{abstract}


%% ------------------------------------------------------------------------ %%
%
%  TEXT
%
%% ------------------------------------------------------------------------ %%

\textit{Note: Pending the approval of coauthors C.C. Chaston and
  J. LaBelle, as well as incorporation of any suggested revisions, the
  study presented in this chapter will be submitted for publication
  to} Geophysical Research Letters.

%%%%%%%%%%%%%%%%%%%%%%%%%%%%
  \section{Introduction}
%%%%%%%%%%%%%%%%%%%%%%%%%%%%

  The development of quasi-steady potential differences along the
  geomagnetic field lines connecting the plasma sheet and
  high-latitude magnetosphere to the ionosphere is a basic consequence
  of ongoing convection of geomagnetic field lines populated with both
  solar wind and terrestrial plasma \citep[e.g., review by][Chapter
  3]{Paschmann2003}. Without assigning a mechanism, \citet{Knight1973}
  formally demonstrated the relationship between a field-aligned,
  monotonic potential profile characterized by a total potential
  difference $\Delta \Phi$ and the the corresponding field-aligned
  current density at the ionosphere $j_{\parallel,i}$ generated by
  precipitation of magnetospheric particles subject to a magnetic
  mirror ratio $R_B = B_{i} / B_{m}$,
  % \begin{linenomath*}
    \begin{equation} \label{ch6:eqKnight} j_{\parallel,i} ( \, \Delta \Phi ;
      T_m, n_m, R_B ) = - e n_m \Big ( \dfrac{k_B T_m}{2 \pi m_e} \Big
      )^{\frac{1}{2}} R_B \Bigg [ 1 - \Big ( 1 - R_B^{-1} \Big )
      \textrm{exp} \Big \{ - \dfrac{e \Delta \Phi}{k_B T_m ( R_B - 1 )}
      \Big \} \Bigg],
    \end{equation}
  % \end{linenomath*}
  where $T_m$ and $n_m$ are respectively the temperature and density
  of precipitating electrons at the magnetospheric source.

  % From Boström [2003]:
  % Observations of auroral electron, and ionospheric ion, acceleration
  % [e.g., Reiffetal., 1988; Lundin and Eliasson, 1991, and references
  % therein], electric fields [Mozer and Hull, 2001] and the AKR source
  % region [e.g., Bahnsen et al., 1989] suggest that the acceleration
  % occurs at altitudes of 1–2 RE which corresponds to z = 8–30 (for
  % flux tubes with the ionospheric end at a magnetic latitude of 70? or
  % more) although a strict limit to the acceleration region may be
  % difficult to establish experimentally

  Equation (\hyperref{ch6:eqKnight}), otherwise known as the J-V or Knight
  relation, elucidates the role of field-aligned potential differences
  within the large-scale current system connecting the magnetosphere
  and the ionosphere that was posited by Kristian Birkeland more than
  a century ago, and represents a nexus in both the historical and
  theoretical developments that have led to present understanding of
  the magnetosphere-ionosphere current system
  \citep[e.g.,][]{Temerin1997,Hultqvist1999,Cowley2000,Bostrom2003a,Paschmann2003,Pierrard2007a,Karlsson2012}.

  The Knight relation assumes that the magnetospheric source
  population is in thermal equilibrium and is thus described by the
  Maxwellian distribution
  % \begin{linenomath*}
  \begin{equation} \label{ch6:eqMax1D} f_{M}( E ; T, n) \, = n \Big (
    \dfrac{m}{2 \pi k_{B} T} \Big )^{\frac{3}{2}} \textrm{exp} \Big (
    - \dfrac{E}{ k_B T } \Big ).
  \end{equation}
  % \end{linenomath*}
  The spectral shape of this distribution is governed by the
  parameters $T$ and $n$, and neglects the suprathermal tails of
  magnetospheric electron and ion distributions that have often been
  observed by in situ spacecraft. The first such reports appeared
  almost three decades ago \citep{Christon1989,Christon1991}, which
  were followed by reports of suprathermal tail observations at low
  altitudes (<~1000~km) and in the high-latitude plasma sheet
  \citep{Wing1998,Kletzing2003}.

  Despite more frequent association at ionospheric altitudes with
  diffuse, or unaccelerated, precipitation than with accelerated
  precipitation \citep[see, e.g., ][]{Newell2009,McIntosh2014}, the
  possibility of a source distribution with a ``high energy tail'' was
  in fact acknowledged by \citet{Knight1973}, and reformulations of
  the J-V relation (\hyperref{ch6:eqKnight}) assuming a variety of
  alternative source distributions have since been developed
  \citep{Pierrard1996,Janhunen1998,Dors1999,Bostrom2003a,Bostrom2004}.

  One such alternative distribution employed with increasing frequency
  is the kappa distribution \citep{Livadiotis2013}
  % \begin{linenomath*}
  \begin{equation} \label{ch6:eqKappa1D} f_{\kappa}(E; T, n, \kappa) =
    n \Bigg ( \dfrac{m}{2 \pi (\kappa - \frac{3}{2}) k_{B} T \, } \Bigg
    )^{\frac{3}{2}} \dfrac{\Gamma \big ( \kappa + 1 \big ) }{\Gamma
      \big ( \kappa - \frac{1}{2} \big ) } \Bigg ( 1 + \dfrac{E}{ \big
      ( \kappa - \frac{3}{2} \big ) k_B T } \Bigg )^{-1 - \kappa},
  \end{equation}
 % \end{linenomath*}
 which originally entered the space physics community as a model for
 high-energy tails of observed solar wind plasmas
 \citep{Vasyliunas1968}. Relaxing the assumption of a magnetospheric
 source population in thermal equilibrium, \citet{Dors1999} showed
 that the J-V relation (\hyperref{ch6:eqKnight}) becomes
  % \begin{linenomath*}
 \begin{align}
   \label{ch6:eqDors}
 \begin{array}{ll}
   j_{\parallel,i} ( \, \Delta \Phi ; T_m, n_m, \kappa, R_B ) &= - e n_m \sqrt{ \dfrac{ k_B T_m (\kappa -
       \frac{3}{2})}{ 2 \pi m_e}} \dfrac{\Gamma (\kappa + 1)}{\Gamma (\kappa - \frac{1}{2})} \dfrac{R_B}{\kappa (\kappa - 1)} \\
   &\times \Bigg[1 - \Big(1 - R_B^{-1} \Big) \Bigg(1+\dfrac{e \Delta \Phi}{ k_B
     T_m (\kappa - \frac{3}{2})(R_B - 1)} \Bigg)^{1-\kappa} \Bigg].
 \end{array}
 \end{align}
 % \end{linenomath*}

 The additional parameter $\kappa \in [ \, 3/2, \infty )$ in the
 distribution (\hyperref{ch6:eqKappa1D}) and J-V relation
 (\hyperref{ch6:eqDors}) parameterizes the degree to which particle motion
 is correlated, and is therefore related to the ``thermodynamic
 distance'' between a given stationary, non-equilibrium state and
 thermal equilibrium. The theoretical minimum (in three dimensions)
 $\kappa_{\textrm{min}} = 3/2$ corresponds to perfectly correlated
 motion of particles while $\kappa \rightarrow \infty$ corresponds to
 uncorrelated motion of particles, or thermal equilibrium
 \citep{Livadiotis2010,Livadiotis2011,Livadiotis2013}. \citet{Livadiotis2010}
 review both theoretical and observational evidence suggesting that
 $\kappa_t \simeq 2.45$ marks a transition between stationary states
 that are far from thermal equilibrium, $\kappa \in [ \, 3/2, \kappa_T
 )$, and ``near-equilibrium'' states for which $\kappa \in [ \,
 \kappa_T, \infty )$. When all other parameters are held fixed,
 $\kappa \simeq 1.63$ yields the minimum-entropy kappa distribution.

 Regarding physical origins, a host of mechanisms leading to
 production of a suprathermal tail exist \citep[e.g., review
 by][]{Pierrard2010}, many of which are instances of the general
 process by which superdiffusivity arises in a geometrically
 constrained, crowded driven physical system \citep{Benichou2013}.
 The correlated particle motion accompanying non-equilibrium
 stationarity also arises in processes exhibiting self-similar, or
 fractal, scales; hence its connection with turbulent and
 scale-invariant processes \citep{West1990,Treumann1999a,Leubner2004}.

 In this chapter we reexamine previously reported FAST observations of
 precipitating monoenergetic electrons above the auroral ionosphere,
 and find instances of magnetospheric source populations in stationary
 states that are far from thermal equilibrium.

%%%%%%%%%%%%%%%%%%%%%%%%%%%%
  \section{Observations and Distribution Modeling}
%%%%%%%%%%%%%%%%%%%%%%%%%%%%

  Using 2-D model distributions based on either a 1-D Maxwellian
  distribution (\hyperref{ch6:eqMax1D}) or a 1-D kappa distribution
  (\hyperref{ch6:eqKappa1D}), we perform fits to the electron
  precipitation within the precipitating loss cone as measured by FAST
  electrostatic analyzers (ESAs) \citep{Carlson2001}. Four adjustable
  parameters are used for the kappa distribution and three parameters
  for the Maxwellian distribution. The parameters common to each
  distribution are $E^*$ (discussed momentarily), $T$, and $n$; the
  fourth parameter $\kappa$ is used only for the kappa
  distribution. For some applications \citep[e.g.,][]{Sutherland2012}
  it may be useful to relate the kinetic temperature of the kappa
  distribution to a thermalized ``core temperature'' $T_c = T (1-3/2
  \kappa)$. The distinction is not made here, however, since the
  kinetic and thermodynamic definitions of temperature are consistent
  for each type of distribution \citep{Livadiotis2010}.

  Several authors \citep{Maggs1981,Bingham1999,Bingham2000,Mutel2007} have
  modeled 2-D distributions with the general form
  % \begin{linenomath*}
    \begin{equation} \label{ch6:eq2DMod} f_{2D}(\, E^*, \theta ; T, n,
      \kappa) \, = f_{1D}(\, E^*; T, n, \kappa) \, g(\, \theta),
    \end{equation}
  % \end{linenomath*}
    where $E^*$ and $\theta$ are respectively the (possibly modified)
    electron energy and pitch angle, $f_{1D}$ is either the 1-D
    Maxwellian distribution (\hyperref{ch6:eqMax1D}) or the 1-D kappa
    distribution (\hyperref{ch6:eqKappa1D}), and $g(\theta )$ is a
    function describing, for example, the effects of the mirror force
    or the maser instability on the source distribution. Each group of
    authors employing the 2-D model (\ref{ch6:eq2DMod}) prescribe
    various forms for $g(\theta)$. An alternative strategy presented
    by \citet{Pritchett1999} involves reduction of observed 2-D
    distributions to one dimension via integration.

  % ---------------- FIGURE 1

  \begin{figure}
    \centering
    \noindent\includegraphics[width=0.8\textwidth]{./ch6/figs/Kappa_summary--1773-classics}
    \caption[Inverted V precipitation and best-fit
    Maxwellian and kappa distribution parameters (Orbit 1773)]{Previously reported
      FAST observations of inverted V precipitation on Feb 1, 1997
      \citep{Elphic1998}, and corresponding 2-D fit
      parameters. Parameters related to best-fit Maxwellian and kappa
      distributions are respectively demarked with red crosses and
      blue x's in panels c--f. (a) $>$300~eV electron pitch-angle
      distribution. (b) Electron energy spectrum within the
      precipitating loss cone ($\theta_{lc} <$~30$^\circ$). (c) $\kappa$
      fit parameter for the best-fit kappa distribution. (d) Reduced
      chi-squared statistic $\chi^2_{\mathrm{red}}$ for each fit
      type. (e) Observed (black diamonds) and best-fit
      temperatures. (f) Observed (black diamonds) and best-fit
      densities. Observed temperatures are calculated from the
      differential flux measured within the precipitating loss cone and
      over the energy range .3--30~keV. Observed densities are
      calculated over the same energy range, and over the much broader
      range of angles -150$^\circ < \theta <$~150$^\circ$, which is
      taken to be the magnetospheric source cone. Uncertainties of
      observed temperatures and densities are calculated using
      analytic expressions for moment uncertainties of an arbitrary
      plasma distribution function [\citealt{Gershman2015}; see
      Appendix~A].}
    \label{ch6:Fig1}
  \end{figure}

  % ----------------

  The focus of this study is the portion of the source population
  within the precipitating loss cone $\theta_{\textrm{lc}}$, which is
  often $\sim$30$^\circ$ near FAST apogee at 4175~km. We therefore use
  $g(\theta) =$~1 in the 2-D model (\ref{ch6:eq2DMod}) and assume the
  electron energy $E$ is offset by a field-aligned potential drop
  $\Delta \Phi$, or $E^* = \big( \sqrt{e \Delta \Phi} - \sqrt{E}
  \big)^2$.

  Our procedure for estimating the parameters of $f_{2D}$ in
  (\ref{ch6:eq2DMod}) from measured electron distributions is the
  following: first, a 1-D differential energy flux spectrum is
  calculated by averaging the differential energy flux at each energy
  over the range of angles within the precipitating loss cone, after
  which a 1-D fit of the resulting average spectrum is
  performed. Parameters from the 1-D fit are then used as the initial
  estimate of the parameters used to perform a direct 2-D fit of the
  observed differential energy flux distribution over a range of
  energies and angles selected to avoid inclusion of possible
  secondaries and other background populations. The range of angles
  includes all within the precipitating loss cone
  $\theta_{\textrm{lc}}$; the range of energies extends from two
  energy bins below that in which the peak energy flux is observed to
  the ESA energy limit at 30~keV. Throughout this study the reported
  fit parameters are those resulting from 2-D fits.

  \section{FAST Orbit 1773}

  During an approximately 60-s interval on Feb 1, 1997 that was
  originally reported by \citet{Elphic1998}, the FAST satellite
  observed inverted V electron precipitation
  (Figures~\hyperref[ch6:Fig1]{6.1a} and \hyperref[ch6:Fig1]{6.1b})
  and intermittent upgoing ion beams (not shown) near midnight MLT
  during a period of low geomagnetic activity ($K_p =$~1$^-$).

  Performing 2-D fits of the electron distributions within the
  precipitating loss cone shows that many of the distributions are
  highly nonthermal, with $\kappa \lesssim$~2.0
  (Figure~\ref{ch6:Fig1}c). Values of the reduced chi-squared
  statistic $\chi^2_{\mathrm{red}} = \frac{\sum_i Y(x) -
    y_i(x)}{\mathrm{dof}}$ for 2-D fits with a kappa distribution are
  at times much smaller than corresponding $\chi^2_{\textrm{red}}$
  values for fits with a Maxwellian distribution
  (Figure~\ref{ch6:Fig1}c), most markedly during the period between
  red bars (09:26:54--09:27:05~UT) in Figure~\ref{ch6:Fig1} when
  $\kappa$ is frequently near the theoretical minimum
  $\kappa_{\mathrm{min}} =$~3/2.

  % ---------------- FIGURE 2

  \begin{figure}
    \centering
    \noindent\includegraphics[width=0.8\textwidth]{./ch6/figs/20170423-nFlux_fit-09_27_01__57-ees--2_avgs-orb_1773}
    \noindent\includegraphics[width=0.8\textwidth,angle=90]{./ch6/figs/orb_1773-Kappa_fit-09_27_01__57}
    \caption[Example of one- and two-dimensional fits of observed
    inverted-V electron distributions (Orbit 1773)]{Electron spectra
      observed at 09:27:00.94~UT. (a) 1-D differential number flux
      spectrum (black crosses) obtained by averaging all differential
      number flux spectra within the magnetospheric source cone, with
      best-fit Maxwellian and kappa distributions overlaid (red
      dash-dotted line and blue dashed line, respectively). The
      uncertainty of each observed differential number flux is
      calculated by conversion of the corresponding particle count
      uncertainty $\sqrt{N}$ to units of differential number flux. (b)
      Best-fit 2-D kappa distribution (solid contours) with the
      observed 2-D differential energy flux spectrum overlaid (contour
      lines). }
    \label{ch6:Fig2}
  \end{figure}

  % ----------------

  As a representative example of electron observations during the
  period between bars in Figure~\ref{ch6:Fig1}, Figure~\ref{ch6:Fig2}
  shows a 1-D slice and the full 2-D distribution observed at
  09:27:00.94~UT.  In Figure~\ref{ch6:Fig1}a the 1-D best-fit kappa
  and Maxwellian distributions (blue dashed line and red dash-dotted
  line, respectively) overlay the observed distribution (black
  crosses). The best-fit kappa distribution is seen to describe both
  the suprathermal tail and the narrow peak of the observed
  distribution, whereas the best-fit Maxwellian distribution fails to
  describe the suprathermal tail but is able to describe the narrow
  peak via a best-fit temperature that is low compared to the observed
  temperature (40~eV versus $\sim$600~eV).

  Typical ranges of electron densities and temperatures in the distant
  plasma sheet are respectively 0.01--0.5~cm$^{-3}$ and 400-900~eV
  \citep{Kletzing2003,Paschmann2003}. Figures~\ref{ch6:Fig1}e--f show
  that within the inverted V precipitation, the observed (black
  diamonds) and best-fit kappa distribution (blue \emph{x} symbols)
  densities and temperatures are within these typical ranges; best-fit
  Maxwellian densities are also within the corresponding typically
  observed range, but the range of best-fit Maxwellian temperatures is
  somewhat low (20--300~eV) compared to the corresponding typical
  range in the plasma sheet. 

  % ---------------- FIGURE 3

  \begin{figure}
    \centering
    \noindent\includegraphics[width=0.91\textwidth]{./ch6/figs/orb_1773_lump1_20170423__kappa_RB_map}
    \noindent\includegraphics[width=0.91\textwidth]{./ch6/figs/orb_1773_lump1_20170423__kappa_RB_map__superbestfit}
    \caption[Orbit 1773: Reduced chi-squared values for observed J-V
    curves with variable $\kappa$ and $R_B$, fixed $T$, and
    magnetospheric density $n_m \big ( n_F, \bar{\phi}, \alpha_F \big
    )$, and corresponding best-fit J-V curves.]{Results of J-V curve
      fits using J-V relations (\ref{ch6:eqKnight}) and
      (\ref{ch6:eqDors}). The top panel shows reduced chi-squared
      values for the observed J-V relationship between 09:26:54 and
      09:27:05~UT in Figure~\ref{ch6:Fig1}, with variable $\kappa$ and
      $R_B$, fixed $T$, and magnetospheric density $n_m \big ( n_F,
      \bar{\phi}, \alpha_F \big )$. The bottom panel shows the
      observed J-V relationship and corresponding best-fit Maxwellian and
      kappa J-V relations.}
    \label{ch6:Fig3}
  \end{figure}

  % ----------------

  To independently infer $\kappa$, or the degree to which the
  magnetospheric source population is in a nonthermal stationary
  state, we use the J-V relations (\ref{ch6:eqKnight}) and
  (\ref{ch6:eqDors}), to fit observed potential drops $\Delta \Phi$
  and electron current densities $j_{\parallel,i}$ that have been
  mapped to 100~km. The necessary elements are a reliable set of
  measurements of $\Delta \Phi$ and $j_{\parallel,i}$ from which to
  form an observed J-V relation, which can then be compared with a
  predicted J-V relation, and the parameters $P_m = \{ n_m, T_m, R_B,
  \kappa \, \}$, which represent the source population in the
  magnetosphere and are to be used as inputs for J-V relations
  (\ref{ch6:eqKnight}) and (\ref{ch6:eqDors}). In lieu of in situ
  magnetosphere observations, we infer $P_m$ on the basis of FAST
  measurements.

  To form pairs of ($j_{\parallel,i}, \Delta \Phi$) observations
  suitable for comparing with theory, $j_\parallel$ is calculated from
  observed velocity distributions over the range of pitch angles
  within the precipitating loss cone, $\vert \theta \vert
  < 30^\circ$, and over all energies exceeding 300~eV.
  $j_{\parallel,i}$ is then obtained by mapping $j_\parallel$ to
  100~km using International Geomagnetic Reference Field 11, and the
  corresponding total potential drop $\Delta \Phi = \bar{E}_e +
  \bar{E}_i$ (white dotted line, Figure~\ref{ch6:Fig1}b) is
  calculated, where $\bar{E}_e$ is the average source-cone electron
  energy and $\bar{E}_i$ is that of the upgoing ion beam, if present,
  and zero otherwise.

  In practice only a small number of observed J-V pairs $(
  j_{\parallel,i} , \Delta \Phi )$ are suitable for our purposes: both
  Maxwellian and kappa J-V relations (\ref{ch6:eqKnight})
  (\ref{ch6:eqDors}) require a single set of magnetospheric source
  parameters $P_m$, which in turn requires identification of periods
  during which observed $P_m$ are sufficiently steady to be
  meaningfully related to the putative source population. Perhaps not
  coincidentally, the period shown in Figure~\ref{ch6:Fig1} during
  which $\kappa \lesssim 2$ (09:26:54--09:27:05~UT) also delineates a
  period during which the observed density, temperature, and $\kappa$
  are relatively steady. Having identified an appropriate set of
  observations, we now describe our methodology for estimating the
  source parameters $P_m$.

  The extreme values of $\kappa$ (Figure~\ref{ch6:Fig1}c) together
  with the steady temperatures (Figures~\ref{ch6:Fig1}e) and densities
  (Figure~\ref{ch6:Fig1}f) during the period of interest imply
  adiabatic field-aligned transport, i.e., $T_m$ and $\kappa$ are
  largely unmodified in undergoing transport from the magnetospheric
  source region to FAST altitudes. The average observed temperature
  $\bar{T}_F =$~627.9~eV during this period is therefore identified as
  the source temperature $T_m$, and is treated as a fixed parameter
  (i.e., $T_m$ is not allowed to vary), but $\kappa$ is initialized
  with $\kappa =$~10 and is allowed to vary as part of the fitting
  procedure.
  
  From the vantage point of the magnetosphere-ionosphere transition
  region, the two remaining parameters $n_m$ and $R_B$ are tied as a
  consequence of the mirror force and the frozen-in flux
  condition. Confident estimation of each parameter solely on the
  basis of the observations in Figure~\ref{ch6:Fig1} presents the most
  serious challenge to this analysis. We assume the source density
  $n_m$ is a function of the total electron density at FAST $n_F$
  (excluding the density within the range of anti-earthward angles
  within which the source population has already been depleted), the
  mirror ratio between FAST and the magnetospheric source $R_{B,F} =
  B_\textrm{FAST} / B_m$ (as opposed to the magnetosphere-ionosphere
  mirror ratio $R_B$), and potential drop $\Delta \Phi$ via the
  relationship \citep{Ratner1976,Barbosa1977}
  \begin{subequations}
    \begin{align} n_m \big ( n_F, \bar{\phi}, \alpha_F \big ) &= \dfrac{n_F}{Q \big ( \bar{\phi}, \alpha_F \big ) }, \\
      Q \big ( \bar{\phi}, \alpha_F \big ) &\equiv 1 + 2 \big ( 1 -
      \alpha_F \big ) \dfrac{ \bar{\phi}^{1/2} \, e^{- \alpha_F
          \bar{\phi}} }{\textrm{erfc} \big ( - \bar{\phi}^{1/2} \big
        )} { \displaystyle \int_{-\sqrt{\bar{\phi}}}^{\infty} } \, dx
      \, e^{\alpha_F x^2} \, \textrm{erfc} \big ( x \big
      ), \label{ch6:eqBarbosa}
    \end{align}
  \end{subequations}
  % \end{linenomath*}
  %   \end{equation}
  % \end{linenomath*}
  where $\bar{\phi} = \Delta \Phi / k_B T $ and $\alpha_F = 1 /
  R_{B,F}$. Equation (\ref{ch6:eqBarbosa}) is a sigmoid function
  \citep[Figure 1b][]{Barbosa1977} that describes how the number
  density of a hot plasma beam moving adiabatically into a region of
  converging magnetic field lines is modified by field-line
  compression and the mirror force. Since $n_F$ is fixed by
  observation, (\ref{ch6:eqBarbosa}) specifies the relationship
  between $R_{B,F}$ and $n_m$. The final issue is estimation of $R_B$,
  or equivalently $R_{B,F}$.

  Figure~\ref{ch6:Fig3}a displays contours of $\chi^2_{\textrm{red}} (
  \kappa, R_B )$, where
  \begin{subequations}
    \begin{align} \chi^2_{\mathrm{red}} = \dfrac{\sum_i j_{\parallel,i} (\Delta \Phi) - j_{\textrm{pred}\parallel,i}(\Delta \Phi; P_m)}{\mathrm{dof}}; \\
      P_m = \Big \{ n_m ( \hat{n}_F, \bar{\phi} , \alpha_F ),
      \hat{T}_m, R_B, \kappa \, \Big \}. \label{paramFit}
    \end{align}
  \end{subequations}
  Parameters with hats in (\ref{paramFit}) are fixed, and
  $j_{\textrm{pred}\parallel,i} $ is the current density predicted
  either by J-V relation (\ref{ch6:eqKnight}) or
  (\ref{ch6:eqDors}). Figure~\ref{ch6:Fig3}a indicates that only two
  scenarios are plausible: one is that the source population is
  located at or beyond $\sim$4.7~$R_E$ and is extremely nonthermal
  ($\kappa \lesssim 1.7$); the other is that acceleration takes place
  at altitudes below 2~$R_E$.

  The cause of the large gap in ($\kappa, R_B$) parameter space that
  separates these scenarios is bifold; both have to do with the
  linearity condition $\Delta \Phi / k_B T \ll R_B $. First, relative
  to the thermal J-V relation (\ref{ch6:eqKnight}), for $\kappa
  \lesssim 2.5$ the kappa J-V relation (\ref{ch6:eqDors}) is greatly
  influenced by saturation effects, which arise as the linearity
  condition is violated. Second, for equal number density and
  temperature, the current density predicted by the kappa J-V relation
  (\ref{ch6:eqDors}) with $\kappa \simeq 1.65$ is more than double the
  current density predicted by the Maxwellian J-V relation
  (\ref{ch6:eqKnight}) at $\bar{\phi} =$~10, for example.

  As a result, a highly nonthermal source population is plausible in
  Figure~\ref{ch6:Fig3}a in two cases: (1) The source population is
  distant ($\gtrsim$~4.7~$R_E$) so that according to equation
  (\ref{ch6:eqBarbosa}) the source density is reduced relative to the
  density observed at FAST. The much higher current predicted by the
  kappa J-V relation for extremely nonthermal $\kappa$ values
  compensates for the reduced number density. (2) The source
  population is nearby ($<$~2.2~$R_E$) so that the much higher number
  density compensates for current saturation.

  The distinction to be borne in mind is that, for the period
  09:26:54--09:27:05~UT, a highly nonthermal source is compatible with
  either a nearby or a distant source region. A source in thermal
  equilibrium is only compatible with a nearby source region.

  Both nonthermal and thermal sources are represented in
  Figure~\ref{ch6:Fig3}b, which shows the observed J-V relation (black
  plus signs with error bars calculated as described in
  Appendix~\ref{app:A}), and best-fit nonthermal and thermal J-V
  relations given correspondingly by equation (\ref{ch6:eqKnight})
  (blued dashed line) and equation (\ref{ch6:eqDors}) (red dash-dotted
  line). The best-fit nonthermal J-V relation, or that for which
  $\chi^2_{\textrm{red}} ( \kappa, R_B )$ is the global minimum in
  Figure~\ref{ch6:Fig3}a, is given by $\kappa =$~1.58 and $R_B =$799,
  or a source at $\sim$9~$R_E$ (though Figure~\ref{ch6:Fig3}a shows
  that sources as close as $\sim$4.8~$R_E$ are similarly plausible for
  $\kappa \simeq$~1.6). The best-fit thermal J-V relation, for which
  $\kappa \rightarrow \infty$, is obtained for $R_B =$~8.6,
  corresponding to a source at $\sim$2~$R_E$.

  To observationally constrain $R_B$ we assume that acceleration of
  the precipitating electron population occurs prior to pitch-angle
  spreading due to the mirror force. The nearly isotropic appearance
  of 2-D distributions during the period of interest in
  Figure~\ref{ch6:Fig1}, of which Figure~\ref{ch6:Fig2}b is an
  example, seems to justify this assumption. On this point
  \citet{Bostrom2003a} has observed, ``If some preacceleration has
  taken place at much higher altitudes the mirror effect will make the
  distribution look more like an isotropically accelerated
  distribution.''

  With the foregoing assumption we define a ``thermal angle'' $\phi_T
  = \textrm{tan}^{-1} ( \frac{T_m}{2 \Delta \Phi} )$ such that prior
  to mirroring and for $\Delta \Phi > T_m$, the majority of the phase
  space density of the precipitating electron beam is contained within
  the pitch angle range $ \theta < \phi_T$.  This angle range implies
  the condition $R_B \geq R_{B,T} \equiv 4 ( \frac{\Delta \Phi}{T} )^2
  + 1$ for production of an isotropic distribution as the beam moves
  into an increasingly stronger magnetic field. For $\Delta \Phi
  \simeq 850$~V and $T =$~600~eV, $\phi_T = 19^\circ$ and $R_{B,T} =
  9.0$. 
  % \simeq 1.3$~kV and $T =$~600~eV, $\phi_T = 13^\circ$ and $R_{B,T} =
  % 19.8$.
  In other words, the isotropic distributions observed throughout the
  period of interest in Figure~\ref{ch6:Fig1} suggest a source region
  of at least $\geq$2~$R_E$.

  Figure~\ref{ch6:Fig3}a shows that a source region above 2~$R_E$ is
  implausible for the values of $\kappa$ seen in
  Figures~\ref{ch6:Fig1} and \ref{ch6:Fig2}, which are generally below
  $\kappa \simeq$~1.7. By these evidences a distant acceleration
  region appears most plausible.

  % A possible objection to this interpretation centers on secondary
  % electron production (cite cite). A more conservative estimate is
  % that after mirroring (as has been assumed for
  % Figure~\ref{ch6:Fig2}b), $\phi_T =$~30$^\circ$ and therefore $R_{B,T}
  % =$~4.9.

  % \dots

  % I conclude that the kappa scenario seems a little more likely.

  % \dots

% but the
%   relative absence of electrons over more than a decade of energies
%   between the thermal population below $\sim$20~eV and the
%   monoenergetic population above several hundred~eV in
%   Figure~\ref{ch6:Fig1}a
  
  \section{FAST Orbit 1843}

  % ---------------- FIGURE 4

  \begin{figure}
    \centering
    \noindent\includegraphics[width=0.9\textwidth]{./ch6/figs/Kappa_summary--1843-classics-1-Ergun_et_al_1998}
    \caption[Inverted V precipitation and best-fit
    Maxwellian and kappa distribution parameters (Orbit 1843)]{Previously reported
      FAST observations of inverted V precipitation on Feb 8, 1997
      \citep{Ergun1998a,Ergun1998}. See caption for Figure~\ref{ch6:Fig1} for a
      description of the layout.}
    \label{ch6:Fig4}
  \end{figure}

  % ----------------

  Figure~\ref{ch6:Fig4} shows electron ESA observations on Feb 7, 1997
  during FAST orbit 1843, in the same layout as
  Figure~\ref{ch6:Fig1}. These observations were originally reported
  by \citet{Ergun1998a,Ergun1998}, and while those authors used
  burst-mode ESA observations, we present survey-mode observations in
  order to improve the uncertainty related to counting statistics in
  the calculated temperatures and densities
  (Appendix~\ref{app:A}). 

  \citet{Ergun1998} used observations of auroral kilometric radiation
  (AKR) to conclude that the source region for the precipitating
  electrons in Figure~\ref{ch6:Fig4} lies ``near the apogee of FAST.''
  We may compare this information about the source region altitude,
  which was derived from the FAST fields instrument \citep{Ergun2001},
  with the results of an analysis of orbit 1843 paralleling the
  analysis of orbit 1773 in the previous section.

  During the period 20:49:48--20:50:11~UT between red bars, the
  observed properties of the electrons within the precipitating loss
  cone are $\kappa \simeq $~1.55--2.5, $T =$~2--3~keV, and $n
  =$~0.7--1.0~cm$^{-3}$. In contrast with the marked period in
  Figure~\ref{ch6:Fig1}c, during which $\kappa < 1.7$ over most of the
  marked period, Figure~\ref{ch6:Fig4}c instead shows $\kappa > 2$
  over much of the period of interest during this orbit.

  % ---------------- FIGURE 5

  \begin{figure}
    \centering
    \noindent\includegraphics[width=0.95\textwidth,angle=90]{./ch6/figs/orb_1843-Kappa_fit-20_50_08__87}
    \caption[Example two-dimensional fit of inverted V precipitation
    (Orbit 1843)]{Observed 2-D differential energy flux spectrum at
      20:50:08.87~UT (contour lines) overlaying the best-fit 2-D kappa
      distribution (solid contours).}
    \label{ch6:Fig5}
  \end{figure}

  % ----------------

  Figure~\ref{ch6:Fig5} shows a typical example of the 2-D electron
  distributions observed during this period. The observed distribution
  (contour lines) overlays the best-fit 2-D kappa distribution (solid
  contours). The legend in Figure~\ref{ch6:Fig5} shows the peak energy
  is $E_b =$~5.25~keV, and Figure~\ref{ch6:Fig4} shows $T
  \simeq$~2.3~keV; the assumption $E_b > T$ underpinning the
  definition of $\phi_T$ is therefore valid. The rapid dropoff in flux
  with increasing pitch angle suggests, however, that the isotropic
  condition is not satisfied, so that $R_B < R_{B,T} = 22$, or
  equivalently that the source region must be located below
  approximately 2.5~$R_E$.

  % ---------------- FIGURE 6

  \begin{figure}
    \centering
    \noindent\includegraphics[width=0.9\textwidth]{./ch6/figs/orb_1843__kappa_RB_map__20170422}
    \noindent\includegraphics[width=0.9\textwidth]{./ch6/figs/orb_1843__map_linear__bestfit__20170422}
    \caption[Orbit 1843: Reduced chi-squared values for observed J-V
    curves and corresponding best-fit J-V curves.]{Results of J-V
      curve fits using equations \ref{ch6:eqKnight} and
      \ref{ch6:eqDors}. The top panel shows reduced chi-squared values
      for the J-V relationship observed between 20:49:48 and
      20:50:11~UT in Figure~\ref{ch6:Fig4}, as in
      Figure~\ref{ch6:Fig3}. The bottom panel shows the observed J-V
      relation and corresponding best-fit Maxwellian and kappa J-V relations.}
    \label{ch6:Fig6}
  \end{figure}

  % ----------------
  \hyperref[ch6:Fig6]{Figure~6.6} displays ($\kappa, R_B$) parameter
  space in the same format as \hyperref[ch6:Fig3]{Figure~6.3a}. The
  same three most likely physical scenarios shown in
  Figure~\ref{ch6:Fig3}a, a distant nonthermal source, a nearby
  nonthermal source, and a thermal source, are also evident in
  Figure~\ref{ch6:Fig6}a. Because $\kappa$ often exceeds 2 in
  \ref{ch6:Fig4}, and because \hyperref[ch6:Fig5]{Figure~6.5} argues
  that $R_B$ cannot exceed $\sim$22, the likelihood of a distant
  nonthermal source is remote; instead, the balance of the evidence in
  this case is in favor of a source population at less than
  2.5~$R_E$. This conclusion about the source altitude is compatible
  with the work of \citet{Ergun1998} discussed at the beginning of
  this section.


%%%%%%%%%%%%%%%%%%%%%%%%%%%%
  \section{Discussion and Conclusions}
%%%%%%%%%%%%%%%%%%%%%%%%%%%%

  We have performed analyses of orbits 1773 and 1843 in parallel
  fashion. In the former case we conclude that the evidence is in
  favor of a highly nonthermal source with $\kappa < 1.7$ at altitudes
  near or exceeding 4.7~$R_E$. In the latter case we conclude that the
  balance of the evidence is in favor of a source that is nonthermal
  (though less so than the former case) with $\kappa \simeq 2$, at an
  altitude not exceeding $\sim$2.5~$R_E$ apparently in connection with
  AKR.

  The results of this study suggest that a nonthermal source
  population may significantly modify the Knight relation. Given the
  restriction on source regions to $>$~4.7~$R_E$, the results in
  \hyperref[ch6:Fig3]{Figure~6.3a} in particular show that the observed J-V
  relation is only compatible with a highly nonthermal source
  population. Further study is clearly necessary to determine both the
  conditions leading to development of highly nonthermal
  distributions, as well as the source altitude.

  Regarding the occurrence frequency of highly nonthermal populations,
  we have conducted a preliminary examination of previously reported
  FAST observations of inverted V precipitation
  \citep{McFadden1998a,Carlson2001,Janhunen2001,Dombeck2013}, which
  suggest geomagnetically quiet periods favor production of
  ``far-equilibrium'' ($\kappa \leq$~2.5) distributions, though recent
  statistical work with DMSP satellites seems to represent
  counterevidence, with precipitation that is best fit with a kappa
  distribution occurring more regularly during active periods
  \citep[e.g., Figure~7 in][]{McIntosh2014}.

  Results from analysis of orbit 1773 could represent the
  low-altitude counterpart of observations reported by
  \citet{Wygant2002} and \citet{Schriver2003}, where the former have
  stated ``in addition to the auroral particle energization processes
  known to occur at altitudes between 0.5 and 2~$R_E$, there are
  important heating and acceleration mechanisms operating at these
  higher altitudes in the plasma sheet.''

  Previous studies of precipitating electrons in or below the
  magnetosphere-ionosphere transition region and involving kappa
  distributions \citep{Olsson1998,Ogasawara2006,Kaeppler2014a} have
  reported $\kappa \geq$~2.5, values which are within the
  ``near-equilibrium'' regime \citep{Livadiotis2010}, meaning the
  corresponding distributions do not deviate strongly from a
  Maxwellian distribution. In this study we have reported
  distributions with $\kappa \simeq$~1.6.

  In situ magnetosphere and plasma sheet observations
  \citet{Christon1989,Christon1991,Kletzing2003} have shown $\kappa
  <$~4 , but these were not discussed.

  Passing mention is owed to some existing reports in the literature
  showing $\kappa < 1.5$ and even $\kappa \simeq$~0, but because the
  three-dimensional theoretical minimum $\kappa_{\mathrm{min}} =$~3/2
  corresponds to a correlation coefficient of 1, or perfectly
  correlated particle motion, how these results were obtained is
  unclear.

  As attention turns increasingly to dynamic M-I coupling, the
  observations that we have presented indicate the possibility that
  the effects of deviation from thermal equilibrium (as parameterized
  by $\kappa$ in the present work) may not be ignorable either in the
  development of first principles--based models of auroral
  precipitation, or in the study and interpretation of magnetospheric
  energization and acceleration processes.

% and recent theoretical work \citep{Khazanov2015,Khazanov2016}
% indicates that magnetosphere-ionosphere coupling, wave-particle
% interactions, and multiple atmospheric reflections are all necessary
% ingredients for producing the observed distributions. 
  

\bibliographystyle{agufull08}
\bibliography{refs6}
