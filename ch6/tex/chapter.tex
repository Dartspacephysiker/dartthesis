%------------------------------------------------------------------------%
% Example chapter for the thesis template.                               %
%                                                                        %
%      Author: Gregory Alexander Feiden                                  %
%   Institute: Dartmouth College                                         %
%        Date: 2014 May 17                                               %
%                                                                        %
%     License: Beerware (revision 42)                                    %
%              ----------------------                                    %
%              Gregory Feiden wrote this file. As long as you retain     %
%              this notice you can do whatever you want with this code.  %
%              If we meet some day and you think this code is worth it,  %
%              you can buy me a beer in return.                          %
%                                                                        %
%------------------------------------------------------------------------%

\chapter{Extremely nonthermal monoenergetic precipitation in the auroral
  acceleration region: in situ observations}
\label{chp:6}

% \usepackage{graphicx}
% \graphicspath{ {/home/spencerh/Desktop/Spence_paper_drafts/2017/Kappa_aurora/Figs/} }

% \usepackage[figuresleft]{rotating}

% \usepackage{amsmath}
% \usepackage{amssymb}

% \usepackage{array}
% \newcolumntype{C}[1]{>{\centering\let\newline\\\arraybackslash\hspace{0pt}}m{#1}}

% \begin{document}

%% ------------------------------------------------------------------------ %%
%  Title
% 
% (A title should be specific, informative, and brief. Use
% abbreviations only if they are defined in the abstract. Titles that
% start with general keywords then specific terms are optimized in
% searches)
%
%% ------------------------------------------------------------------------ %%

% Example: \title{This is a test title}

% \title{}

%% ------------------------------------------------------------------------ %%
%
%  AUTHORS AND AFFILIATIONS
%
%% ------------------------------------------------------------------------ %%

% Authors are individuals who have significantly contributed to the
% research and preparation of the article. Group authors are allowed, if
% each author in the group is separately identified in an appendix.)

% List authors by first name or initial followed by last name and
% separated by commas. Use \affil{} to number affiliations, and
% \thanks{} for author notes.  
% Additional author notes should be indicated with \thanks{} (for
% example, for current addresses). 

% Example: \authors{A. B. Author\affil{1}\thanks{Current address, Antartica}, B. C. Author\affil{2,3}, and D. E.
% Author\affil{3,4}\thanks{Also funded by Monsanto.}}

% \authors{Spencer M. Hatch,\affil{1}
% Christopher C. Chaston\affil{2},
% James W. LaBelle\affil{1}}


% \affiliation{1}{Department of Physics and Astronomy, Dartmouth College,
%   Hanover, New Hampshire, USA.}

% \affiliation{2}{Space Sciences Laboratory, University of California,
%   Berkeley, California, USA}

% \affiliation{3}{School of Physics, University of Sydney, Camperdown,
%   New South Wales, Australia}

%% Corresponding Author:
% Corresponding author mailing address and e-mail address:

% (include name and email addresses of the corresponding author.  More
% than one corresponding author is allowed in this LaTeX file and for
% publication; but only one corresponding author is allowed in our
% editorial system.)  

% Example: \correspondingauthor{First and Last Name}{email@address.edu}

% \correspondingauthor{S. M. Hatch}{Spencer.M.Hatch.GR@dartmouth.edu}

%% Keypoints, final entry on title page.

%  List up to three key points (at least one is required)
%  Key Points summarize the main points and conclusions of the article
%  Each must be 100 characters or less with no special characters or punctuation 

% \begin{keypoints}
% \item We present the first reported observations in the auroral acceleration
%   region of extremely nonthermal ($\kappa \simeq$~1.5) electron precipitation.
% \item Observations suggest that a nonthermal source population may significantly
%   modify the current-voltage relationship.
% % \item Intra-arc kappa, temperature, and density variations suggest
% %   distinct precipitating magnetospheric subpopulations
% \end{keypoints}

%% ------------------------------------------------------------------------ %%
%
%  ABSTRACT
%
%
%% ------------------------------------------------------------------------ %%

\begin{abstract}
  We report direct in-situ measurements through the auroral acceleration region
  that reveal the prevalence of extremely nonthermal monoenergetic electron
  distributions. These auroral primaries are indicative of source populations in
  the plasma sheet well described as kappa distributions with $\kappa
  \lesssim$~2. We show from observations and modeling how this large deviation
  from Maxwellian form modifies the the acceleration potential required to drive
  current closure through the auroral ionosphere. This violation of the
  classically assumed relationship between field-aligned current and auroral
  acceleration potential has fundamental implications for present-day
  understanding of magnetosphere-ionosphere coupling that should be incorporated
  in global magnetospheric models.
\end{abstract}

%% ------------------------------------------------------------------------ %%
%
%  TEXT
%
%% ------------------------------------------------------------------------ %%

% ---BEGIN INSPIRATION QUOTE 
\begin{flushright}
  \begin{minipage}[]{0.6\linewidth}
    \begin{flushright}
      He was playing the big towns. Chicago, Detroit, and then it was Pittsburgh
      one night. \dots [T]hree quarters way through his act, a rope broke, down
      came the backdrop, right on the back of the neck, and he went flat and
      something broke \dots . It hurt way down deep inside \dots . He knew
      now. Man he really knew now. But it was too late and all he wanted was to
      make this crowd laugh, and they were laughing, but now he knew. \dots And
      you should have seen the bookings come in. \dots The Palladium, MCA,
      William Morris. But it was too late. \dots He really knew now. William
      Morris sends regrets. \\{\small \emph{--- Title track on} The Clown
        \emph{by Charles Mingus}}
    \end{flushright}
  \end{minipage}
\end{flushright}
\vspace{\baselineskip}
% ---END INSPIRATIONAL QUOTE

\textit{Note: Pending the approval of coauthors C. C. Chaston and J. LaBelle, as
  well as incorporation of any suggested revisions, the study presented in this
  chapter will be submitted for publication to} Geophysical Research Letters.

%%%%%%%%%%%%%%%%%%%%%%%%%%%%
  \section{Introduction}
%%%%%%%%%%%%%%%%%%%%%%%%%%%%

  The development of quasi-steady potential differences along geomagnetic field
  lines connecting the plasma sheet and high-latitude magnetosphere to the
  ionosphere is a consequence of the requirement for current closure by hot
  tenuous electrons through a strong converging magnetic
  field. \citet{Knight1973} formally demonstrated the relationship between a
  field-aligned, monotonic potential profile characterized by a total potential
  difference $\Delta \Phi$ and the the corresponding field-aligned current
  density at the ionosphere $j_{\parallel,i}$ generated by precipitation of
  magnetospheric particles subject to a magnetic mirror ratio $R_B = B_{i} /
  B_{m}$,
  % \begin{linenomath*}
  \begin{equation} \label{ch6:eqKnight} j_{\parallel,i} ( \, \Delta \Phi ; T_m,
    n_m, R_B ) = - e n_m \Big ( \dfrac{k_B T_m}{2 \pi m_e} \Big )^{\frac{1}{2}}
    R_B \Bigg [ 1 - \Big ( 1 - R_B^{-1} \Big ) \textrm{exp} \Big \{ - \dfrac{e
      \Delta \Phi}{k_B T_m ( R_B - 1 )} \Big \} \Bigg],
  \end{equation}
  % \end{linenomath*}
  where $T_m$ and $n_m$ are respectively the temperature and density of
  precipitating electrons at the magnetospheric source.

  % From Boström [2003]:
  % Observations of auroral electron, and ionospheric ion, acceleration
  % [e.g., Reiffetal., 1988; Lundin and Eliasson, 1991, and references
  % therein], electric fields [Mozer and Hull, 2001] and the AKR source
  % region [e.g., Bahnsen et al., 1989] suggest that the acceleration
  % occurs at altitudes of 1–2 RE which corresponds to z = 8–30 (for
  % flux tubes with the ionospheric end at a magnetic latitude of 70? or
  % more) although a strict limit to the acceleration region may be
  % difficult to establish experimentally

  Equation (\ref{ch6:eqKnight}) is an example of a current voltage (J-V)
  relation (specifically the Knight relation) that elucidates the role of
  field-aligned potential differences within the large-scale current system
  connecting the magnetosphere, and has led in part to present understanding of
  the magnetosphere-ionosphere current system
  \citep[e.g.,][]{Temerin1997,Hultqvist1999,Cowley2000,Bostrom2003a,Paschmann2003,Pierrard2007a,Karlsson2012}.

  The Knight relation assumes that the magnetospheric source population is in
  thermal equilibrium and is thus described by the Maxwellian distribution
  % \begin{linenomath*}
  \begin{equation} \label{ch6:eqMax1D} f_{M}( E ; T, n) \, = n \Big (
    \dfrac{m}{2 \pi k_{B} T} \Big )^{\frac{3}{2}} \textrm{exp} \Big (
    - \dfrac{E}{ k_B T } \Big ).
  \end{equation}
  % \end{linenomath*}
  The spectral shape of this distribution is governed by the parameters $T$ and
  $n$, and neglects the suprathermal tails of magnetospheric electron and ion
  distributions that have often been observed by in situ spacecraft. The first
  such reports appeared almost three decades ago
  \citep{Christon1989,Christon1991}, which were followed by reports of
  suprathermal tail observations at low altitudes (<~1000~km) and in the
  high-latitude plasma sheet \citep{Wing1998,Kletzing2003}.

  Despite more frequent association at ionospheric altitudes with diffuse, or
  unaccelerated, precipitation than with accelerated precipitation \citep[see,
  e.g., ][]{Newell2009,McIntosh2014}, the possibility of a source distribution
  with a ``high energy tail'' was in fact acknowledged by \citet{Knight1973},
  and reformulations of the J-V relation (\ref{ch6:eqKnight}) assuming a variety
  of alternative source distributions have since been developed
  \citep{Pierrard1996,Janhunen1998,Dors1999,Bostrom2003a,Bostrom2004}.

  One such alternative distribution employed with increasing frequency is the
  kappa distribution \citep{Livadiotis2013}
  % \begin{linenomath*}
    \begin{equation} \label{ch6:eqKappa1D} f_{\kappa}(E; T, n, \kappa) = n \Bigg
      ( \dfrac{m}{2 \pi (\kappa - \frac{3}{2}) k_{B} T \, } \Bigg
      )^{\frac{3}{2}} \dfrac{\Gamma \big ( \kappa + 1 \big ) }{\Gamma \big (
        \kappa - \frac{1}{2} \big ) } \Bigg ( 1 + \dfrac{E}{ \big ( \kappa -
        \frac{3}{2} \big ) k_B T } \Bigg )^{-1 - \kappa},
    \end{equation}
  % \end{linenomath*}
  which originally entered the space physics community as a model for
  high-energy tails of observed solar wind plasmas
  \citep{Vasyliunas1968}. Relaxing the assumption of a magnetospheric source
  population in thermal equilibrium, \citet{Dors1999} showed that the J-V
  relation (\ref{ch6:eqKnight}) becomes
  % \begin{linenomath*}
    \begin{align}
      \label{ch6:eqDors}
      \begin{array}{ll}
        j_{\parallel,i} ( \, \Delta \Phi ; T_m, n_m, \kappa, R_B ) &= - e n_m \sqrt{ \dfrac{ k_B T_m (\kappa -
            \frac{3}{2})}{ 2 \pi m_e}} \dfrac{\Gamma (\kappa + 1)}{\Gamma (\kappa - \frac{1}{2})} \dfrac{R_B}{\kappa (\kappa - 1)} \\
        &\times \Bigg[1 - \Big(1 - R_B^{-1} \Big) \Bigg(1+\dfrac{e \Delta \Phi}{ k_B
          T_m (\kappa - \frac{3}{2})(R_B - 1)} \Bigg)^{1-\kappa} \Bigg].
      \end{array}
    \end{align}
  % \end{linenomath*}
  The additional parameter $\kappa \in [ \, 3/2, \infty )$ in the distribution
  (\ref{ch6:eqKappa1D}) and J-V relation (\ref{ch6:eqDors}) parameterizes the
  degree to which particle motion is correlated, and is therefore related to
  the ``thermodynamic distance'' between a given stationary, non-equilibrium
  state and thermal equilibrium. The theoretical minimum (in three dimensions)
  $\kappa_{\textrm{min}} = 3/2$ corresponds to perfectly correlated motion of
  particles while $\kappa \rightarrow \infty$ corresponds to uncorrelated
  motion of particles, or thermal equilibrium
  \citep{Livadiotis2010,Livadiotis2011,Livadiotis2013}. 

  \citet{Livadiotis2010} review both theoretical and observational evidence
  suggesting that $\kappa_t \simeq 2.45$ marks a transition between stationary
  states that are far from thermal equilibrium, $\kappa \in [ \, 3/2, \kappa_T
  )$, and ``near-equilibrium'' states for which $\kappa \in [ \, \kappa_T,
  \infty )$, as well as evidence of a preferred ``fundamental state'' with
  $\kappa \simeq 1.63$ that corresponds to the minimum-entropy kappa
  distribution.

  Regarding physical origins, a host of mechanisms leading to production of a
  suprathermal tail exist \citep[e.g., review by][]{Pierrard2010}, many of which
  are instances of the general process by which superdiffusivity arises in a
  geometrically constrained, crowded driven physical system
  \citep{Benichou2013}.  The correlated particle motion accompanying
  non-equilibrium stationarity also arises in processes exhibiting self-similar,
  or fractal, scales; hence its connection with turbulent and scale-invariant
  processes \citep{West1990,Treumann1999a,Leubner2004}.

  In this chapter we reexamine two previously reported FAST observations of
  precipitating monoenergetic electrons above the auroral ionosphere
  \citep{Elphic1998,Ergun1998a}. In each case we first show via 2-D fits based
  on kappa and Maxwellian distributions that observed electrons are out of
  thermal equilibrium. We then estimate the relationship between current density
  and potential drop, and show that only a subset of all possible source
  altitudes and $\kappa$ values are consistent with (1) the inferred J-V
  relationships and (2) the pitch-angle spreading of the precipitating
  electrons.

%%%%%%%%%%%%%%%%%%%%%%%%%%%%
  \section{Methodology}
%%%%%%%%%%%%%%%%%%%%%%%%%%%%

  \subsection{2-D Model Distribution for Precipitating Electrons} \label{ss2D}

  Using 2-D model distributions based on either a 1-D Maxwellian distribution
  (\ref{ch6:eqMax1D}) or a 1-D kappa distribution (\ref{ch6:eqKappa1D}), we
  perform fits to the electron precipitation within the precipitating loss cone
  as measured by FAST electron electrostatic analyzers (EESAs) \citep{Carlson2001}. Four
  adjustable parameters are used for the kappa distribution and three parameters
  for the Maxwellian distribution. The parameters common to each distribution
  are $E^*$ (discussed momentarily), $T$, and $n$; the fourth parameter $\kappa$
  is used only for the kappa distribution. For some applications
  \citep[e.g.,][]{Sutherland2012} it may be useful to relate the kinetic
  temperature of the kappa distribution to a thermalized ``core temperature''
  $T_c = T (1-3/2 \kappa)$. The distinction is not made in this study, however,
  since the kinetic and thermodynamic definitions of temperature are consistent
  for each type of distribution \citep{Livadiotis2010}.

  Several authors \citep{Maggs1981,Bingham1999,Bingham2000,Mutel2007} have
  modeled 2-D distributions with the general form
  % \begin{linenomath*}
    \begin{equation} \label{ch6:eq2DMod} f_{2D}(\, E^*, \theta ; T, n,
      \kappa) \, = f_{1D}(\, E^*; T, n, \kappa) \, g(\, \theta),
    \end{equation}
  % \end{linenomath*}
  where $E^*$ and $\theta$ are respectively the (possibly modified) electron
  energy and pitch angle, $f_{1D}$ is either the 1-D Maxwellian distribution
  (\ref{ch6:eqMax1D}) or the 1-D kappa distribution (\ref{ch6:eqKappa1D}), and
  $g(\theta )$ is a function that describes, for example, the effects of the
  mirror force or the maser instability on the source distribution. Each group
  of authors employing the 2-D model (\ref{ch6:eq2DMod}) prescribe various forms
  for $g(\theta)$, with the exception of \citet{Pritchett1999}, who reduce
  observed 2-D distributions to one dimension via integration.

  Our focus in this chapter is the monoenergetically accelerated source
  population. We accordingly set $g(\theta) =$~1 in the 2-D model
  (\ref{ch6:eq2DMod}), and perform 2-D fits over a range of energies and angles
  selected to avoid inclusion of possible secondaries and other background
  populations. The range of angles includes all within the earthward loss cone;
  the range of energies extends from two energy bins below that in which the
  peak energy flux is observed to the EESA energy limit at 30~keV. We also
  assume $E^* = \big( \sqrt{E} - \sqrt{E_p} \big)^2$, or that the electron
  energy $E$ is offset by the energy $E_p$ at which the observed electron
  distribution is peaked.

  The detailed procedure for estimating the parameters of $f_{2D}$ in
  (\ref{ch6:eq2DMod}) from measured electron distributions is the following:
  first, a 1-D differential energy flux spectrum is calculated by averaging the
  differential energy flux at each energy over the range of angles within the
  earthward loss cone (sometimes referred to as the source cone), after which
  1-D fits of the resulting average spectrum are performed using the 1-D
  Maxwellian (\ref{ch6:eqMax1D}) and 1-D kappa (\ref{ch6:eqKappa1D})
  distributions. Best-fit parameters from the 1-D Maxwellian and kappa fits are
  then used as initial estimates of the parameters used to perform a direct 2-D
  Maxwellian and kappa fits, respectively, of the observed differential energy
  flux distribution over the previously described range of pitch angles and
  energies. Throughout this study the reported fit parameters are those
  resulting from 2-D fits.

  \subsection{The J-V Relationship: Observation and Model Comparison} \label{ssJV}

  We use the J-V relations (\ref{ch6:eqKnight}) and (\ref{ch6:eqDors}) to fit
  observed potential drops $\Delta \Phi$ and electron current densities
  $j_{\parallel,i}$ that have been mapped to 100~km. These provide an additional
  means of constraining $\kappa$, or the degree to which the magnetospheric
  source population is in a nonthermal stationary state. The necessary elements
  are a reliable set of measurements of $\Delta \Phi$ and $j_{\parallel,i}$ from
  which to form an observed J-V relation, which can then be compared with a
  predicted J-V relation, and the parameters
  $P_m = \{ n_m, T_m, R_B, \kappa \, \}$, which represent the source population
  in the magnetosphere and are to be used as inputs for J-V relations
  (\ref{ch6:eqKnight}) and (\ref{ch6:eqDors}). In lieu of in situ magnetosphere
  observations, we infer $P_m$ on the basis of FAST measurements.

  To form pairs of ($j_{\parallel,i}, \Delta \Phi$) observations suitable for
  comparing with theory, $j_\parallel$ is calculated from observed velocity
  distributions over the range of pitch angles within the precipitating loss
  cone, and over all energies exceeding 200~eV.  $j_{\parallel,i}$ is then
  obtained by mapping $j_\parallel$ to 100~km using International Geomagnetic
  Reference Field~11 (IGRF~11). The corresponding total potential drop
  $\Delta \Phi = \bar{E}_e + \bar{E}_i$ is then calculated, where $\bar{E}_e$ is
  the average source-cone electron energy and $\bar{E}_i$ is that of the upgoing
  ion beam, if present, and zero otherwise. (The average energy is defined as
  the ratio of energy flux and number flux moments, calculated within the
  earthward portion of the loss cone for electrons, and the anti-earthward
  portion of the loss cone for ions.)

  In practice only a small number of observed J-V pairs
  $( j_{\parallel,i} , \Delta \Phi )$ are suitable for our purposes: both
  Maxwellian and kappa J-V relations (\ref{ch6:eqKnight}) and (\ref{ch6:eqDors})
  require a single set of magnetospheric source parameters $P_m$, which in turn
  requires identification of periods during which observed $P_m$ are
  sufficiently steady to be meaningfully related to the putative source
  population. We therefore manually identify periods during which the calculated
  temperature and density are steady. As we will show, for the two orbits
  considered these periods of steady temperature and density also correspond to
  steady $\kappa$ values.

  Once an appropriate set of J-V pairs $( j_{\parallel,i} , \Delta \Phi )$ has
  been identified we estimate the source parameters $P_m$. The steadiness of
  temperature, density, and $\kappa$ implies adiabatic field-aligned transport,
  i.e., $T_m$ and $\kappa$ are largely unmodified in undergoing transport from
  the magnetospheric source region to FAST altitudes. We therefore identify the
  average observed temperature at FAST during each period $\bar{T}_F$, as the
  source temperature $T_m$ and treat $T_m$ as a fixed parameter (i.e., $T_m$ is
  not allowed to vary), but $\kappa$ is initialized with $\kappa =$~10 and is
  allowed to vary as part of the fitting procedure.

  From the vantage point of the magnetosphere-ionosphere transition region, the
  two remaining source parameters $n_m$ and $R_B$ are tied as a consequence of
  the mirror force and the frozen-in flux condition. Confident estimation of
  each parameter solely on the basis of FAST observations presents the most
  serious challenge to this analysis. We assume the source density $n_m$ is a
  function of the total electron density at FAST $n_F$ (excluding the
  contribution from the range of anti-earthward angles within which the source
  population has already been depleted), the mirror ratio between FAST and the
  magnetospheric source $R_{B,F} = B_\textrm{FAST} / B_m$ (as opposed to the
  magnetosphere-ionosphere mirror ratio $R_B$), and potential drop $\Delta \Phi$
  via the relationship \citep{Barbosa1977}
  \begin{subequations}
    \begin{align} n_m \big ( n_F, \bar{\phi}, \alpha_F \big ) &= \dfrac{n_F}{Q \big ( \bar{\phi}, \alpha_F \big ) }, \\
      Q \big ( \bar{\phi}, \alpha_F \big ) &\equiv 1 + 2 \big ( 1 - \alpha_F
                                             \big ) \dfrac{ \bar{\phi}^{1/2} \,
                                             e^{- \alpha_F \bar{\phi}}
                                             }{\textrm{erfc} \big ( -
                                             \bar{\phi}^{1/2} \big )} {
                                             \displaystyle
                                             \int_{-\sqrt{\bar{\phi}}}^{\infty}
                                             } \, dx \, e^{\alpha_F x^2} \,
                                             \textrm{erfc} \big ( x \big
                                             ), \label{ch6:eqBarbosa}
    \end{align}
  \end{subequations}
  % \end{linenomath*}
  % \end{equation}
  % \end{linenomath*}
  where $\bar{\phi} = \Delta \Phi / k_B T $ and $\alpha_F = 1 / R_{B,F}$.
  $Q (\bar{\phi}, \alpha_F)$ is a sigmoid function \citep[Figure
  1b][]{Barbosa1977} that describes how the number density of a hot plasma beam
  moving adiabatically into a region of converging magnetic field lines is
  modified by field-line compression and the mirror force. Since $n_F$ is fixed
  by observation, (\ref{ch6:eqBarbosa}) specifies the relationship between
  $R_{B,F}$ and $n_m$. The final issue is estimation of $R_B$, or equivalently
  $R_{B,F}$.

  \subsection{Source altitude estimation via pitch-angle spreading} \label{ssSourceAlt}

  We seek to observationally constrain $R_B$ by assuming that acceleration of
  the precipitating electron population occurs prior to pitch-angle spreading
  due to the mirror force. The nearly isotropic, shell-like appearance of 2-D
  distributions that we present in the following sections seems to justify this
  assumption. On this point \citet{Bostrom2003a} has observed, ``If some
  preacceleration has taken place at much higher altitudes the mirror effect
  will make the distribution look more like an isotropically accelerated
  distribution.''

  We define the angle $\phi_T = \textrm{tan}^{-1} ( \frac{T_m}{2 E_p} )$ such
  that if $E_p > T_m$, the majority of the phase space density of the
  precipitating electron beam is contained within the pitch angle range
  $ \theta < \phi_T$ prior to mirroring. As the beam moves toward lower
  altitudes, the mirror ratio for which all beam particles with pitch angles
  $\theta \geq \phi_T$ have already mirrored is
  \begin{equation} \label{ch6:RBTherm} R_{B,T} = \dfrac{1}{\mathrm{sin}^2 \big
      (\phi_T \big ) } = 4 \Big ( \dfrac{E_p}{T} \Big )^2 + 1.
  \end{equation}
  Provided the electron beam does not mix with secondary populations, the beam
  distribution appears increasingly shell-like (or ``isotropically
  accelerated'') as electrons originally at pitch angles $\vert \theta \vert <
  \phi_T$ spread over an increasing range of angles by virtue of the mirror
  force. Upon reaching low altitudes for which $R_B \gtrsim R_{B,T}$, the
  electron distribution appears almost entirely shell-like except within the
  anti-earthward portion of the losscone.

  % ---------------- FIGURE 1

  % \begin{figure}
  %   \centering
  %   \noindent\includegraphics[width=0.8\textwidth]{./ch6/figs/Kappa_summary--1773-classics}
  %   \caption[Inverted V precipitation and best-fit Maxwellian and kappa
  %   distribution parameters (Orbit 1773)]{Spacecraft observations of inverted V
  %     precipitation on Feb 1, 1997, and corresponding 2-D fit
  %     parameters. Parameters related to best-fit Maxwellian and kappa
  %     distributions are respectively demarked with red crosses and blue x's in
  %     panels c--f. (a) $>$300~eV electron pitch-angle distribution. (b) Electron
  %     energy spectrum within the precipitating loss cone. (c) $\kappa$ fit
  %     parameter for the best-fit kappa distribution. (d) Reduced chi-squared
  %     statistic $\chi^2_{\mathrm{red}}$ for each fit type. (e) Observed (black
  %     diamonds) and best-fit temperatures. (f) Observed (black diamonds) and
  %     best-fit densities. Observed temperatures are calculated from the
  %     differential flux measured within the precipitating loss cone and over the
  %     energy range .3--30~keV. Observed densities are calculated over the same
  %     energy range, and over all angles except those within the anti-earthward
  %     loss cone. Uncertainties of observed temperatures and densities are
  %     calculated using analytic expressions for moment uncertainties of an
  %     arbitrary plasma distribution function [\citealt{Gershman2015}; see
  %     Appendix~A]. This time interval was first studied by \citet{Elphic1998}}
  %   \label{ch6:Fig1}
  % \end{figure}
  \begin{figure}
    \centering
    % \noindent\includegraphics[width=0.98\textwidth]{./ch6/figs/Kappa_summary--1773-classics}
    \noindent\includegraphics[width=0.98\textwidth]{./ch6/figs/Kappa_summary--1773-classics__v2__itvl3integ}
    \caption[Inverted V precipitation and best-fit Maxwellian and kappa
    distribution parameters (Orbit 1773)]{(See caption on next page.)}
    \label{ch6:Fig1}
  \end{figure}
  \begin{figure}
    \centering \contcaption{EESA observations of inverted V precipitation on Feb
      1, 1997, and corresponding 2-D fit parameters. Parameters related to
      best-fit Maxwellian and kappa distributions are respectively demarked with
      red crosses and blue x's in panels c--f. (a) $>$300~eV electron
      pitch-angle distribution. (b) Average electron energy spectrum within the
      earthward loss cone. (c) $\kappa$ fit parameter for the best-fit kappa
      distribution. (d) Reduced chi-squared statistic $\chi^2_{\mathrm{red}}$
      for each fit type. (e) Observed (black diamonds) and best-fit
      temperatures. (f) Observed (black diamonds) and best-fit
      densities. Observed temperatures are calculated from the differential flux
      measured within the earthward loss cone and over the energy range
      .2--30~keV. Observed densities are calculated over the same energy range,
      and over all angles except those within the anti-earthward loss
      cone. Uncertainties of observed temperatures and densities are calculated
      using analytic expressions for moment uncertainties related to counting
      statistics for an arbitrary plasma distribution function
      [\citealt{Gershman2015}; see Appendix~\ref{app:A}]. EESA observations have been
      integrated over three times the survey-mode EESA sampling frequency to
      improve the related counting statistics. This time interval was first
      reported by \citet{Elphic1998}.}
  \end{figure}

  % ----------------

  \section{FAST Orbit 1773}

  During an approximately 60-s interval on Feb 1, 1997 that has previously been
  reported \citep{Elphic1998,Chaston2002b}, the FAST satellite observed inverted
  V electron precipitation (Figures~\ref{ch6:Fig1}a and \ref{ch6:Fig1}b) and
  intermittent upgoing ion beams (not shown) near midnight MLT during a period
  of low geomagnetic activity ($K_p =$~1$^-$).

  % Ultimately wants to be replaced with F-test and p value (could place p value on RH colorbar)

  Performing 2-D fits of the electron distributions within the precipitating
  loss cone shows that many of the distributions are highly nonthermal, with
  $\kappa \lesssim$~2.0 (Figure~\ref{ch6:Fig1}c). Values of the reduced
  chi-squared statistic
  $\chi^2_{\mathrm{red}} = \frac{\sum_i Y(x) - y_i(x)}{\mathrm{dof}}$ for 2-D
  fits with a kappa distribution are at times much smaller than corresponding
  $\chi^2_{\textrm{red}}$ values for fits with a Maxwellian distribution
  (Figure~\ref{ch6:Fig1}c), most markedly during the period between red bars
  (09:26:54--09:27:05~UT) in Figure~\ref{ch6:Fig1} when $\kappa$ is frequently
  near the theoretical minimum $\kappa_{\mathrm{min}} =$~3/2. Parameter
  estimates before 09:26:54~UT are dubious, since \citet{Chaston2002b} have
  reported that the particle observations before this period are immersed in
  strong coherent ion cyclotron wave emission at frequencies faster than the
  EESA sampling frequency and are therefore subject to aliasing.

  % ---------------- FIGURE 2

  % \begin{figure}
  %   \centering
  %   \noindent\includegraphics[width=0.8\textwidth]{./ch6/figs/20170423-nFlux_fit-09_27_01__57-ees--2_avgs-orb_1773}
  %   \noindent\includegraphics[width=0.8\textwidth,angle=90]{./ch6/figs/orb_1773-Kappa_fit-09_27_01__57}
  %   \caption[Example of one- and two-dimensional fits of observed inverted-V
  %   electron distributions (Orbit 1773)]{Electron spectra observed at
  %     09:27:00.94~UT. (a) 1-D differential number flux spectrum (black crosses)
  %     obtained by averaging all differential number flux spectra within the
  %     magnetospheric source cone, with best-fit Maxwellian and kappa
  %     distributions overlaid (red dash-dotted line and blue dashed line,
  %     respectively). The uncertainty of each observed differential number flux
  %     is calculated by conversion of the corresponding particle count
  %     uncertainty $\sqrt{N}$ to units of differential number flux. (b) Best-fit
  %     2-D kappa distribution (solid contours) with the observed 2-D differential
  %     energy flux spectrum overlaid (contour lines). }
  %   \label{ch6:Fig2}
  % \end{figure}
  \begin{figure}
    \centering
    \noindent\includegraphics[width=0.8\textwidth]{./ch6/figs/20170423-nFlux_fit-09_27_01__57-ees--2_avgs-orb_1773}
    \noindent\includegraphics[width=0.8\textwidth,angle=90]{./ch6/figs/orb_1773-Kappa_fit-09_27_01__57}
    \caption[Example of one- and two-dimensional fits of observed inverted-V
    electron distributions (Orbit 1773)]{(See caption on next page.)}
    \label{ch6:Fig2}
  \end{figure}

  \begin{figure}
    \centering \contcaption{Electron spectra observed at 09:27:00.94~UT. (a) 1-D
      differential number flux spectrum (black crosses) obtained by averaging
      all differential number flux spectra within the magnetospheric source
      cone, with best-fit Maxwellian and kappa distributions overlaid (red
      dash-dotted line and blue dashed line, respectively). The uncertainty of
      each observed differential number flux is calculated by conversion of the
      corresponding particle count uncertainty $\sqrt{N}$ to units of
      differential number flux. (b) Best-fit 2-D kappa distribution (solid
      contours) with the observed 2-D differential energy flux spectrum overlaid
      (contour lines). For each pitch angle black asterisks indicate the peak
      energy $E_p = 849.69$~eV and red plus signs indicate the lower edge of the
      minimum energy bins used to perform the 2-D fit, as described in
      \ref{ss2D}.}
  \end{figure}

  % ----------------

  As a representative example of electron observations during the period between
  bars in Figure~\ref{ch6:Fig1}, Figure~\ref{ch6:Fig2} shows a 1-D slice and the
  full 2-D distribution observed at 09:27:00.94~UT.  In Figure~\ref{ch6:Fig1}a
  the 1-D best-fit kappa and Maxwellian distributions (blue dashed line and red
  dash-dotted line, respectively) overlay the observed distribution (black
  crosses). The best-fit kappa distribution is seen to describe both the
  suprathermal tail and the narrow peak of the observed distribution, whereas
  the best-fit Maxwellian distribution fails to describe the suprathermal tail
  but is able to describe the narrow peak via a best-fit temperature that is low
  compared to the observed temperature (40~eV versus $\sim$600~eV).

  Typical ranges of electron densities and temperatures in the distant plasma
  sheet are respectively 0.01--0.5~cm$^{-3}$ and 400-900~eV
  \citep{Kletzing2003,Paschmann2003}. Figures~\ref{ch6:Fig1}e--f show that
  within the inverted V precipitation, the observed (black diamonds) and
  best-fit kappa distribution (blue \emph{x} symbols) densities and temperatures
  are within these typical ranges; best-fit Maxwellian densities are also within
  the corresponding typically observed range, but the range of best-fit
  Maxwellian temperatures is somewhat low (20--300~eV) compared to the
  corresponding typical range in the plasma sheet.

  % ---------------- FIGURE 3

  \begin{figure}
    \centering
    % \noindent\includegraphics[width=0.91\textwidth]{./ch6/figs/orb_1773_lump1_20170423__kappa_RB_map}
    % \noindent\includegraphics[width=0.91\textwidth]{./ch6/figs/orb_1773_lump1_20170423__kappa_RB_map__superbestfit}
    \noindent\includegraphics[width=0.91\textwidth]{./ch6/figs/orb_1773_lump1_20170605__kappa_RB_map__v2}
    \noindent\includegraphics[width=0.91\textwidth]{./ch6/figs/orb_1773_lump1_20170605__kappa_RB_map__superbestfit__v2}
    \caption[Orbit 1773: Reduced chi-squared values for observed J-V curves with
    $T$ held fixed and magnetospheric density $n_m \big ( n_F, \bar{\phi},
    \alpha_F \big )$ $\kappa$ and $R_B$ allowed to vary, and corresponding
    best-fit J-V curves.]{Results of J-V curve fits using J-V relations
      (\ref{ch6:eqKnight}) and (\ref{ch6:eqDors}). The top panel shows
      $\chi^2_{\mathrm{red}} ( \kappa, R_B)$ for the
      % reduced chi-squared values for the
      observed J-V relationship between 09:26:54 and 09:27:05~UT in
      Figure~\ref{ch6:Fig1}, with $T$ held fixed and magnetospheric density $n_m
      \big ( n_F, \bar{\phi}, \alpha_F \big )$, $\kappa$ and mirror ratio $R_B$
      allowed to vary. The bottom panel shows the observed J-V relationship and
      corresponding best-fit Maxwellian and kappa J-V relations.}
    \label{ch6:Fig3}
  \end{figure}

  % ----------------

  Perhaps not coincidentally, the period shown in Figure~\ref{ch6:Fig1} during
  which $\kappa \lesssim 2$ (09:26:54--09:27:05~UT) also delineates a period
  during which the observed density, temperature, and $\kappa$ are relatively
  steady. Following the procedure outlined in section~\ref{ssJV} we assume
  $T_m = \bar{T}_F =$~621.7~eV and form a set of of observed J-V pairs
  $( j_{\parallel,i} , \Delta \Phi )$ from the current densities, which are
  mapped to 100~km, and potential drops (white line, \ref{ch6:Fig1}b) during
  this period.

  The top panel of Figure~\ref{ch6:Fig3} displays contours of
  $\chi^2_{\textrm{red}} ( \kappa, R_B )$, where
  \begin{subequations}
    \begin{align} \chi^2_{\mathrm{red}} = \dfrac{\sum_i j_{\parallel,i} (\Delta \Phi) - j_{\textrm{pred}\parallel,i}(\Delta \Phi; P_m)}{\mathrm{dof}}; \\
      P_m = \Big \{ n_m ( \hat{n}_F, \bar{\phi} , \alpha_F ), \hat{T}_m, R_B,
      \kappa \, \Big \}. \label{paramFit}
    \end{align}
  \end{subequations}
  Parameters with hats in (\ref{paramFit}) are fixed, and
  $j_{\textrm{pred}\parallel,i} $ is the current density predicted either by J-V
  relation (\ref{ch6:eqKnight}) or (\ref{ch6:eqDors}). 

  The top panel of Figure~\ref{ch6:Fig3} indicates that only two scenarios are
  plausible: one involves a source population that is located at or beyond
  $\sim$4.7~$R_E$ and is extremely nonthermal ($\kappa \lesssim 1.7$), and the
  other involves acceleration at altitudes near FAST and below $\sim$2.2~$R_E$
  with either a thermal or nonthermal source. 

  The cause of the large gap in ($\kappa, R_B$) parameter space that separates
  these scenarios is bifold; each has to do with the linearity condition
  $\Delta \Phi / k_B T \ll R_B $. First, relative to the thermal J-V relation
  (\ref{ch6:eqKnight}), for $\kappa \lesssim 2.5$ the kappa J-V relation
  (\ref{ch6:eqDors}) is greatly influenced by saturation effects, which arise as
  the linearity condition is violated. Second, for equal number density and
  temperature, the current density predicted by the kappa J-V relation
  (\ref{ch6:eqDors}) with $\kappa \simeq 1.65$ is more than double the current
  density predicted by the Maxwellian J-V relation (\ref{ch6:eqKnight}) at
  $\bar{\phi} =$~10, for example.

  The best-fit nonthermal J-V relation (defined as that for which
  $\chi^2_{\textrm{red}} ( \kappa, R_B )$ is the global minimum) is marked with
  a gray diamond and is given by $\kappa =$~1.58 and $R_B =$690, corresponding
  to a source at $\sim$8.5~$R_E$. However, it is apparent that a source region
  down to altitudes of $\sim$4.7~$R_E$ is similarly plausible for
  $\kappa \simeq$~1.6. The best-fit thermal J-V relation, for which
  $\kappa \rightarrow \infty$, is shown as a gray triangle above the plot area
  and is given by $R_B =$~8.55, corresponding to a source near 2~$R_E$.

  The lower panel in Figure~\ref{ch6:Fig3} shows the observed J-V relation
  (black plus signs, with error bars calculated as described in
  Appendix~\ref{app:A}), and best-fit nonthermal and thermal J-V relations given
  correspondingly by equation (\ref{ch6:eqKnight}) (blued dashed line) and
  equation (\ref{ch6:eqDors}) (red dash-dotted line).

  % SOME LINES ABOUT WHY REDUCED CHI-SQ IS BADDDDD 
  % The J-V relations (\ref{ch6:eqKnight}) and (\ref{ch6:eqDors})are nonlinear
  % models for the observed J-V relationship, and $\chi^2_{\mathrm{red}} (\kappa,
  % R_B)$ thus cannot be used to quantitatively estimate the goodness of fit for
  % each model \citep{Spiess2010}. Qualitatively, however, the
  % $\chi^2_{\mathrm{red}} (\kappa, R_B)$ values shown in the top panel of
  % Figure~\ref{ch6:Fig3} indicate that only two scenarios are plausible: one
  % involves a source population that is located at or beyond $\sim$4.7~$R_E$ and
  % is extremely nonthermal ($\kappa \lesssim 1.7$); the other involves
  % acceleration at altitudes near FAST and below $\sim$2.2~$R_E$ with either a
  % thermal or nonthermal source. The cause of the large gap in ($\kappa, R_B$)
  % parameter space that separates these scenarios is bifold; each has to do with
  % the linearity condition $\Delta \Phi / k_B T \ll R_B $. First, relative to the
  % thermal J-V relation (\ref{ch6:eqKnight}), for $\kappa \lesssim 2.5$ the kappa
  % J-V relation (\ref{ch6:eqDors}) is greatly influenced by saturation effects,
  % which arise as the linearity condition is violated. Second, for equal number
  % density and temperature, the current density predicted by the kappa J-V
  % relation (\ref{ch6:eqDors}) with $\kappa \simeq 1.65$ is more than double the
  % current density predicted by the Maxwellian J-V relation (\ref{ch6:eqKnight})
  % at $\bar{\phi} =$~10, for example.

  As a result of these effects, a highly nonthermal source population is
  plausible in the top panel of Figure~\ref{ch6:Fig3} in two cases: (1) The
  source population is distant ($\gtrsim$~4.7~$R_E$) so that according to
  equation (\ref{ch6:eqBarbosa}) the source density is reduced relative to the
  density observed at FAST. The much higher current predicted by the kappa J-V
  relation for extremely nonthermal $\kappa$ values compensates for the reduced
  number density. (2) The source population is nearby ($<$~2.2~$R_E$) so that
  the much higher number density compensates for current saturation effects.

  The distinction to be borne in mind is that according to
  $\chi^2_{\mathrm{red}}$ values shown in the top panel of \ref{ch6:Fig3} a
  highly nonthermal population is compatible with either low
  ($\lesssim$~2~$R_E$) or high ($\gtrsim$~4.7~$R_E$) source altitudes, but a
  source in thermal equilibrium is only compatible with a nearby source
  region. Importantly, the electron distributions in Figure~\ref{ch6:Fig2}
  exhibit a very pronounced high-energy tail, which argues strongly against a
  source in thermal equilibrium and therefore against the scenario involving a
  nearby ($<$~2.2~$R_E$) source in thermal equilbrium.

  If the preacceleration scenario described in section~\ref{ssSourceAlt} is
  assumed to be true for the precipitating electrons observed during the period
  of interest in Figure~\ref{ch6:Fig1}, with $E_p \simeq 850$~V and
  $T =$~600~eV, one obtains $\phi_T = 19^\circ$ and therefore $R_{B,T} = 9$. The
  abscissa of the top panel in Figure~\ref{ch6:Fig3} shows that $R_{B,T} = 9$
  corresponds to a minimum source altitude $R_E \approx 2$.
  % \simeq 1.3$~kV and $T =$~600~eV, $\phi_T = 13^\circ$ and $R_{B,T} =
  % 19.8$.
  The shell-like distributions observed throughout the period of interest in
  Figure~\ref{ch6:Fig1} therefore suggest that the source region is located at
  altitudes near or above 2.2~$R_E$.
  This piece of evidence is consistent with the scenario involving a highly
  nonthermal source at or above~4.7~$R_E$,
  and mostly inconsistent with the scenario involving a thermal source
  population below 2.2~$R_E$.

  In summary, the observed highly nonthermal electron distributions
  (Figures~\ref{ch6:Fig1} and \ref{ch6:Fig2}), which show $\kappa
  \simeq$~1.7, and the shell-like morphologies of the observed electron
  distributions, which appear to indicate a minimum source altitude $R_E
  \approx 2$, are most consistent with a source region near or above
  4.7~$R_E$ (top panel of Figure~\ref{ch6:Fig3}).

  % A possible objection to this interpretation centers on secondary
  % electron production (cite cite). A more conservative estimate is
  % that after mirroring (as has been assumed for
  % Figure~\ref{ch6:Fig2}b), $\phi_T =$~30$^\circ$ and therefore $R_{B,T}
  % =$~4.9.

  % \dots

  % I conclude that the kappa scenario seems a little more likely.

  % \dots

% but the
%   relative absence of electrons over more than a decade of energies
%   between the thermal population below $\sim$20~eV and the
%   monoenergetic population above several hundred~eV in
%   Figure~\ref{ch6:Fig1}a
  
  \section{FAST Orbit 1843}

  % ---------------- FIGURE 4

  \begin{figure}
    \centering
    % \noindent\includegraphics[width=0.9\textwidth]{./ch6/figs/Kappa_summary--1843-classics-1-Ergun_et_al_1998}
    \noindent\includegraphics[width=0.9\textwidth]{./ch6/figs/Kappa_summary--1843-classics-1-Ergun_et_al_1998__v2__itvl3integ}
    \caption[Inverted V precipitation and best-fit Maxwellian and kappa
    distribution parameters (Orbit 1843)]{EESA observations of inverted V
      precipitation on Feb 8, 1997, and corresponding 2-D fit parameters. (a)
      $>$300~eV electron pitch-angle distribution. (b) Average electron energy spectrum
      within the precipitating loss cone. (c) $\kappa$ fit parameter for the
      best-fit kappa distribution. (d) Reduced chi-squared statistic
      $\chi^2_{\mathrm{red}}$ for each fit type. (e) Observed (black diamonds)
      and best-fit temperatures. (f) Observed (black diamonds) and best-fit
      densities. This time interval has also been studied by
      \citet{Ergun1998a,Ergun1998}.}
    \label{ch6:Fig4}
  \end{figure}

  % ----------------

  Figure~\ref{ch6:Fig4} shows EESA observations on Feb 7, 1997 during FAST orbit
  1843, in the same layout as Figure~\ref{ch6:Fig1}. These observations have
  also been reported by \citet{Ergun1998a,Ergun1998}. While those authors used
  burst-mode EESA observations, we present survey-mode EESA observations
  integrated over three times the survey-mode EESA sampling frequency in order
  to decrease the uncertainty related to counting statistics in the calculated
  temperatures and densities (Appendix~\ref{app:A}).

  \citet{Ergun1998} used observations of auroral kilometric radiation (AKR) made
  by the FAST fields instrument \citep{Ergun2001} to conclude that the source
  region for the precipitating electrons in Figure~\ref{ch6:Fig4} lies ``near
  the apogee of FAST'' at approximately 4200~km. We may compare this information
  about the altitude of the source population with the results of an analysis of
  orbit 1843 that parallels the foregoing analysis of orbit 1773.

  During the period 20:49:48--20:50:11~UT that is shown between red bars, the
  observed properties of the electrons within the precipitating loss cone are
  $\kappa \simeq $~1.55--2.5, $T_F =$~2--3~keV, and $n_F =$~0.7--1.0~cm$^{-3}$.
  Best-fit parameters shown in Figure suggest that $\kappa \gtrsim 2$ over much of this period,
  in contrast to the more extreme best-fit values $\kappa \lesssim 1.7$ during
  the marked period in Figure~\ref{ch6:Fig1}c.

  % ---------------- FIGURE 5

  \begin{figure}
    \centering
    \noindent\includegraphics[width=0.95\textwidth,angle=90]{./ch6/figs/orb_1843-Kappa_fit-20_50_08__87}
    \caption[Example two-dimensional fit of inverted V precipitation (Orbit
    1843)]{Observed 2-D differential energy flux spectrum at 20:50:08.87~UT
      (contour lines) overlaying the best-fit 2-D kappa distribution (solid
      contours).}
    \label{ch6:Fig5}
  \end{figure}

  % ----------------

  Figure~\ref{ch6:Fig5} shows a typical example of the 2-D electron
  distributions observed during this period. The observed distribution (contour
  lines) overlays the best-fit 2-D kappa distribution (solid contours). The
  assumption $E_p > T$ underpinning the pitch-angle spreading argument (section
  \ref{ssSourceAlt}) is valid since $E_p =$~5.25~keV (legend in
  Figure~\ref{ch6:Fig5}) and $T \simeq$~2.8~keV (Figure~\ref{ch6:Fig4}e). The
  mirror ratio $R_{B,T} = 22$ (or a source altitude of approximately 2.5~$R_E$)
  is accordingly that for which a preaccelerated source distribution would
  appear approximately isotropic. Since differential energy flux drops off
  somewhat rapidly with increasing pitch angle in the lower panel of
  Figure~\ref{ch6:Fig5} (i.e., the distribution is not isotropic) we assume the
  mirror ratio between the source region and FAST is $R_B \lesssim R_{B,T}$, or
  equivalently that the source altitude is near or below 2.5~$R_E$.

  % ---------------- FIGURE 6

  \begin{figure}
    \centering
    % \noindent\includegraphics[width=0.9\textwidth]{./ch6/figs/orb_1843__kappa_RB_map__20170422}
    % \noindent\includegraphics[width=0.9\textwidth]{./ch6/figs/orb_1843__map_linear__bestfit__20170422}
    \noindent\includegraphics[width=0.9\textwidth]{./ch6/figs/orb_1843__kappa_RB_map__20170605__v2}
    \noindent\includegraphics[width=0.9\textwidth]{./ch6/figs/orb_1843__map_linear__bestfit__20170605__v2}
    % \caption[Orbit 1843: Reduced chi-squared values for observed J-V curves and
    % corresponding best-fit J-V curves.]{Results of J-V curve fits using
    %   equations \ref{ch6:eqKnight} and \ref{ch6:eqDors}. The top panel shows
    %   reduced chi-squared values for the J-V relationship observed between
    %   20:49:48 and 20:50:11~UT in Figure~\ref{ch6:Fig4}, as in
    %   Figure~\ref{ch6:Fig3}. The bottom panel shows the observed J-V relation
    %   and corresponding best-fit Maxwellian and kappa J-V relations.}
    \caption[Orbit 1843: Reduced chi-squared values for observed J-V curves and
    corresponding best-fit J-V curves.]{The top panel shows
      $\chi^2_{\mathrm{red}} ( \kappa, R_B)$ for the observed J-V relationship
      between 20:49:48 and 20:50:11~UT in Figure~\ref{ch6:Fig4}, with $T_m$ held
      fixed and magnetospheric density $n_m \big ( n_F, \bar{\phi}, \alpha_F
      \big )$, $\kappa$ and mirror ratio $R_B$ allowed to vary. The bottom panel
      shows the observed J-V relationship and corresponding best-fit Maxwellian
      and kappa J-V relations.}
    \label{ch6:Fig6}
  \end{figure}

  % ----------------
  
  Again following the procedure outlined in section~\ref{ssJV}, we assume
  $T_m = \bar{T}_F =$~2791.3~eV and form a set of of observed J-V pairs
  $( j_{\parallel,i} , \Delta \Phi )$ from mapped current densities and
  potential drops during this period. The top panel of Figure~\ref{ch6:Fig6}
  displays contours of $\chi^2_{\textrm{red}} ( \kappa, R_B )$ in the same
  format as the top panel of Figure~\ref{ch6:Fig3}. Two scenarios appear most
  likely: (1) an extremely nonthermal source at or above $\sim$3.7~$R_E$, and
  (2) a thermal or perhaps slightly nonthermal source at or below
  $\sim$2.6~$R_E$. 

  Because $\kappa \gtrsim$~2 is frequently observed during the period of
  interest in Figure~\ref{ch6:Fig4}, and because Figure~\ref{ch6:Fig5} argues
  that $R_B$ cannot exceed $\sim$22, the overall likelihood of the first
  scenario involving a high-altitude nonthermal source is remote. Instead, the
  balance of the evidence in this case is in favor of a weakly nonthermal source
  population near or below 2.5~$R_E$. This conclusion about the source altitude
  is compatible with the work of \citet{Ergun1998} discussed at the beginning of
  this section.
  
%%%%%%%%%%%%%%%%%%%%%%%%%%%%
  \section{Discussion and Conclusions}
%%%%%%%%%%%%%%%%%%%%%%%%%%%%

  We have performed analyses of orbits 1773 and 1843 in parallel fashion. In the
  former case we conclude that the evidence is in favor of a highly nonthermal
  source with $\kappa < 1.7$ that is near or above 4.7~$R_E$. In the latter case
  we conclude that the balance of the evidence is in favor of a source that is
  nonthermal (though not as extreme as in orbit 1773) with $\kappa \simeq 2$, at
  an altitude not exceeding $\sim$2.5~$R_E$ and reportedly in connection with
  AKR \citep{Ergun1998}.

  The results of this study suggest that a nonthermal source population may
  significantly modify the Knight relation. Given the shell-like appearance of
  precipitating electron distributions during the 11-second period marked in
  Figure~\ref{ch6:Fig1}, which we interpret as evidence that the source altitude
  can be no less than $>$~2~$R_E$, the results in Figure~\ref{ch6:Fig3}a in
  particular indicate that the observed J-V relation is only compatible with a
  highly nonthermal source population located at or above~4.7~$R_E$. Further
  study is clearly necessary to determine both the conditions leading to
  development of highly nonthermal distributions, as well as the source
  altitude.

  Regarding the occurrence frequency of highly nonthermal populations, we have
  conducted a preliminary examination of previously reported FAST observations
  of inverted V precipitation
  \citep{McFadden1998a,Carlson2001,Janhunen2001,Dombeck2013}, which suggest
  geomagnetically quiet periods favor production of ``far-equilibrium''
  ($\kappa \leq$~2.5) distributions. However, recent statistical work with DMSP
  satellites performed by \citet{McIntosh2014} may represent counterevidence,
  since they show that diffuse precipitation that is best fit with a kappa
  distribution occurs more regularly during active periods \citep[e.g., Figure~7
  in][]{McIntosh2014}. Because \citet{McIntosh2014} only distinguish between
  Maxwellian and kappa distributions for diffuse (as opposed to monoenergetic)
  precipitation, whether the same trend away from thermal equilibrium with
  increasing geomagnetic activity exists for monoenergetic precipitation remains
  to be seen.

  Results from analysis of orbit 1773 could represent the low-altitude
  counterpart of observations reported by \citet{Wygant2002} and
  \citet{Schriver2003}, where the former have stated ``in addition to the
  auroral particle energization processes known to occur at altitudes between
  0.5 and 2~$R_E$, there are important heating and acceleration mechanisms
  operating at these higher altitudes in the plasma sheet.''

  Previous studies of precipitating electrons in or below the
  magnetosphere-ionosphere transition region and involving kappa distributions
  \citep{Olsson1998,Ogasawara2006,Kaeppler2014a} have reported $\kappa
  \geq$~2.5, values which are within the ``near-equilibrium'' regime
  \citep{Livadiotis2010}, meaning the corresponding distributions do not deviate
  strongly from a Maxwellian distribution. At higher altitudes, in situ
  magnetosphere and plasma sheet observations
  \citet{Christon1989,Christon1991,Kletzing2003} show evidence of $\kappa <$~4,
  but these were not discussed.

  Passing mention is owed to some existing reports in the literature showing
  $\kappa < 1.5$ and even $\kappa \simeq$~0, but because the three-dimensional
  theoretical minimum $\kappa_{\mathrm{min}} =$~3/2 corresponds to perfectly
  correlated ($r =$~1) particle motion, how these results were obtained is
  unclear.

  Regarding methodology, we have performed the entire analyses of orbits 1773
  and 1843 using a version of the procedure described in section~\ref{ssJV} that
  is modified in two ways to obtain ``local'' J-V pairs that rely only on
  electron observations, and neglect supplementary information available from
  ion observations and IGRF~11. These modifications are (1) Information about
  the upgoing ion beam is discarded so that the potential drop
  $\Delta \Phi = \bar{E}_e$ instead of $\Delta \Phi = \bar{E}_e + \bar{E}_i$;
  (2) Current densities calculated from the observed electron distributions are
  calculated over all pitch angles (instead of restricting the pitch-angle range
  to those within the earthward loss cone), and are not mapped to 100~km. We
  briefly summarize the results of this alternative analysis.

  The first issue we have encountered in this alternative analysis is 
  significantly higher uncertainty in the calculated current density due to the
  inclusion of counting statistics from energy-angle bins at large pitch angles
  (Appendix~\ref{app:A}). The second issue is
  !!!FINISH APPENDIX

  In conclusion, the importance of the observations presented here is that they
  demonstrate the existence of highly nonthermal ($\kappa \simeq$~1.6)
  monoenergetic precipitation within the auroral acceleration region, clear from
  contamination by secondary electrons and other thermalization processes. As
  attention turns increasingly to dynamic M-I coupling, these observations
  suggest that the effects of deviation from thermal equilibrium (as
  parameterized by $\kappa$ in the present work) may not be ignorable either in
  the development of first principles--based models of auroral precipitation, or
  in the study and interpretation of magnetospheric energization and
  acceleration processes. A comprehensive study of the conditions that give rise
  to such observations is intended for future work.

% \textbf{CHRIS: You could emphasize the basic difference of your
%     observations here relative to those done previously. Some of which were in
%     the collisional topside ionosphere. Your work is the first to be performed
%     in the acceleration region - clear from contamination of secondaries etc and
%     collisional or other thermalization processes. This is perhaps why you see
%     much lower Kappa values.  Your values are also lower than in the PS reported
%     by Kletzing and others - Well you have much better energy time and pitch
%     angle resolution than they had - BUT - the difference may be due to the
%     action of additional processes in the accel region. You alluded to this
%     indirectly above but you could emphasize that this may be why you get lower
%     Kappa than kletzing et al.  This is really nice work Spencer. I had to think
%     hard to keep up with the text you on this. }


  % and recent theoretical work \citep{Khazanov2015,Khazanov2016} indicates that
  % magnetosphere-ionosphere coupling, wave-particle interactions, and multiple
  % atmospheric reflections are all necessary ingredients for producing the
  % observed distributions.
  
  \section*{Acknowledgments}

  We are grateful to Craig Markwardt for making publicly available an
  Interactive Data Language version of the MINPACK-1 fitting routines, which we
  have used extensively in this research. Work at Dartmouth College is supported
  by NASA Headquarters under the NASA Earth and Space Science Fellowship
  Program--Grant NNX14AO03H, and at Space Sciences Laboratory by NASA grants
  NNX15AF57G and NNX16AG69G.

\bibliographystyle{agufull08}
\bibliography{refs6}
